
\PassOptionsToPackage{unicode}{hyperref}
\PassOptionsToPackage{hyphens}{url}
\PassOptionsToPackage{usenames,dvipsnames}{xcolor}


\documentclass[]{article}
\usepackage{lmodern}

\usepackage{amsmath, amsthm, amssymb, amsfonts, amsxtra, amscd, mathtools}

\usepackage{tikz}
\usetikzlibrary{cd}

\usepackage{ifxetex,ifluatex}

\usepackage{bookmark}
\usepackage{hyperref}
\usepackage{xurl}
\urlstyle{same} % disable monospaced font for URLs

\hypersetup{
    pdftitle={UGA Real Analysis Qualifying Exam Questions}
    colorlinks,
    linktoc=all,    
    citecolor=blue,
    filecolor=blue,
    linkcolor=blue,
    urlcolor=blue
}
\ifnum 0\ifxetex 1\fi\ifluatex 1\fi=0 % if pdftex
  \usepackage[T1]{fontenc}
  \usepackage[utf8]{inputenc}
  \usepackage{textcomp} % provide euro and other symbols
\else % if luatex or xetex
  \usepackage{unicode-math}
  \defaultfontfeatures{Scale=MatchLowercase}
  \defaultfontfeatures[\rmfamily]{Ligatures=TeX,Scale=1}
\fi

% Use upquote if available, for straight quotes in verbatim environments
\IfFileExists{upquote.sty}{\usepackage{upquote}}{}
\IfFileExists{microtype.sty}{% use microtype if available
  \usepackage[]{microtype}
  \UseMicrotypeSet[protrusion]{basicmath} % disable protrusion for tt fonts
}{}

\makeatletter
\@ifundefined{KOMAClassName}{% if non-KOMA class
  \IfFileExists{parskip.sty}{%
    \usepackage{parskip}
  }{% else
    \setlength{\parindent}{0pt}
    \setlength{\parskip}{6pt plus 2pt minus 1pt}}
}{% if KOMA class
  \KOMAoptions{parskip=half}}
\makeatother

\setlength{\emergencystretch}{3em} % prevent overfull lines
\providecommand{\tightlist}{%
  \setlength{\itemsep}{0pt}\setlength{\parskip}{0pt}
}

\setcounter{secnumdepth}{-\maxdimen} % remove section numbering

% Highlight quote
\usepackage{environ}
\colorlet{block-gray}{SpringGreen}
\NewEnviron{myblock}
{\colorbox{block-gray}{%
\parbox{\dimexpr\linewidth-2\fboxsep\relax}{%
\small\addtolength{\leftskip}{10mm}
\addtolength{\rightskip}{10mm}
\BODY}}
}

\renewcommand{\quote}{\myblock}
\renewcommand{\endquote}{\endmyblock}


%%%%%%%%%%%%%%%%%%%%%%%%%%%%%%%%%%%%%%%%%%
\newtheoremstyle{break}% name
  {}%         Space above, empty = `usual value'
  {2em}%         Space below
  {\normalfont}% Body font
  {}%         Indent amount (empty = no indent, \parindent = para indent)
  {\bfseries}% Thm head font
  {.}%        Punctuation after thm head
  {\newline}% Space after thm head: \newline = linebreak
  {}%         Thm head spec
  
\theoremstyle{break}
\newtheorem{problem}{Problem}
\theoremstyle{definition}
\newtheorem{solution}{Solution}

\newcommand{\RR}[0]{{\mathbb{R}}}
\newcommand{\CC}[0]{{\mathbb{C}}}
\newcommand{\QQ}[0]{{\mathbb{Q}}}
\newcommand{\NN}[0]{{\mathbb{N}}}
\newcommand{\ZZ}[0]{{\mathbb{Z}}}

\newcommand{\abs}[1]{{\left\lvert {#1} \right\rvert}}
\newcommand{\inv}[0]{^{-1}}

\newcommand{\suchthat}[0]{{~\mid- ~}}
\newcommand{\theset}[1]{\left\{{#1}\right\}}
\newcommand{\norm}[1]{{\left\lVert {#1} \right\rVert}}
\newcommand{\wait}[0]{{\,\cdot\,}}
\newcommand{\inner}[2]{{\left\langle {#1},~{#2} \right\rangle}}

\everymath{\displaystyle}
%%%%%%%%%%%%%%%%%%%%%%%%%%%%%%%%%%%%%%%%%%
\synctex=1

\usepackage{environ}
\NewEnviron{killcontents}{}
\let\proof\killcontents
\let\endproof\endkillcontents

\title{UGA Real Analysis Qualifying Exams}
\date{}

\begin{document}
\maketitle
\tableofcontents 
\newpage

\hypertarget{fall-2019}{%
\section{Fall 2019}\label{fall-2019}}

\hypertarget{section}{%
\subsection{1.}\label{section}}

Let \(\{a_n\}_{n=1}^\infty\) be a sequence of real numbers.

\begin{enumerate}
\def\labelenumi{\alph{enumi}.}
\item
  Prove that if \(\displaystyle\lim_{n\to\infty} a_n = 0\), then
  \[
  \lim _{n \rightarrow \infty} \frac{a_{1}+\cdots+a_{n}}{n}=0
  \]
\item
  Prove that if \(\displaystyle\sum_{n=1}^{\infty} \frac{a_{n}}{n}\)
  converges, then \[
  \lim _{n \rightarrow \infty} \frac{a_{1}+\cdots+a_{n}}{n}=0
  \]
\end{enumerate}

\begin{proof}

\end{proof}
%%%%%%%%%%%%%%%%%%%%%%%%%%%%%

\hypertarget{section-1}{%
\subsection{2.}\label{section-1}}

Prove that \[
\left|\frac{d^{n}}{d x^{n}} \frac{\sin x}{x}\right| \leq \frac{1}{n}
\]

for all \(x \neq 0\) and positive integers \(n\).

\begin{quote}
Hint: Consider \(\displaystyle\int_0^1 \cos(tx) dt\)
\end{quote}

\begin{proof}

\end{proof}
%%%%%%%%%%%%%%%%%%%%%%%%%%%%%

\hypertarget{section-2}{%
\subsection{3.}\label{section-2}}

Let \(( X, \mathcal B, \mu )\) be a measure space with \(\mu(X) = 1\) and
\(\{B_n\}_{n=1}^\infty\) be a sequence of \(\mathcal B\)-measurable
subsets of \(X\), and \[
B \coloneqq \theset{x\in X \mid x\in B_n \text{ for infinitely many } n}.
\]

\begin{enumerate}
\def\labelenumi{\alph{enumi}.}
\item
  Argue that \(B\) is also a \(\mathcal{B} -\)measurable subset of
  \(X\).
\item
  Prove that if \(\sum_{n=1}^\infty \mu(B_n) < \infty\) then
  \(\mu(B)= 0\).
\item
  Prove that if \(\sum_{n=1}^\infty \mu(B_n) = \infty\) \textbf{and} the
  sequence of set complements \(\theset{B_n^c}_{n=1}^\infty\) satisfies
  \[
  \mu\left(\bigcap_{n=k}^{K} B_{n}^{c}\right)=\prod_{n=k}^{K}\left(1-\mu\left(B_{n}\right)\right)
  \] for all positive integers \(k\) and \(K\) with \(k < K\), then
  \(\mu(B) = 1\).
\end{enumerate}

\begin{quote}
Hint: Use the fact that \(1 - x \leq e^{-x}\) for all \(x\).
\end{quote}

\begin{proof}

\end{proof}
%%%%%%%%%%%%%%%%%%%%%%%%%%%%%

\hypertarget{section-3}{%
\subsection{4.}\label{section-3}}

Let \(\{u_n\}_{n=1}^\infty\) be an orthonormal sequence in a Hilbert space
\(\mathcal{H}\).

\begin{enumerate}
\def\labelenumi{\alph{enumi}.}
\item
  Prove that for every \(x \in \mathcal H\) one has \[
  \displaystyle\sum_{n=1}^{\infty}\left|\left\langle x, u_{n}\right\rangle\right|^{2} \leq\|x\|^{2}
  \]
\item
  Prove that for any sequence \(\{a_n\}_{n=1}^\infty \in \ell^2(\NN)\)
  there exists an element \(x\in\mathcal H\) such that \[
    a_n = \inner{x}{u_n} \text{ for all } n\in \NN
    \] and \[
    \norm{x}^2 = \sum_{n=1}^{\infty}\left|\left\langle x, u_{n}\right\rangle\right|^{2}
    \]
\end{enumerate}

\begin{proof}

\end{proof}
%%%%%%%%%%%%%%%%%%%%%%%%%%%%%

\hypertarget{section-4}{%
\subsection{5.}\label{section-4}}

\begin{enumerate}
\def\labelenumi{\alph{enumi}.}
\item
  Show that if \(f\) is continuous with compact support on \(\RR\), then
  \[
  \lim _{y \rightarrow 0} \int_{\mathbb{R}}|f(x-y)-f(x)| d x=0
  \]
\item
  Let \(f\in L^1(\RR)\) and for each \(h > 0\) let \[
  \mathcal{A}_{h} f(x):=\frac{1}{2 h} \int_{|y| \leq h} f(x-y) d y
  \]
\end{enumerate}

\begin{enumerate}
\def\labelenumi{\roman{enumi}.}
\setcounter{enumi}{0}
\tightlist
\item
  Prove that \(\left\|\mathcal{A}_{h} f\right\|_{1} \leq\|f\|_{1}\) for
  all \(h > 0\).
\item
  Prove that \(\mathcal{A}_h f \to f\) in \(L^1(\RR)\) as \(h \to 0^+\).
\end{enumerate}

\begin{proof}

\end{proof}
%%%%%%%%%%%%%%%%%%%%%%%%%%%%%

\hypertarget{spring-2019}{%
\section{Spring 2019}\label{spring-2019}}

\hypertarget{section}{%
\subsection{1}\label{section}}

Let \(C([0, 1])\) denote the space of all continuous real-valued
functions on \([0, 1]\).

\begin{enumerate}
\def\labelenumi{\alph{enumi}.}
\tightlist
\item
  Prove that \(C([0, 1])\) is complete under the uniform norm
  \(\norm{f}_u := \displaystyle\sup_{x\in [0,1]} |f (x)|\).
\item
  Prove that \(C([0, 1])\) is not complete under the \(L^1-\)norm
  \(\norm{f}_1 = \displaystyle\int_0^1 |f (x)| ~dx\).
\end{enumerate}

\hypertarget{section-1}{%
\subsection{2}\label{section-1}}

Let \(\mathcal B\) denote the set of all Borel subsets of \(\RR\) and
\(\mu : \mathcal B \to [0, \infty)\) denote a finite Borel measure on
\(\RR\).

\begin{enumerate}
\def\labelenumi{\alph{enumi}.}
\tightlist
\item
  Prove that if \(\{F_k\}\) is a sequence of Borel sets for which
  \(F_k \supseteq F_{k+1}\) for all \(k\), then \[
    \lim _{k \rightarrow \infty} \mu\left(F_{k}\right)=\mu\left(\bigcap_{k=1}^{\infty} F_{k}\right)
    \]
\item
  Suppose \(\mu\) has the property that \(\mu(E) = 0\) for every
  \(E \in \mathcal B\) with Lebesgue measure \(m(E) = 0\). Prove that
  for every \(\varepsilon > 0\) there exists \(\delta > 0\) so that if
  \(E \in \mathcal B\) with \(m(E) < \delta\), then \(\mu(E) < \varepsilon\).
\end{enumerate}

\hypertarget{section-2}{%
\subsection{3}\label{section-2}}

Let \(\{f_k\}\) be any sequence of functions in \(L^2([0, 1])\)
satisfying \(\norm{f_k}_2 \leq M\) for all \(k \in \NN\).

Prove that if \(f_k \to f\) almost everywhere, then \(f \in L^2([0, 1])\)
with \(\norm{f}_2 \leq M\) and \[
\lim _{k \rightarrow \infty} \int_{0}^{1} f_{k}(x) dx = \int_{0}^{1} f(x) d x
\]

\begin{quote}
Hint: Try using Fatou's Lemma to show that \(\norm{f}_2 \leq M\) and then
try applying Egorov's Theorem.
\end{quote}

\hypertarget{section-3}{%
\subsection{4}\label{section-3}}

Let \(f\) be a non-negative function on \(\RR^n\) and
\(\mathcal A = \{(x, t) \in \RR^n \times \RR : 0 \leq t \leq f (x)\}\).

Prove the validity of the following two statements:

\begin{enumerate}
\def\labelenumi{\alph{enumi}.}
\item
  \(f\) is a Lebesgue measurable function on \(\RR^n \iff \mathcal A\)
  is a Lebesgue measurable subset of \(\RR^{n+1}\)
\item
  If \(f\) is a Lebesgue measurable function on \(\RR^n\), then \[
    m(\mathcal{A})=\int_{\mathbb{R}^{n}} f(x) d x=\int_{0}^{\infty} m\left(\left\{x \in \mathbb{R}^{n}: f(x) \geq t\right\}\right) d t
    \]
\end{enumerate}

\hypertarget{section-4}{%
\subsection{5}\label{section-4}}

\begin{enumerate}
\def\labelenumi{\alph{enumi}.}
\item
  Show that \(L^2([0, 1]) \subseteq L^1([0, 1])\) and argue that \(L^2([0, 1])\)
  in fact forms a dense subset of \(L^1([0, 1])\).
\item
  Let \(\Lambda\) be a continuous linear functional on \(L^1([0, 1])\).

  Prove the Riesz Representation Theorem for \(L^1([0, 1])\) by
  following the steps below:

  \begin{enumerate}
  \def\labelenumii{\roman{enumii}.}
  \tightlist
  \item
    Establish the existence of a function \(g \in L^2([0, 1])\) which
    represents \(\Lambda\) in the sense that \[
      \Lambda(f ) = \int_0^1 f(x) \overline{g(x)} dx \text{ for all } f \in L^2([0, 1]).
    \]
  \end{enumerate}

  \begin{quote}
  Hint: You may use, without proof, the Riesz Representation Theorem for
  \(L^2([0, 1])\).
  \end{quote}

  \begin{enumerate}
  \def\labelenumii{\roman{enumii}.}
  \setcounter{enumii}{1}
  \tightlist
  \item
    Argue that the \(g\) obtained above must in fact belong to
    \(L^\infty([0, 1])\) and represent \(\Lambda\) in the sense that \[
    \Lambda(f)=\int_{0}^{1} f(x) \overline{g(x)} d x \quad \text { for all } f \in L^{1}([0,1])
    \] with \[
    \|g\|_{L^{\infty}([0,1])}=\|\Lambda\|_{L^{1}([0,1])^*}
    \]
  \end{enumerate}
\end{enumerate}

\hypertarget{fall-2018}{%
\section{Fall 2018}\label{fall-2018}}

\hypertarget{section}{%
\subsection{1}\label{section}}

Let \(f(x) = \frac 1 x\). Show that \(f\) is uniformly continuous on
\((1, \infty)\) but not on \((0,\infty)\).

\hypertarget{section-1}{%
\subsection{2}\label{section-1}}

Let \(E\subset \RR\) be a Lebesgue measurable set. Show that there is a
Borel set \(B \subset E\) such that \(m(E\setminus B) = 0\).

\hypertarget{section-2}{%
\subsection{3}\label{section-2}}

Suppose \(f(x)\) and \(xf(x)\) are integrable on \(\RR\). Define \(F\)
by \[
F(t):=\int_{-\infty}^{\infty} f(x) \cos (x t) d x
\] Show that \[
F^{\prime}(t)=-\int_{-\infty}^{\infty} x f(x) \sin (x t) d x.
\]

\hypertarget{section-3}{%
\subsection{4}\label{section-3}}

Let \(f\in L^1([0, 1])\). Prove that \[
\lim _{n \rightarrow \infty} \int_{0}^{1} f(x)|\sin n x| d x=\frac{2}{\pi} \int_{0}^{1} f(x) d x
\]

\begin{quote}
Hint: Begin with the case that \(f\) is the characteristic function of
an interval.
\end{quote}

\hypertarget{section-4}{%
\subsection{5}\label{section-4}}

Let \(f \geq 0\) be a measurable function on \(\RR\). Show that \[
\int_{\mathbb{R}} f=\int_{0}^{\infty} m(\{x: f(x)>t\}) d t
\]

\hypertarget{section-5}{%
\subsection{6}\label{section-5}}

Compute the following limit and justify your calculations: \[
\lim _{n \rightarrow \infty} \int_{1}^{n} \frac{d x}{\left(1+\frac{x}{n}\right)^{n} \sqrt[n]{x}}
\]

\hypertarget{spring-2018}{%
\section{Spring 2018}\label{spring-2018}}

\hypertarget{section}{%
\subsection{1}\label{section}}

Define \[
E:=\left\{x \in \mathbb{R}:\left|x-\frac{p}{q}\right|<q^{-3} \text { for infinitely many } p, q \in \mathbb{N}\right\}.
\]

Prove that \(m(E) = 0\).

\hypertarget{section-1}{%
\subsection{2}\label{section-1}}

Let \[
f_{n}(x):=\frac{x}{1+x^{n}}, \quad x \geq 0.
\]

\begin{enumerate}
\def\labelenumi{\alph{enumi}.}
\item
  Show that this sequence converges pointwise and find its limit. Is the
  convergence uniform on \([0, \infty)\)?
\item
  Compute \[
  \lim _{n \rightarrow \infty} \int_{0}^{\infty} f_{n}(x) d x
  \]
\end{enumerate}

\hypertarget{section-2}{%
\subsection{3}\label{section-2}}

Let \(f\) be a non-negative measurable function on \([0, 1]\).

Show that \[
\lim _{p \rightarrow \infty}\left(\int_{[0,1]} f(x)^{p} d x\right)^{\frac{1}{p}}=\|f\|_{\infty}.
\]

\hypertarget{section-3}{%
\subsection{4}\label{section-3}}

Let \(f\in L^2([0, 1])\) and suppose \[
\int_{[0,1]} f(x) x^{n} d x=0 \text { for all integers } n \geq 0.
\] Show that \(f = 0\) almost everywhere.

\hypertarget{section-4}{%
\subsection{5}\label{section-4}}

Suppose that

\begin{itemize}
\tightlist
\item
  \(f_n, f \in L^1\),
\item
  \(f_n \to f\) almost everywhere, and
\item
  \(\int\left|f_{n}\right| \rightarrow \int|f|\).
\end{itemize}

Show that \(\int f_{n} \rightarrow \int f\)

\hypertarget{fall-2017}{%
\section{Fall 2017}\label{fall-2017}}

\hypertarget{section}{%
\subsection{1}\label{section}}

Let \[
f(x) = \sum_{n=0}^{\infty} \frac{x^{n}}{n !}.
\]

Describe the intervals on which \(f\) does and does not converge
uniformly.

\hypertarget{section-1}{%
\subsection{2}\label{section-1}}

Let \(f(x) = x^2\) and \(E \subset [0, \infty) \coloneqq \RR^+\).

\begin{enumerate}
\def\labelenumi{\arabic{enumi}.}
\item
  Show that \[
  m^*(E) = 0 \iff m^*(f(E)) = 0.
  \]
\item
  Deduce that the map
\end{enumerate}

\begin{align*}
\phi: \mathcal{L}(\RR^+) &\to \mathcal{L}(\RR^+) \\
E &\mapsto f(E)
\end{align*} is a bijection from the class of Lebesgue measurable sets
of \([0, \infty)\) to itself.

\hypertarget{section-2}{%
\subsection{3}\label{section-2}}

Let \[
S = \mathrm{span}_\CC\theset{\chi_{(a, b)} \mid a, b \in \RR},
\] the complex linear span of characteristic functions of intervals of
the form \((a, b)\).

Show that for every \(f\in L^1(\RR)\), there exists a sequence of
functions \(\theset{f_n} \subset S\) such that \[
\lim _{n \rightarrow \infty}\left\|f_{n}-f\right\|_{1}=0
\]

\hypertarget{section-3}{%
\subsection{4}\label{section-3}}

Let \[
f_{n}(x)=n x(1-x)^{n}, \quad n \in \mathbb{N}.
\]

\begin{enumerate}
\def\labelenumi{\arabic{enumi}.}
\item
  Show that \(f_n \to 0\) pointwise but not uniformly on \([0, 1]\).

  \begin{quote}
  Hint: Consider the maximum of \(f_n\).
  \end{quote}
\item
  \[
    \lim _{n \rightarrow \infty} \int_{0}^{1} n(1-x)^{n} \sin(x)~dx=0
  \]
\end{enumerate}

\hypertarget{section-4}{%
\subsection{5}\label{section-4}}

Let \(\phi\) be a compactly supported smooth function that vanishes
outside of an interval \([-N, N]\) such that
\(\int_{\RR} \phi(x) d x=1\).

For \(f\in L^1(\RR)\), define \[
K_{j}(x):=j \phi(j x), \quad \quad f \ast K_{j}(x):=\int_{\mathbb{R}} f(x-y) K_{j}(y)~dy
\] and prove the following:

\begin{enumerate}
\def\labelenumi{\arabic{enumi}.}
\item
  Each \(f\ast K_j\) is smooth and compactly supported.
\item
  \[
  \lim _{j \rightarrow \infty}\left\|f * K_{j}-f\right\|_{1}=0
  \]
\end{enumerate}

\begin{quote}
Hint: \[
\lim _{y \rightarrow 0} \int_{\mathbb{R}}|f(x-y)-f(x)| d y=0
\]
\end{quote}

\hypertarget{section-5}{%
\subsection{6}\label{section-5}}

Let \(X\) be a complete metric space and define a norm \[
\|f\|:=\max \{|f(x)|: x \in X\}.
\]

Show that \((C^0(\RR), \norm{\wait} )\) (the space of continuous
functions \(f: X\to \RR\)) is complete.

\hypertarget{spring-2017}{%
\section{Spring 2017}\label{spring-2017}}

\hypertarget{section}{%
\subsection{1}\label{section}}

Let \(K\) be the set of numbers in \([0, 1]\) whose decimal expansions
do not use the digit \(4\).

\begin{quote}
We use the convention that when a decimal number ends with 4 but all
other digits are different from 4, we replace the digit \(4\) with
\(399\cdots\). For example, \(0.8754 = 0.8753999\cdots\).
\end{quote}

Show that \(K\) is a compact, nowhere dense set without isolated points,
and find the Lebesgue measure \(m(K)\).

\hypertarget{section-1}{%
\subsection{2}\label{section-1}}

\begin{enumerate}
\def\labelenumi{\alph{enumi}.}
\item
  Let \(\mu\) be a measure on a measurable space \((X, \mathcal M)\) and
  \(f\) a positive measurable function.

  Define a measure \(\lambda\) by \[
  \lambda(E):=\int_{E} f ~d \mu, \quad E \in \mathcal{M}
  \]

  Show that for \(g\) any positive measurable function, \[
  \int_{X} g ~d \lambda=\int_{X} f g ~d \mu
  \]
\item
  Let \(E \subset \RR\) be a measurable set such that \[
  \int_{E} x^{2} ~d m=0.
  \] Show that \(m(E) = 0\).
\end{enumerate}

\hypertarget{section-2}{%
\subsection{3}\label{section-2}}

Let \[
f_{n}(x)=a e^{-n a x}-b e^{-n b x} \quad \text{ where } 0 < a < b.
\]

Show that

\begin{enumerate}
\def\labelenumi{\alph{enumi}.}
\tightlist
\item
  \(\sum_{n=1}^{\infty}\left|f_{n}\right| \text { is not in } L^{1}([0, \infty), m)\)
\end{enumerate}

\begin{quote}
Hint: \(f_n(x)\) has a root \(x_n\).
\end{quote}

\begin{enumerate}
\def\labelenumi{\alph{enumi}.}
\setcounter{enumi}{1}
\tightlist
\item
  \[
  \sum_{n=1}^{\infty} f_{n} \text { is in } L^{1}([0, \infty), m) 
  \quad \text { and } \quad 
  \int_{0}^{\infty} \sum_{n=1}^{\infty} f_{n}(x) ~d m=\ln \frac{b}{a}
  \]
\end{enumerate}

\hypertarget{section-3}{%
\subsection{4}\label{section-3}}

Let \(f(x, y)\) on \([-1, 1]^2\) be defined by \[
f(x, y) = \begin{cases}
\frac{x y}{\left(x^{2}+y^{2}\right)^{2}} & (x, y) \neq (0, 0) \\
0 & (x, y) = (0, 0)
\end{cases}
\] Determine if \(f\) is integrable.

\hypertarget{section-4}{%
\subsection{5}\label{section-4}}

Let \(f, g \in L^2(\RR)\). Prove that the formula \[
h(x):=\int_{-\infty}^{\infty} f(t) g(x-t) d t
\] defines a uniformly continuous function \(h\) on \(\RR\).

\hypertarget{section-5}{%
\subsection{5}\label{section-5}}

Show that the space \(C^1([a, b])\) is a Banach space when equipped with
the norm \[
\|f\|:=\sup _{x \in[a, b]}|f(x)|+\sup _{x \in[a, b]}\left|f^{\prime}(x)\right|.
\]

\hypertarget{fall-2016-neil-ish}{%
\section{Fall 2016 (Neil-ish)}\label{fall-2016-neil-ish}}

\hypertarget{section}{%
\subsection{1}\label{section}}

Define \[
f(x) = \sum_{n=1}^{\infty} \frac{1}{n^{x}}.
\]

Show that \(f\) converges to a differentiable function on
\((1, \infty)\) and that \[
f'(x)  =\sum_{n=1}^{\infty}\left(\frac{1}{n^{x}}\right)^{\prime}.
\]

\begin{quote}
Hint: \[
\left(\frac{1}{n^{x}}\right)^{\prime}=-\frac{1}{n^{x}} \ln n
\]
\end{quote}

\hypertarget{section-1}{%
\subsection{2}\label{section-1}}

Let \(f, g: [a, b] \to \RR\) be measurable with \[
\int_{a}^{b} f(x) ~d x=\int_{a}^{b} g(x) ~d x.
\]

Show that either

\begin{enumerate}
\def\labelenumi{\arabic{enumi}.}
\tightlist
\item
  \(f(x) = g(x)\) almost everywhere, or
\item
  There exists a measurable set \(E \subset [a, b]\) such that \[
  \int_{E} f(x) ~d x>\int_{E} g(x) ~d x
  \]
\end{enumerate}

\hypertarget{section-2}{%
\subsection{3}\label{section-2}}

Let \(f\in L^1(\RR)\). Show that \[
\lim _{x \rightarrow 0} \int_{\mathbb{R}}|f(y-x)-f(y)| d y=0
\]

\hypertarget{section-3}{%
\subsection{4}\label{section-3}}

Let \(( X, \mathcal{M}, \mu )\) be a measure space and suppose
\(\theset{E_n} \subset \mathcal M\) satisfies \[
\lim _{n \rightarrow \infty} \mu\left(X \backslash E_{n}\right)=0.
\]

Define \[
G \coloneqq \theset{x\in X : x \in E_n \text{ for only finitely many  } n}.
\]

Show that \(G \in \mathcal M\) and \(\mu(G) = 0\).

\hypertarget{section-4}{%
\subsection{5}\label{section-4}}

Let \(\phi\in L^\infty(\RR)\). Show that the following limit exists and
satisfies the equality \[
\lim _{n \rightarrow \infty}\left(\int_{\mathbb{R}} \frac{|\phi(x)|^{n}}{1+x^{2}} d x\right)^{\frac{1}{n}} = \norm{\phi}_\infty.
\]

\hypertarget{section-5}{%
\subsection{6}\label{section-5}}

Let \(f, g \in L^2(\RR)\). Show that \[
\lim _{n \rightarrow \infty} \int_{\mathbb{R}} f(x) g(x+n) d x=0
\]

\hypertarget{spring-2016-neil-ish}{%
\section{Spring 2016 (Neil-ish)}\label{spring-2016-neil-ish}}

\hypertarget{section}{%
\subsection{1}\label{section}}

For \(n\in \NN\), define \[
e_{n}=\left(1+\frac{1}{n}\right)^{n} 
\quad \text { and } \quad 
E_{n}=\left(1+\frac{1}{n}\right)^{n+1}
\]

Show that \(e_n < E_n\), and prove Bernoulli's inequality: \[
(1+x)^{n} \geq 1+n x \text { for }-1<x<\infty \text { and } n \in \mathbb{N}
\]

Use this to show the following:

\begin{enumerate}
\def\labelenumi{\arabic{enumi}.}
\tightlist
\item
  The sequence \(e_n\) is increasing.
\item
  The sequence \(E_n\) is decreasing.
\item
  \(2 < e_n < E_n < 4\).
\item
  \(\lim _{n \rightarrow \infty} e_{n}=\lim _{n \rightarrow \infty} E_{n}\).
\end{enumerate}

\hypertarget{section-1}{%
\subsection{2}\label{section-1}}

Let \(0 < \lambda < 1\) and construct a Cantor set \(C_\lambda\) by
successively removing middle intervals of length \(\lambda\).

Prove that \(m(C_\lambda) = 0\).

\hypertarget{section-2}{%
\subsection{3}\label{section-2}}

Let \(f\) be Lebesgue measurable on \(\RR\) and \(E \subset \RR\) be
measurable such that \[
0<A=\int_{E} f(x) d x<\infty.
\]

Show that for every \(0 < t < 1\), there exists a measurable set
\(E_t \subset E\) such that \[
\int_{E_{t}} f(x) d x=t A.
\]

\hypertarget{section-3}{%
\subsection{4}\label{section-3}}

Let \(E \subset \RR\) be measurable with \(m(E) < \infty\). Define \[
f(x)=m(E \cap(E+x)).
\]

Show that

\begin{enumerate}
\def\labelenumi{\arabic{enumi}.}
\tightlist
\item
  \(f\in L^1(\RR)\).
\item
  \(f\) is uniformly continuous.
\item
  \(\lim _{|x| \rightarrow \infty} f(x)=0\)
\end{enumerate}

\begin{quote}
Hint: \[
\chi_{E \cap(E+x)}(y)=\chi_{E}(y) \chi_{E}(y-x)
\]
\end{quote}

\hypertarget{section-4}{%
\subsection{5}\label{section-4}}

Let \((X, \mathcal M, \mu)\) be a measure space. For \(f\in L^1(\mu)\)
and \(\lambda > 0\), define \[
\phi(\lambda)=\mu(\{x \in X | f(x)>\lambda\}) 
\quad \text { and } \quad 
\psi(\lambda)=\mu(\{x \in X | f(x)<-\lambda\})
\]

Show that \(\phi, \psi\) are Borel measurable and \[
\int_{X}|f| ~d \mu=\int_{0}^{\infty}[\phi(\lambda)+\psi(\lambda)] ~d \lambda
\]

\hypertarget{section-5}{%
\subsection{6}\label{section-5}}

Without using the Riesz Representation Theorem, compute \[
\sup \left\{\left|\int_{0}^{1} f(x) e^{x} d x\right| \backepsilon f \in L^{2}([0,1], m) \text{ and } \|f\|_{2} \leq 1\right\}
\]

\hypertarget{fall-2015}{%
\section{Fall 2015}\label{fall-2015}}

\hypertarget{section}{%
\subsection{1}\label{section}}

Define \[
f(x)=c_{0}+c_{1} x^{1}+c_{2} x^{2}+\ldots+c_{n} x^{n} \text { with } n \text { even and } c_{n}>0.
\]

Show that there is a number \(x_m\) such that \(f(x_m) \leq f(x)\) for
all \(x\in \RR\).

\hypertarget{section-1}{%
\subsection{2}\label{section-1}}

Let \(f: \RR \to \RR\) be Lebesgue measurable.

\begin{enumerate}
\def\labelenumi{\arabic{enumi}.}
\tightlist
\item
  Show that there is a sequence of simple functions \(s_n(x)\) such that
  \(s_n(x) \to f(x)\) for all \(x\in \RR\).
\item
  Show that there is a Borel measurable function \(g\) such that
  \(g = f\) almost everywhere.
\end{enumerate}

\hypertarget{section-2}{%
\subsection{3}\label{section-2}}

Compute the following limit: \[
\lim _{n \rightarrow \infty} \int_{1}^{n} \frac{n e^{-x}}{1+n x^{2}} ~\sin \left(\frac x n\right) ~d x
\]

\hypertarget{section-3}{%
\subsection{4}\label{section-3}}

Let \(f: [1, \infty) \to \RR\) such that \(f(1) = 1\) and \[
f^{\prime}(x)= \frac{1} {x^{2}+f(x)^{2}}
\]

Show that the following limit exists and satisfies the equality \[
\lim _{x \rightarrow \infty} f(x) \leq 1 + \frac \pi 4
\]

\hypertarget{section-4}{%
\subsection{5}\label{section-4}}

Let \(f, g \in L^1(\RR)\) be Borel measurable.

\begin{enumerate}
\def\labelenumi{\arabic{enumi}.}
\tightlist
\item
  Show that
\end{enumerate}

\begin{itemize}
\tightlist
\item
  The function \[F(x, y) \coloneqq f(x-y) g(y)\] is Borel measurable on
  \(\RR^2\), and
\item
  For almost every \(y\in \RR\), \[F_y(x) \coloneqq f(x-y)g(y)\] is
  integrable with respect to \(y\).
\end{itemize}

\begin{enumerate}
\def\labelenumi{\arabic{enumi}.}
\setcounter{enumi}{1}
\tightlist
\item
  Show that \(f\ast g \in L^1(\RR)\) and \[
  \|f * g\|_{1} \leq\|f\|_{1}\|g\|_{1}
  \]
\end{enumerate}

\hypertarget{section-5}{%
\subsection{6}\label{section-5}}

Let \(f: [0, 1] \to \RR\) be continuous. Show that \[
\sup \left\{\|f g\|_{1} \suchthat g \in L^{1}[0,1],~~ \|g\|_{1} \leq 1\right\}=\|f\|_{\infty}
\]

\hypertarget{spring-2015}{%
\section{Spring 2015}\label{spring-2015}}

\hypertarget{section}{%
\subsection{1}\label{section}}

Let \((X, d)\) and \((Y, \rho)\) be metric spaces, \(f: X\to Y\), and
\(x_0 \in X\).

Prove that the following statements are equivalent:

\begin{enumerate}
\def\labelenumi{\arabic{enumi}.}
\tightlist
\item
  For every \(\varepsilon > 0 \quad \exists \delta > 0\) such that
  \(\rho( f(x), f(x_0) ) < \varepsilon\) whenever
  \(d(x, x_0) < \delta\).
\item
  The sequence \(\theset{f(x_n)}_{n=1}^\infty \to f(x_0)\) for every
  sequence \(\theset{x_n} \to x_0\) in \(X\).
\end{enumerate}

\hypertarget{section-1}{%
\subsection{2}\label{section-1}}

Let \(f: \RR \to \CC\) be continuous with period 1. Prove that \[
\lim _{N \rightarrow \infty} \frac{1}{N} \sum_{n=1}^{N} f(n \alpha)=\int_{0}^{1} f(t) d t \quad \forall \alpha \in \RR\setminus\QQ.
\]

\begin{quote}
Hint: show this first for the functions \(f(t) = e^{2\pi i k t}\) for
\(k\in \ZZ\).
\end{quote}

\hypertarget{section-2}{%
\subsection{3}\label{section-2}}

Let \(\mu\) be a finite Borel measure on \(\RR\) and \(E \subset \RR\)
Borel. Prove that the following statements are equivalent:

\begin{enumerate}
\def\labelenumi{\arabic{enumi}.}
\tightlist
\item
  \(\forall \varepsilon > 0\) there exists \(G\) open and \(F\) closed
  such that \[
  F \subseteq E \subseteq G \quad \text{and} \quad \mu(G\setminus F) < \varepsilon.
  \]
\item
  There exists a \(V \in G_\delta\) and \(H \in F_\sigma\) such that \[
  H \subseteq E \subseteq V \quad \text{and}\quad \mu(V\setminus H) = 0
  \]
\end{enumerate}

\hypertarget{section-3}{%
\subsection{4}\label{section-3}}

Define \[
f(x, y):=\left\{\begin{array}{ll}{\frac{x^{1 / 3}}{(1+x y)^{3 / 2}}} & {\text { if } 0 \leq x \leq y} \\ {0} & {\text { otherwise }}\end{array}\right.
\]

Carefully show that \(f \in L^1(\RR^2)\).

\hypertarget{section-4}{%
\subsection{5}\label{section-4}}

Let \(\mathcal H\) be a Hilbert space.

\begin{enumerate}
\def\labelenumi{\arabic{enumi}.}
\tightlist
\item
  Let \(x\in \mathcal H\) and \(\theset{u_n}_{n=1}^N\) be an orthonormal
  set. Prove that the best approximation to \(x\) in \(\mathcal H\) by
  an element in \(\mathrm{span}_\CC\theset{u_n}\) is given by \[
    \hat x \coloneqq \sum_{n=1}^N \inner{x}{u_n}u_n.
    \]
\item
  Conclude that finite dimensional subspaces of \(\mathcal H\) are
  always closed.
\end{enumerate}

\hypertarget{section-5}{%
\subsection{6}\label{section-5}}

Let \(f \in L^1(\RR)\) and \(g\) be a bounded measurable function on
\(\RR\).

\begin{enumerate}
\def\labelenumi{\arabic{enumi}.}
\tightlist
\item
  Show that the convolution \(f\ast g\) is well-defined, bounded, and
  uniformly continuous on \(\RR\).
\item
  Prove that one further assumes that \(g \in C^1(\RR)\) with bounded
  derivative, then \(f\ast g \in C^1(\RR)\) and \[
  \frac{d}{d x}(f * g)=f *\left(\frac{d}{d x} g\right)
  \]
\end{enumerate}

\hypertarget{fall-2014}{%
\section{Fall 2014}\label{fall-2014}}

\hypertarget{section}{%
\subsection{1}\label{section}}

Let \(\theset{f_n}\) be a sequence of continuous functions such that
\(\sum f_n\) converges uniformly.

Prove that \(\sum f_n\) is also continuous.

\hypertarget{section-1}{%
\subsection{2}\label{section-1}}

Let \(I\) be an index set and \(\alpha: I \to (0, \infty)\).

\begin{enumerate}
\def\labelenumi{\arabic{enumi}.}
\item
  Show that \[
  \sum_{i \in I} a(i):=\sup _{\substack{ J \subset I \\ J \text { finite }}} \sum_{i \in J} a(i)<\infty \implies I \text{ is countable.}
  \]
\item
  Suppose \(I = \QQ\) and \(\sum_{q \in \mathbb{Q}} a(q)<\infty\).
  Define \[
    f(x):=\sum_{\substack{q \in \mathbb{Q}\\ q \leq x}} a(q).
    \] Show that \(f\) is continuous at \(x \iff x\not\in \QQ\).
\end{enumerate}

\hypertarget{section-2}{%
\subsection{3}\label{section-2}}

Let \(f\in L^1(\RR)\). Show that \[
\forall\varepsilon > 0 ~~\exists \delta > 0 \text{ such that } m(E) < \delta \implies \int_{E}|f(x)| d x<\varepsilon
\]

\hypertarget{section-3}{%
\subsection{4}\label{section-3}}

Let \(g\in L^\infty([0, 1])\) Prove that \[
\int_{[0,1]} f(x) g(x) d x=0 \quad\text{for all continuous } f:[0, 1] \to \RR \implies g(x) = 0 \text{ almost everywhere. }
\]

\hypertarget{section-4}{%
\subsection{5}\label{section-4}}

\begin{enumerate}
\def\labelenumi{\arabic{enumi}.}
\item
  Let \(f \in C_c^0(\RR^n)\), and show \[
  \lim _{t \rightarrow 0} \int_{\mathbb{R}^{n}}|f(x+t)-f(x)| d x=0.
  \]
\item
  Extend the above result to \(f\in L^1(\RR^n)\) and show that \[
  f\in L^1(\RR^n),~ g\in L^\infty(\RR^n) \implies f \ast g \text{ is bounded and uniformly continuous. }
  \]
\end{enumerate}

\hypertarget{section-5}{%
\subsection{6}\label{section-5}}

Let \(1 \leq p,q \leq \infty\) be conjugate exponents, and show that \[
f \in L^p(\RR^n) \implies \|f\|_{p}=\sup _{\|g\|_{q}=1}\left|\int f(x) g(x) d x\right|
\]

\hypertarget{spring-2014}{%
\section{Spring 2014}\label{spring-2014}}

\hypertarget{section}{%
\subsection{1}\label{section}}

\begin{enumerate}
\def\labelenumi{\arabic{enumi}.}
\item
  Give an example of a continuous \(f\in L^1(\RR)\) such that
  \(f(x) \not\to 0\) as\(\abs x \to \infty\).
\item
  Show that if \(f\) is \emph{uniformly} continuous, then \[
  \lim_{\abs x \to \infty} f(x) = 0.
  \]
\end{enumerate}

\hypertarget{section-1}{%
\subsection{2}\label{section-1}}

Let \(\theset{a_n}\) be a sequence of real numbers such that \[
\theset{b_n} \in \ell^2(\NN) \implies \sum a_n b_n < \infty.
\] Show that \(\sum a_n^2 < \infty\).

\begin{quote}
Note: Assume \(a_n, b_n\) are all non-negative.
\end{quote}

\hypertarget{section-2}{%
\subsection{3}\label{section-2}}

Let \(f: \RR \to \RR\) and suppose \[
\forall x\in \RR,\quad f(x) \geq \limsup _{y \rightarrow x} f(y)
\] Prove that \(f\) is Borel measurable.

\hypertarget{section-3}{%
\subsection{4}\label{section-3}}

Let \((X, \mathcal M, \mu)\) be a measure space and suppose \(f\) is a
measurable function on \(X\). Show that \[
\lim _{n \rightarrow \infty} \int_{X} f^{n} ~d \mu =
\begin{cases}
\infty & \text{or} \\
\mu(f\inv(1)),
\end{cases}
\] and characterize the collection of functions of each type.

\hypertarget{section-4}{%
\subsection{5}\label{section-4}}

Let \(f, g \in L^1([0, 1])\) and for all \(x\in [0, 1]\) define \[
F(x):= \int_{0}^{x} f(y) d y \quad \text { and } \quad G(x):= \int_{0}^{x} g(y) d y.
\]

Prove that \[
\int_{0}^{1} F(x) g(x) d x=F(1) G(1)-\int_{0}^{1} f(x) G(x) d x
\]

\end{document}
