\documentclass[12pt]{article}
\usepackage[margin=1in]{geometry}

\usepackage{stackrel}
\usepackage{array}
\usepackage{mathtools}
\usepackage{amsmath, amsthm, amssymb, amsfonts, amsxtra, amscd, thmtools, xpatch}
\usepackage{calligra, mathrsfs}
\usepackage{colonequals}
\usepackage{graphicx}
\usepackage{enumitem}
\usepackage{mleftright}
\usepackage{stackengine}
\usepackage{glossaries}
\usepackage{xcolor}
\usepackage{tikz}
\usetikzlibrary{arrows,positioning, shapes.geometric, calc}
\usepackage{hyperref}

\setlength{\parindent}{0in}


\newcommand{\NN}[0]{{\mathbb{N}}}
\newcommand{\HH}[0]{{\mathbb{H}}}
\newcommand{\RR}[0]{{\mathbb{R}}}
\newcommand{\ZZ}[0]{{\mathbb{Z}}}
\newcommand{\CC}[0]{{\mathbb{C}}}
\newcommand{\QQ}[0]{{\mathbb{Q}}}
\newcommand{\DD}[0]{{\mathbb{D}}}
\newcommand{\RP}[0]{{\mathbb{RP}}}
\newcommand{\CP}[0]{{\mathbb{CP}}}
\newcommand{\HP}[0]{{\mathbb{HP}}}
\newcommand{\OP}[0]{{\mathbb{OP}}}
\newcommand{\FF}[0]{{\mathbb{F}}}
\newcommand{\GF}[0]{{\mathbb{GF}}}
\newcommand{\PP}[0]{{\mathbb{P}}}
\newcommand{\Af}[0]{{\mathbb{A}}}
\renewcommand{\AA}[0]{{\mathbb{A}}}
\newcommand{\MM}[0]{{\mathcal{M}}}
\newcommand{\mca}[0]{{\mathcal{A}}}
\newcommand{\mcb}[0]{{\mathcal{B}}}
\newcommand{\mcc}[0]{{\mathcal{C}}}
\newcommand{\mcd}[0]{{\mathcal{D}}}
\newcommand{\mce}[0]{{\mathcal{E}}}
\newcommand{\mcf}[0]{{\mathcal{F}}}
\newcommand{\mcg}[0]{{\mathcal{G}}}
\newcommand{\mch}[0]{{\mathcal{H}}}
\newcommand{\mci}[0]{{\mathcal{I}}}
\newcommand{\mcj}[0]{{\mathcal{J}}}
\newcommand{\mcl}[0]{{\mathcal{L}}}
\newcommand{\mcp}[0]{{\mathcal{P}}}
\newcommand{\mcs}[0]{{\mathcal{S}}}
\newcommand{\mcv}[0]{{\mathcal{V}}}
\newcommand{\mcx}[0]{{\mathcal{X}}}
\newcommand{\mcz}[0]{{\mathcal{Z}}}
\newcommand{\TT}[0]{{\mathbb{T}}}
\newcommand{\Sp}[0]{{\mathbb{S}}}
\newcommand{\KK}[0]{{\mathbb{K}}}
\newcommand{\mfa}[0]{{\mathfrak{a}}}
\newcommand{\mfb}[0]{{\mathfrak{b}}}
\newcommand{\mfc}[0]{{\mathfrak{c}}}
\newcommand{\mfm}[0]{{\mathfrak{m}}}
\newcommand{\mm}[0]{{\mathfrak{m}}}
\newcommand{\pr}[0]{{\mathfrak{p}}}
\newcommand{\mfp}[0]{{\mathfrak{p}}}
\newcommand{\mfr}[0]{{\mathfrak{r}}}
\newcommand{\liegl}[0]{{\mathfrak{gl}}}
\newcommand{\lieg}[0]{{\mathfrak{g}}}
\newcommand{\lien}[0]{{\mathfrak{n}}}
\newcommand{\lieh}[0]{{\mathfrak{h}}}
\newcommand{\lieb}[0]{{\mathfrak{b}}}
\newcommand{\lieu}[0]{{\mathfrak{u}}}
\newcommand{\SL}[0]{{\text{SL}}}
\newcommand{\tors}[0]{{\text{tors}}}
\newcommand{\const}[0]{{\text{const.}}}
\newcommand{\st}[0]{~{\text{s.t.}}~}
\newcommand{\liesl}[0]{{\mathfrak{sl}}}
\newcommand{\SP}[0]{{\text{SP}}}
\newcommand{\liesp}[0]{{\mathfrak{sp}}}
\newcommand{\SO}[0]{{\text{SO}}}
\newcommand{\nil}[0]{{\mathrm{nil}}}
\newcommand{\rad}[0]{{\mathrm{rad}}}
\newcommand{\rk}[0]{{\mathrm{rank}}}
\newcommand{\minpoly}[0]{{\mathrm{minpoly}}}
\newcommand{\stab}[0]{{\mathrm{Stab}}}
\newcommand{\supp}[0]{{\mathrm{supp}}}
\newcommand{\wt}[0]{{\mathrm{wt}}}
\newcommand{\ord}[0]{{\mathrm{Ord}}}
\newcommand{\lieso}[0]{{\mathfrak{so}}}
\newcommand{\OO}[0]{{\mathcal{O}}}
\newcommand{\Gr}[0]{{\text{Gr}}}
\newcommand{\GL}[0]{{\text{GL}}}
\newcommand{\gl}[0]{{\text{gl}}}
\newcommand{\Sch}[0]{{\text{Sch}}}
\newcommand{\Sets}[0]{{\text{Sets}}}
\newcommand{\Set}[0]{{\text{Sets}}}
\newcommand{\set}[0]{{\text{Sets}}}
\newcommand{\mltext}[1]{\left\{\begin{array}{c}#1\end{array}\right\}}
\newcommand{\dual}[0]{^\vee}
\newcommand{\Tr}[0]{\mathrm{Tr}}
\newcommand{\tr}[0]{\mathrm{Tr}}
\def\endo{\operatorname{End}}
\def\Endo{\operatorname{End}}
\def\Ind{\operatorname{Ind}}
\def\res{\operatorname{Res}}
\def\Res{\operatorname{Res}}
\def\ind{\operatorname{Ind}}
\def\res{\operatorname{Res}}
\def\Res{\operatorname{Res}}
\newcommand{\divides}[0]{{~\Bigm|~}}
\newcommand{\notdivides}[0]{{~\not\Bigm|~}}
\newcommand{\sym}[0]{\mathrm{Sym}}
\newcommand{\aut}[0]{\mathrm{Aut}}
\newcommand{\grad}[0]{\mathrm{grad}}
\newcommand{\adj}[0]{\mathrm{adj}}
\newcommand{\rmod}[0]{\mathrm{R-mod}}
\newcommand{\ralg}[0]{\mathrm{R-alg}}
\newcommand{\sign}[0]{\mathrm{sign}}
\newcommand{\spec}[0]{{\mathrm{Spec}}}
\newcommand{\maxspec}[0]{{\mathrm{maxSpec}~}}
\newcommand{\pic}[0]{{\mathrm{Pic}~}}
\newcommand{\Pic}[0]{{\mathrm{Pic}~}}
\newcommand{\spanof}[0]{{\mathrm{span}}}
%\newcommand{\suchthat}[0]{{~\backepsilon ~}}
\newcommand{\suchthat}[0]{{~\mathrel{\Big|}~}}
\newcommand{\uniformlyconverges}[0]{\rightrightarrows}
\newcommand{\mapsvia}[1]{\xrightarrow{#1}}
\newcommand{\converges}[1]{\overset{#1}}
\newcommand{\generators}[1]{\left\langle{#1}\right\rangle}
\newcommand{\theset}[1]{\left\{{#1}\right\}}
\newcommand{\too}[1]{{\xrightarrow{#1}}}
\newcommand{\correspond}[1]{\theset{\substack{#1}}}
\newcommand{\restrictionof}[2]{{\left.{#1}\right|_{#2}}}
\newcommand{\inner}[2]{{\left\langle {#1},~{#2} \right\rangle}}
\newcommand{\indicator}[1]{{\unicode{x1D7D9}\left[#1\right]}}
\newcommand{\equalsbecause}[1]{{\stackrel{\mbox{$\tiny{\text{ #1 }}$}}{=}}}
\newcommand{\conjugate}[1]{{\overline{{#1}}}}
\newcommand{\strike}[1]{{\enclose{horizontalstrike}{#1}}}
\newcommand{\realpart}[1]{{\mathcal{Re}({#1})}}
\newcommand{\imaginarypart}[1]{{\mathcal{Im}({#1})}}
\newcommand{\dd}[2]{{\frac{\partial #1}{\partial #2}}}
\newcommand{\rotate}[2]{{\style{display: inline-block; transform: rotate(#1deg)}{#2}}}
\newcommand{\stirling}[2]{\genfrac\{\}{0pt}{}{#1}{#2}}
\newcommand{\thevector}[1]{{\left[ {#1} \right]}}
%\newcommand{\abs}[2]{{\left\lvert {#2} \right\rvert_{\text{#1}}}}
\newcommand{\norm}[1]{{\left\lVert {#1} \right\rVert}}
\newcommand{\abs}[1]{{\left\lvert {#1} \right\rvert}}
\newcommand{\qty}[1]{{\left(  {#1} \right)}}
\renewcommand{\hat}[1]{\widehat{#1}}
\newcommand{\intersect}[0]{\bigcap}
\newcommand{\union}[0]{\bigcup}
\newcommand{\cok}[0]{\operatorname{coker}}
\newcommand{\coker}[0]{\operatorname{coker}}
\newcommand{\rank}[0]{\operatorname{rank}}
\newcommand{\tensor}[0]{\otimes}
\newcommand{\semidirect}[0]{\rtimes}
\newcommand{\pt}[0]{\{\text{pt}\}}
\newcommand{\bd}[0]{{\del}}
\newcommand{\wait}[0]{{\,\cdot\,}}
\newcommand{\selfmap}[0]{{\circlearrowleft}}
\newcommand{\tor}[0]{\text{Tor}}
\newcommand{\Tor}[0]{\text{Tor}}
\newcommand{\ext}[0]{\text{Ext}}
\newcommand{\actson}[0]{\curvearrowright}
\newcommand{\actsonl}[0]{\curvearrowleft}
\newcommand{\disjoint}[0]{{\coprod}}
\newcommand{\dash}[0]{{\hbox{-}}}
\newcommand{\bigast}[0]{{\mathop{\Large \ast}}}
\newcommand{\from}[0]{\leftarrow}
\newcommand{\covers}[0]{\twoheadrightarrow}
\newcommand{\Zp}[0]{\mathbb{Z}_{(p)}}
\newcommand{\Qp}[0]{\mathbb{Q}_{(p)}}
\newcommand{\ZpZ}[0]{\mathbb{Z}/p\mathbb{Z}}
\newcommand{\ZnZ}[0]{\mathbb{Z}/n\mathbb{Z}}
\newcommand{\Sm}[0]{{\text{Sm}_k}}
\newcommand{\GG}[0]{{\mathbb{G}}}
\newcommand{\bung}[0]{\text{Bun}_G}
\newcommand{\Aut}[0]{{\text{Aut}}}
\newcommand{\maps}[0]{{\text{Maps}}}
\newcommand{\diam}[0]{{\mathrm{diam}}}
\newcommand{\Arg}[0]{\mathrm{Arg}}
\newcommand{\reg}[0]{\mathrm{Reg}}
\newcommand{\transverse}[0]{\pitchfork}
\newcommand{\del}[0]{{\partial}}
\newcommand{\im}[0]{{\text{im}~}}
\newcommand{\homotopic}[0]{\simeq}
\newcommand{\into}[0]{\to}
\newcommand{\cross}[0]{\times}
\newcommand{\definedas}[0]{\coloneqq}
\newcommand{\surjects}[0]{\twoheadrightarrow}
\newcommand{\onto}[0]{\twoheadrightarrow}
\newcommand{\injects}[0]{\hookrightarrow}
\newcommand{\id}[0]{\text{id}}
\newcommand{\inv}[0]{^{-1}}
\newcommand{\normal}[0]{{~\trianglelefteq~}}
\newcommand{\trianglerightneq}{\mathrel{\ooalign{\raisebox{-0.5ex}{\reflectbox{\rotatebox{90}{$\nshortmid$}}}\cr$\triangleright$\cr}\mkern-3mu}}
\newcommand{\normalneq}{\mathrel{\reflectbox{$\trianglerightneq$}}}
\newcommand{\units}[0]{^{\times}}
\newcommand{\sep}[0]{^\text{sep}}
\newcommand{\annd}[0]{{\text{ and }}}
\newcommand{\orr}[0]{{\text{ or }}}
\newcommand{\arcsec}[0]{\mathrm{arcsec}}
\newcommand{\sgn}[0]{\mathrm{sgn}}
\newcommand{\crit}[0]{\mathrm{crit}}
\newcommand{\Gal}[0]{\mathrm{Gal}}
\newcommand{\gal}[0]{\mathrm{Gal}}
\newcommand{\ann}[0]{\mathrm{Ann}}
\newcommand{\Ann}[0]{\mathrm{Ann}}
\newcommand{\lcm}[0]{\mathrm{lcm}}
\newcommand{\eps}[0]{\varepsilon}
\newcommand{\multinomial}[1]{\left(\!\!{#1}\!\!\right)}
\newcommand{\stirlingfirst}[2]{\genfrac{[}{]}{0pt}{}{#1}{#2}}
\newcommand{\floor}[1]{{\left\lfloor #1 \right\rfloor}}
\newcommand{\ad}[0]{\mathrm{ad}~}
\newcommand{\ch}[0]{\mathrm{char}~}
\renewcommand{\mid}[0]{\mathrel{\Big|}}
%\newcommand{\vector}[1]{{\mathbf{ {#1} }}}
%\newcommand{\hom}[0]{\text{Hom}}
\renewcommand{\to}[0]{\longrightarrow}
\renewcommand{\bar}[1]{\overline{#1}}
\renewcommand{\div}[0]{\mathrm{Div}}
%\newcommand{\char}[0]{\text{char}}
\renewcommand{\vector}[1]{\mathbf{#1}}
\newcommand{\cl}[0]{\operatorname{cl}}
\newcommand{\Cl}[0]{\operatorname{Cl}}

\newcommand{\qtext}[1]{{\quad \operatorname{#1} \quad}}
\newcommand{\mct}[0]{{\mathcal{T}}}


%\newtheorem{theorem}{Theorem}[section]
\newcommand{\qty}[1]{\left( {#1} \right)}
 
% Pandoc-specific fixes
\providecommand{\tightlist}{%
  \setlength{\itemsep}{0pt}\setlength{\parskip}{0pt}}

\begin{document}
%%%%%%%%%%%%%%%%%%%%%%%%%%%%%%%%%%%%%%%%%%%%%%%%%%%%%%%%%%%%%%%%%%%%%%%%%%%
\begin{titlepage}
	\begin{center}
        \textbf{\LARGE{MAKEMEAQUAL UNIVERSITY}} \\
                \vspace{2mm} %5mm vertical space
            \textbf{\Large{Department of Mathematics}}\\
                \vspace{15mm} %5mm vertical space
            \LARGE\textsc{PhD Qualifying Examination} \\
                \vspace{2mm} %5mm vertical space
            \Large \textit{in} \\
            \Large\textsc {Mathematics}\\
                \vspace{30mm} %5mm vertical space
            \textbf{\LARGE{\today}}
    \end{center}
\end{titlepage}

\hypertarget{algebra-140-questions}{%
\section{Algebra (140 Questions)}\label{algebra-140-questions}}

\hypertarget{question-1}{%
\subsection{Question 1}\label{question-1}}

Let \(G\) be a finite group with \(n\) distinct conjugacy classes. Let
\(g_1 \cdots g_n\) be representatives of the conjugacy classes of \(G\).

Prove that if \(g_i g_j = g_j g_i\) for all \(i, j\) then \(G\) is
abelian.

\hypertarget{question-2}{%
\subsection{Question 2}\label{question-2}}

Let \(G\) be a group of order 105 and let \(P, Q, R\) be Sylow 3, 5, 7
subgroups respectively.

\begin{enumerate}
\def\labelenumi{(\alph{enumi})}
\item
  Prove that at least one of \(Q\) and \(R\) is normal in \(G\).
\item
  Prove that \(G\) has a cyclic subgroup of order 35.
\item
  Prove that both \(Q\) and \(R\) are normal in \(G\).
\item
  Prove that if \(P\) is normal in \(G\) then \(G\) is cyclic.
\end{enumerate}

\hypertarget{question-3}{%
\subsection{Question 3}\label{question-3}}

Let \(R\) be a ring with the property that for every
\(a \in R, a^2 = a\).

\begin{enumerate}
\def\labelenumi{(\alph{enumi})}
\item
  Prove that \(R\) has characteristic 2.
\item
  Prove that \(R\) is commutative.
\end{enumerate}

\hypertarget{question-4}{%
\subsection{Question 4}\label{question-4}}

Let \(F\) be a finite field with \(q\) elements.

Let \(n\) be a positive integer relatively prime to \(q\) and let
\(\omega\) be a primitive \(n\)th root of unity in an extension field of
\(F\).

Let \(E = F [\omega]\) and let \(k = [E : F]\).

\begin{enumerate}
\def\labelenumi{(\alph{enumi})}
\item
  Prove that \(n\) divides \(q^{k}-1\).
\item
  Let \(m\) be the order of \(q\) in \(\ZZ/n\ZZ\). Prove that \(m\)
  divides \(k\).
\item
  Prove that \(m = k\).
\end{enumerate}

\hypertarget{question-5}{%
\subsection{Question 5}\label{question-5}}

Let \(R\) be a ring and \(M\) an \(R\dash\)module.

\begin{quote}
Recall that the set of torsion elements in M is defined by \[
\Tor(m) = \{m \in M \suchthat \exists r \in R, ~r \neq 0, ~rm = 0\}
.\]
\end{quote}

\begin{enumerate}
\def\labelenumi{(\alph{enumi})}
\item
  Prove that if \(R\) is an integral domain, then \(\Tor(M )\) is a
  submodule of \(M\) .
\item
  Give an example where \(\Tor(M )\) is not a submodule of \(M\).
\item
  If \(R\) has zero-divisors, prove that every non-zero \(R\dash\)module
  has non-zero torsion elements.
\end{enumerate}

\hypertarget{question-6}{%
\subsection{Question 6}\label{question-6}}

Let \(R\) be a commutative ring with multiplicative identity. Assume
Zorn's Lemma.

\begin{enumerate}
\def\labelenumi{(\alph{enumi})}
\item
  Show that \[
  N = \{r \in R \mid r^n = 0 \text{ for some } n > 0\}
  \] is an ideal which is contained in any prime ideal.
\item
  Let \(r\) be an element of \(R\) not in \(N\). Let \(S\) be the
  collection of all proper ideals of \(R\) not containing any positive
  power of \(r\). Use Zorn's Lemma to prove that there is a prime ideal
  in \(S\).
\item
  Suppose that \(R\) has exactly one prime ideal \(P\) . Prove that
  every element \(r\) of \(R\) is either nilpotent or a unit.
\end{enumerate}

\hypertarget{question-7}{%
\subsection{Question 7}\label{question-7}}

Let \(\zeta_n\) denote a primitive \(n\)th root of 1 \(\in \QQ\). You
may assume the roots of the minimal polynomial \(p_n(x)\) of \(\zeta_n\)
are exactly the primitive \(n\)th roots of 1.

Show that the field extension \(\QQ(\zeta_n )\) over \(\QQ\) is Galois
and prove its Galois group is \((\ZZ/n\ZZ)\units\).

How many subfields are there of \(\QQ(\zeta_{20} )\)?

\hypertarget{question-8}{%
\subsection{Question 8}\label{question-8}}

Let \(\{e_1, \cdots, e_n \}\) be a basis of a real vector space \(V\)
and let \[
\Lambda \definedas \theset{ \sum r_i e_i \mid ri \in \ZZ}
\]

Let \(\cdot\) be a non-degenerate (\(v \cdot w = 0\) for all
\(w \in V \iff v = 0\)) symmetric bilinear form on V such that the Gram
matrix \(M = (e_i \cdot e_j )\) has integer entries.

Define the dual of \(\Lambda\) to be

\[
\Lambda \dual \definedas \{v \in V \suchthat v \cdot x \in \ZZ \text{ for all } x \in \Lambda
\}
.\]

\begin{enumerate}
\def\labelenumi{(\alph{enumi})}
\item
  Show that \(\Lambda \subset \Lambda \dual\).
\item
  Prove that \(\det M \neq 0\) and that the rows of \(M\inv\) span
  \(\Lambda\dual\).
\item
  Prove that \(\det M = |\Lambda\dual /\Lambda|\).
\end{enumerate}

\hypertarget{question-9}{%
\subsection{Question 9}\label{question-9}}

Let \(A\) be a square matrix over the complex numbers. Suppose that
\(A\) is nonsingular and that \(A^{2019}\) is diagonalizable over
\(\CC\).

Show that \(A\) is also diagonalizable over \(\CC\).

\hypertarget{question-10}{%
\subsection{Question 10}\label{question-10}}

Let \(F = \FF_p\) , where \(p\) is a prime number.

\begin{enumerate}
\def\labelenumi{(\alph{enumi})}
\item
  Show that if \(\pi(x) \in F[x]\) is irreducible of degree \(d\), then
  \(\pi(x)\) divides \(x^{p^d} - x\).
\item
  Show that if \(\pi(x) \in F[x]\) is an irreducible polynomial that
  divides \(x^{p^n} - x\), then \(\deg \pi(x)\) divides \(n\).
\end{enumerate}

\hypertarget{question-11}{%
\subsection{Question 11}\label{question-11}}

How many isomorphism classes are there of groups of order 45?

Describe a representative from each class.

\hypertarget{question-12}{%
\subsection{Question 12}\label{question-12}}

For a finite group \(G\), let \(c(G)\) denote the number of conjugacy
classes of \(G\).

\begin{enumerate}
\def\labelenumi{(\alph{enumi})}
\item
  Prove that if two elements of \(G\) are chosen uniformly at
  random,then the probability they commute is precisely \[
  \frac{c(G)}{\abs G}
  .\]
\item
  State the class equation for a finite group.
\item
  Using the class equation (or otherwise) show that the probability in
  part (a) is at most \[
  \frac 1 2 + \frac 1 {2[G : Z(G)]}
  .\]
\end{enumerate}

\begin{quote}
Here, as usual, \(Z(G)\) denotes the center of \(G\).
\end{quote}

\hypertarget{question-13}{%
\subsection{Question 13}\label{question-13}}

Let \(R\) be an integral domain. Recall that if \(M\) is an
\(R\dash\)module, the \emph{rank} of \(M\) is defined to be the maximum
number of \(R\dash\)linearly independent elements of \(M\) .

\begin{enumerate}
\def\labelenumi{(\alph{enumi})}
\item
  Prove that for any \(R\dash\)module \(M\), the rank of \(\tor(M )\) is
  0.
\item
  Prove that the rank of \(M\) is equal to the rank of of
  \(M/\tor(M )\).
\item
  Suppose that M is a non-principal ideal of \(R\).
\item
  Prove that \(M\) is torsion-free of rank 1 but not free.
\end{enumerate}

\hypertarget{question-14}{%
\subsection{Question 14}\label{question-14}}

Let \(R\) be a commutative ring with 1.

\begin{quote}
Recall that \(x \in R\) is nilpotent iff \(xn = 0\) for some positive
integer \(n\).
\end{quote}

\begin{enumerate}
\def\labelenumi{(\alph{enumi})}
\item
  Show that every proper ideal of \(R\) is contained within a maximal
  ideal.
\item
  Let \(J(R)\) denote the intersection of all maximal ideals of \(R\).

  Show that \(x \in J(R) \iff 1 + rx\) is a unit for all \(r \in R\).
\item
  Suppose now that \(R\) is finite. Show that in this case \(J(R)\)
  consists precisely of the nilpotent elements in R.
\end{enumerate}

\hypertarget{question-15}{%
\subsection{Question 15}\label{question-15}}

Let \(p\) be a prime number. Let \(A\) be a \(p \times p\) matrix over a
field \(F\) with 1 in all entries except 0 on the main diagonal.

Determine the Jordan canonical form (JCF) of \(A\)

\begin{enumerate}
\def\labelenumi{(\alph{enumi})}
\item
  When \(F = \QQ\),
\item
  When \(F = \FF_p\).
\end{enumerate}

\begin{quote}
Hint: In both cases, all eigenvalues lie in the ground field. In each
case find a matrix \(P\) such that \(P\inv AP\) is in JCF.
\end{quote}

\hypertarget{question-16}{%
\subsection{Question 16}\label{question-16}}

Let \(\zeta = e^{2\pi i/8}\).

\begin{enumerate}
\def\labelenumi{(\alph{enumi})}
\item
  What is the degree of \(\QQ(\zeta)/\QQ\)?
\item
  How many quadratic subfields of \(\QQ(\zeta)\) are there?
\item
  What is the degree of \(\QQ(\zeta, \sqrt[4] 2)\) over \(\QQ\)?
\end{enumerate}

\hypertarget{question-17}{%
\subsection{Question 17}\label{question-17}}

Let \(G\) be a finite group whose order is divisible by a prime number
\(p\). Let \(P\) be a normal \(p\dash\)subgroup of \(G\) (so
\(\abs P = p^c\) for some \(c\)).

\begin{enumerate}
\def\labelenumi{(\alph{enumi})}
\item
  Show that \(P\) is contained in every Sylow \(p\dash\)subgroup of
  \(G\).
\item
  Let \(M\) be a maximal proper subgroup of \(G\). Show that either
  \(P \subseteq M\) or \(|G/M | = p^b\) for some \(b \leq c\).
\end{enumerate}

\hypertarget{question-18}{%
\subsection{Question 18}\label{question-18}}

\begin{enumerate}
\def\labelenumi{(\alph{enumi})}
\item
  Suppose the group \(G\) acts on the set \(X\) . Show that the
  stabilizers of elements in the same orbit are conjugate.
\item
  Let \(G\) be a finite group and let \(H\) be a proper subgroup. Show
  that the union of the conjugates of \(H\) is strictly smaller than
  \(G\), i.e. \[
  \union_{g\in G} gHg\inv \subsetneq G
  \]
\item
  Suppose \(G\) is a finite group acting transitively on a set \(S\)
  with at least 2 elements. Show that there is an element of \(G\) with
  no fixed points in \(S\).
\end{enumerate}

\hypertarget{question-19}{%
\subsection{Question 19}\label{question-19}}

Let \(F \subset K \subset L\) be finite degree field extensions. For
each of the following assertions, give a proof or a counterexample.

\begin{enumerate}
\def\labelenumi{(\alph{enumi})}
\item
  If \(L/F\) is Galois, then so is \(K/F\).
\item
  If \(L/F\) is Galois, then so is \(L/K\).
\item
  If \(K/F\) and \(L/K\) are both Galois, then so is \(L/F\).
\end{enumerate}

\hypertarget{question-20}{%
\subsection{Question 20}\label{question-20}}

Let \(V\) be a finite dimensional vector space over a field (the field
is not necessarily algebraically closed).

Let \(\phi : V \to V\) be a linear transformation. Prove that there
exists a decomposition of \(V\) as \(V = U \oplus W\) , where \(U\) and
\(W\) are \(\phi\dash\)invariant subspaces of \(V\) ,
\(\restrictionof{\phi}{U}\) is nilpotent, and
\(\restrictionof{\phi}{W}\) is nonsingular.

\hypertarget{question-21}{%
\subsection{Question 21}\label{question-21}}

Let \(A\) be an \(n \times n\) matrix.

\begin{enumerate}
\def\labelenumi{(\alph{enumi})}
\item
  Suppose that \(v\) is a column vector such that the set
  \(\{v, Av, . . . , A^{n-1} v\}\) is linearly independent. Show that
  any matrix \(B\) that commutes with \(A\) is a polynomial in \(A\).
\item
  Show that there exists a column vector \(v\) such that the set
  \(\{v, Av, . . . , A^{n-1} v\}\) is linearly independent \(\iff\) the
  characteristic polynomial of A equals the minimal polynomial of A.
\end{enumerate}

\hypertarget{question-22}{%
\subsection{Question 22}\label{question-22}}

Let \(R\) be a commutative ring, and let \(M\) be an \(R\dash\)module.
An \(R\dash\)submodule \(N\) of \(M\) is maximal if there is no
\(R\dash\)module \(P\) with \(N \subsetneq P \subsetneq M\).

\begin{enumerate}
\def\labelenumi{(\alph{enumi})}
\item
  Show that an \(R\dash\)submodule \(N\) of \(M\) is maximal
  \(\iff M /N\) is a simple \(R\dash\)module: i.e., \(M /N\) is nonzero
  and has no proper, nonzero \(R\dash\)submodules.
\item
  Let \(M\) be a \(\ZZ\dash\)module. Show that a \(\ZZ\dash\)submodule
  \(N\) of \(M\) is maximal \(\iff \#M /N\) is a prime number.
\item
  Let \(M\) be the \(\ZZ\dash\)module of all roots of unity in \(\CC\)
  under multiplication. Show that there is no maximal
  \(\ZZ\dash\)submodule of \(M\).
\end{enumerate}

\hypertarget{question-23}{%
\subsection{Question 23}\label{question-23}}

Let \(R\) be a commutative ring.

\begin{enumerate}
\def\labelenumi{(\alph{enumi})}
\item
  Let \(r \in R\). Show that the map \begin{align*}
  r\bullet : R &\to R \\
  x &\mapsto r x
  .\end{align*} is an \(R\dash\)module endomorphism of \(R\).
\item
  We say that \(r\) is a \textbf{zero-divisor} if r\(\bullet\) is not
  injective. Show that if \(r\) is a zero-divisor and \(r \neq 0\), then
  the kernel and image of \(R\) each consist of zero-divisors.
\item
  Let \(n \geq 2\) be an integer. Show: if \(R\) has exactly \(n\)
  zero-divisors, then \(\#R \leq n^2\) .
\item
  Show that up to isomorphism there are exactly two commutative rings
  \(R\) with precisely 2 zero-divisors.
\end{enumerate}

\begin{quote}
You may use without proof the following fact: every ring of order 4 is
isomorphic to exactly one of the following: \[
\frac{ \ZZ }{ 4\ZZ}, \quad
\frac{ \frac{  \ZZ }{ 2\ZZ} [t]}{(t^2 + t + 1)}, \quad
\frac{ \frac{ \ZZ }{ 2\ZZ} [t]}{ (t^2 - t)}, \quad
\frac{ \frac{ \ZZ}{2\ZZ}[t]}{(t^2 )}
.\]
\end{quote}

\hypertarget{question-24}{%
\subsection{Question 24}\label{question-24}}

\begin{enumerate}
\def\labelenumi{(\alph{enumi})}
\item
  Use the Class Equation (equivalently, the conjugation action of a
  group on itself) to prove that any \(p\dash\)group (a group whose
  order is a positive power of a prime integer \(p\)) has a nontrivial
  center.
\item
  Prove that any group of order \(p^2\) (where \(p\) is prime) is
  abelian.
\item
  Prove that any group of order \(5^2 \cdot 7^2\) is abelian.
\item
  Write down exactly one representative in each isomorphism class of
  groups of order \(5^2 \cdot 7^2\).
\end{enumerate}

\hypertarget{question-25}{%
\subsection{Question 25}\label{question-25}}

Let \(f(x) = x^4 - 4x^2 + 2 \in \QQ[x]\).

\begin{enumerate}
\def\labelenumi{(\alph{enumi})}
\item
  Find the splitting field \(K\) of \(f\), and compute \([K: \QQ]\).
\item
  Find the Galois group \(G\) of \(f\), both as an explicit group of
  automorphisms, and as a familiar abstract group to which it is
  isomorphic.
\item
  Exhibit explicitly the correspondence between subgroups of \(G\) and
  intermediate fields between \(\QQ\) and \(k\).
\end{enumerate}

\hypertarget{question-26}{%
\subsection{Question 26}\label{question-26}}

Let \(K\) be a Galois extension of \(\QQ\) with Galois group \(G\), and
let \(E_1 , E_2\) be intermediate fields of \(K\) which are the
splitting fields of irreducible \(f_i (x) \in \QQ[x]\).

Let \(E = E_1 E_2 \subset K\).

Let \(H_i = \Gal(K/E_i)\) and \(H = \Gal(K/E)\).

\begin{enumerate}
\def\labelenumi{(\alph{enumi})}
\item
  Show that \(H = H_1 \cap H_2\).
\item
  Show that \(H_1 H_2\) is a subgroup of \(G\).
\item
  Show that \[
  \Gal(K/(E_1 \cap E_2 )) = H_1 H_2
  .\]
\end{enumerate}

\hypertarget{question-27}{%
\subsection{Question 27}\label{question-27}}

Let

\[
A=\left[\begin{array}{lll}{0} & {1} & {-2} \\ {1} & {1} & {-3} \\ {1} & {2} & {-4}\end{array}\right] \in M_{3}(\mathbb{C})
\]

\begin{enumerate}
\def\labelenumi{(\alph{enumi})}
\item
  Find the Jordan canonical form J of A.
\item
  Find an invertible matrix \(P\) such that \(P\inv AP = J\).
\end{enumerate}

\begin{quote}
You should not need to compute \(P\inv\).
\end{quote}

\hypertarget{question-28}{%
\subsection{Question 28}\label{question-28}}

Let \[
M=\left(\begin{array}{ll}{a} & {b} \\ {c} & {d}\end{array}\right)
\quad \text{and} \quad 
N=\left(\begin{array}{cc}{x} & {u} \\ {-y} & {-v}\end{array}\right)
\]

over a commutative ring \(R\), where \(b\) and \(x\) are units of \(R\).
Prove that \[
M N=\left(\begin{array}{ll}{0} & {0} \\ {0} & {*}\end{array}\right)
\implies MN = 0
.\]

\hypertarget{question-29}{%
\subsection{Question 29}\label{question-29}}

Let \[
M = \{(w, x, y, z) \in \ZZ^4 \suchthat w + x + y + z \in 2\ZZ\}
,\]

and

\[
N = \{(w, x, y, z) \in \ZZ^4 \suchthat 4\divides (w - x),~ 4\divides (x - y),~ 4\divides ( y - z)\}
.\]

\begin{enumerate}
\def\labelenumi{(\alph{enumi})}
\item
  Show that \(N\) is a \(\ZZ\dash\)submodule of \(M\) .
\item
  Find vectors \(u_1 , u_2 , u_3 , u_4 \in \ZZ^4\) and integers
  \(d_1 , d_2 , d_3 , d_4\) such that \[
  \{u_1 , u_2 , u_3 , u_4 \}
  \] is a free basis for \(M\), and \[
  \{d_1 u_1,~ d_2 u_2,~ d_3 u_3,~ d_4 u_4 \}
  \] is a free basis for \(N\) .
\item
  Use the previous part to describe \(M/N\) as a direct sum of cyclic
  \(\ZZ\dash\)modules.
\end{enumerate}

\hypertarget{question-30}{%
\subsection{Question 30}\label{question-30}}

Let \(R\) be a PID and \(M\) be an \(R\dash\)module. Let \(p\) be a
prime element of \(R\). The module \(M\) is called
\emph{\(\generators{p}\dash\)primary} if for every \(m \in M\) there
exists \(k > 0\) such that \(p^k m = 0\).

\begin{enumerate}
\def\labelenumi{(\alph{enumi})}
\item
  Suppose M is \(\generators{p}\dash\)primary. Show that if \(m \in M\)
  and \(t \in R, ~t \not\in \generators{p}\), then there exists
  \(a \in R\) such that \(atm = m\).
\item
  A submodule \(S\) of \(M\) is said to be \emph{pure} if
  \(S \cap r M = rS\) for all \(r \in R\). Show that if \(M\) is
  \(\generators{p}\dash\)primary, then \(S\) is pure if and only if
  \(S \cap p^k M = p^k S\) for all \(k \geq 0\).
\end{enumerate}

\hypertarget{question-31}{%
\subsection{Question 31}\label{question-31}}

Let \(R = C[0, 1]\) be the ring of continuous real-valued functions on
the interval \([0, 1]\). Let I be an ideal of \(R\).

\begin{enumerate}
\def\labelenumi{(\alph{enumi})}
\item
  Show that if \(f \in I, a \in [0, 1]\) are such that \(f (a) \neq 0\),
  then there exists \(g \in I\) such that \(g(x) \geq 0\) for all
  \(x \in [0, 1]\), and \(g(x) > 0\) for all \(x\) in some open
  neighborhood of \(a\).
\item
  If \(I \neq R\), show that the set
  \(Z(I) = \{x \in [0, 1] \suchthat f(x) = 0 \text{ for all } f \in I\}\)
  is nonempty.
\item
  Show that if \(I\) is maximal, then there exists \(x_0 \in [0, 1]\)
  such that \(I = \{ f \in R \suchthat f (x_0 ) = 0\}\).
\end{enumerate}

\hypertarget{question-32}{%
\subsection{Question 32}\label{question-32}}

Suppose the group \(G\) acts on the set \(A\). Assume this action is
faithful (recall that this means that the kernel of the homomorphism
from \(G\) to \(\sym(A)\) which gives the action is trivial) and
transitive (for all \(a, b\) in \(A\), there exists \(g\) in \(G\) such
that \(g \cdot a = b\).)

\begin{enumerate}
\def\labelenumi{(\alph{enumi})}
\item
  For \(a \in A\), let \(G_a\) denote the stabilizer of \(a\) in \(G\).
  Prove that for any \(a \in A\), \[
  \intersect_{\sigma\in G} \sigma G_a \sigma\inv = \theset{1}
  .\]
\item
  Suppose that \(G\) is abelian. Prove that \(|G| = |A|\). Deduce that
  every abelian transitive subgroup of \(S_n\) has order \(n\).
\end{enumerate}

\hypertarget{question-33}{%
\subsection{Question 33}\label{question-33}}

\begin{enumerate}
\def\labelenumi{(\alph{enumi})}
\tightlist
\item
  Classify the abelian groups of order 36.
\end{enumerate}

For the rest of the problem, assume that \(G\) is a non-abelian group of
order 36.

\begin{quote}
You may assume that the only subgroup of order 12 in \(S_4\) is \(A_4\)
and that \(A_4\) has no subgroup of order 6.
\end{quote}

\begin{enumerate}
\def\labelenumi{(\alph{enumi})}
\setcounter{enumi}{1}
\item
  Prove that if the 2-Sylow subgroup of \(G\) is normal, \(G\) has a
  normal subgroup \(N\) such that \(G/N\) is isomorphic to \(A_4\).
\item
  Show that if \(G\) has a normal subgroup \(N\) such that \(G/N\) is
  isomorphic to \(A_4\) and a subgroup \(H\) isomorphic to \(A_4\) it
  must be the direct product of \(N\) and \(H\).
\item
  Show that the dihedral group of order 36 is a non-abelian group of
  order 36 whose Sylow-2 subgroup is not normal.
\end{enumerate}

\hypertarget{question-34}{%
\subsection{Question 34}\label{question-34}}

Let \(F\) be a field. Let \(f(x)\) be an irreducible polynomial in
\(F[x]\) of degree \(n\) and let \(g(x)\) be any polynomial in \(F[x]\).
Let \(p(x)\) be an irreducible factor (of degree \(m\)) of the
polynomial \(f(g(x))\).

Prove that \(n\) divides \(m\). Use this to prove that if \(r\) is an
integer which is not a perfect square, and \(n\) is a positive integer
then every irreducible factor of \(x^{2n} - r\) over \(\QQ[x]\) has even
degree.

\hypertarget{question-35}{%
\subsection{Question 35}\label{question-35}}

\begin{enumerate}
\def\labelenumi{(\alph{enumi})}
\item
  Let \(f (x)\) be an irreducible polynomial of degree 4 in \(\QQ[x]\)
  whose splitting field \(K\) over \(\QQ\) has Galois group \(G = S_4\).

  Let \(\theta\) be a root of \(f(x)\). Prove that \(\QQ[\theta]\) is an
  extension of \(\QQ\) of degree 4 and that there are no intermediate
  fields between \(\QQ\) and \(\QQ[\theta]\).
\item
  Prove that if \(K\) is a Galois extension of \(\QQ\) of degree 4, then
  there is an intermediate subfield between \(K\) and \(\QQ\).
\end{enumerate}

\hypertarget{question-36}{%
\subsection{Question 36}\label{question-36}}

A ring R is called \emph{simple} if its only two-sided ideals are \(0\)
and \(R\).

\begin{enumerate}
\def\labelenumi{(\alph{enumi})}
\item
  Suppose \(R\) is a commutative ring with 1. Prove \(R\) is simple if
  and only if \(R\) is a field.
\item
  Let \(k\) be a field. Show the ring \(M_n (k)\), \(n \times n\)
  matrices with entries in \(k\), is a simple ring.
\end{enumerate}

\hypertarget{question-37}{%
\subsection{Question 37}\label{question-37}}

For a ring \(R\), let \(U(R)\) denote the multiplicative group of units
in \(R\). Recall that in an integral domain \(R\), \(r \in R\) is called
\emph{irreducible} if \(r\) is not a unit in R, and the only divisors of
\(r\) have the form \(ru\) with \(u\) a unit in \(R\).

We call a non-zero, non-unit \(r \in R\) \emph{prime} in \(R\) if
\(r \divides ab \implies r \divides a\) or \(r \divides b\). Consider
the ring \(R = \{a + b \sqrt{-5}\suchthat a, b \in Z\}\).

\begin{enumerate}
\def\labelenumi{(\alph{enumi})}
\item
  Prove \(R\) is an integral domain.
\item
  Show \(U(R) = \{\pm1\}\).
\item
  Show \(3, 2 + \sqrt{-5}\), and \(2 - \sqrt{-5}\) are irreducible in
  \(R\).
\item
  Show 3 is not prime in \(R\).
\item
  Conclude \(R\) is not a PID.
\end{enumerate}

\hypertarget{question-38}{%
\subsection{Question 38}\label{question-38}}

Let \(F\) be a field and let \(V\) and \(W\) be vector spaces over \(F\)
.

Make \(V\) and \(W\) into \(F[x]\dash\)modules via linear operators
\(T\) on \(V\) and \(S\) on \(W\) by defining \(X \cdot v = T (v)\) for
all \(v \in V\) and \(X \cdot w = S(w)\) for all w \(\in\) W .

Denote the resulting \(F[x]\dash\)modules by \(V_T\) and \(W_S\)
respectively.

\begin{enumerate}
\def\labelenumi{(\alph{enumi})}
\item
  Show that an \(F[x]\dash\)module homomorphism from \(V_T\) to \(W_S\)
  consists of an \(F\dash\)linear transformation \(R : V \to W\) such
  that \(RT = SR\).
\item
  Show that \(VT \cong WS\) as \(F[x]\dash\)modules \(\iff\) there is an
  \(F\dash\)linear isomorphism \(P : V \to W\) such that
  \(T = P\inv SP\).
\item
  Recall that a module \(M\) is \emph{simple} if \(M \neq 0\) and any
  proper submodule of \(M\) must be zero. Suppose that \(V\) has
  dimension 2. Give an example of \(F\), \(T\) with \(V_T\) simple.
\item
  Assume \(F\) is algebraically closed. Prove that if \(V\) has
  dimension 2, then any \(V_T\) is not simple.
\end{enumerate}

\hypertarget{question-39}{%
\subsection{Question 39}\label{question-39}}

Classify the groups of order \(182 = 2 \cdot 7 \cdot 13\).

\hypertarget{question-40}{%
\subsection{Question 40}\label{question-40}}

Let \(G\) be a finite group of order \(p^nm\) where \(p\) is a prime and
\(m\) is not divisible by \(p\). Prove that if \(H\) is a subgroup of
\(G\) of order \(p^k\) for some \(k<n\), then the normalizer of \(H\) in
\(G\) properly contains \(H\).

\hypertarget{question-41}{%
\subsection{Question 41}\label{question-41}}

Let \(H\) be a subgroup of \(S_n\) of index \(n\). Prove:

\begin{enumerate}
\def\labelenumi{\arabic{enumi}.}
\item
  There is an isomorphism \(f: S_n \to S_n\) such that \(f(H)\) is the
  subgroup of \(S_n\) stabilizing \(n\). In particular, \(H\) is
  isomorphic to \(S_{n-1}\).
\item
  The only subgroups of \(S_n\) containing \(H\) are \(S_n\) and \(H\).
\end{enumerate}

\hypertarget{question-42}{%
\subsection{Question 42}\label{question-42}}

\begin{itemize}
\item
  Prove that a group of order \(351=3^3\cdot 13\) cannot be simple.
\item
  Prove that a group of order \(33\) must be cyclic.
\end{itemize}

\hypertarget{question-43}{%
\subsection{Question 43}\label{question-43}}

\begin{enumerate}
\def\labelenumi{\arabic{enumi}.}
\item
  Let \(G\) be a group, and \(Z(G)\) the center of \(G\). Prove that if
  \(G/Z(G)\) is cyclic, then \(G\) is abelian.
\item
  Prove that a group of order \(p^n\), where \(p\) is a prime and
  \(n \geq 1\), has non-trivial center.
\item
  Prove that a group of order \(p^2\) must be abelian.
\end{enumerate}

\hypertarget{question-44}{%
\subsection{Question 44}\label{question-44}}

Let \(G\) be a finite group.

\begin{enumerate}
\def\labelenumi{\arabic{enumi}.}
\item
  Prove that if \(H < G\) is a proper subgroup, then \(G\) is not the
  union of conjugates of \(H\).
\item
  Suppose that \(G\) acts transitively on a set \(X\) with \(|X| > 1\).
  Prove that there exists an element of \(G\) with no fixed points in
  \(X\).
\end{enumerate}

\hypertarget{question-45}{%
\subsection{Question 45}\label{question-45}}

Classify all groups of order \(15\) and of order \(30\).

\hypertarget{question-46}{%
\subsection{Question 46}\label{question-46}}

Count the number of \(p\)-Sylow subgroups of \(S_p\).

\hypertarget{question-47}{%
\subsection{Question 47}\label{question-47}}

\begin{enumerate}
\def\labelenumi{\arabic{enumi}.}
\item
  Let \(G\) be a group of order \(n\). Suppose that for every divisor
  \(d\) of \(n\), \(G\) contains at most one subgroup of order \(d\).
  Show that \(G\) is clyclic.
\item
  Let \(F\) be a field. Show that every finite subgroup of the group of
  units \(F^\times\) is cyclic.
\end{enumerate}

\hypertarget{question-48}{%
\subsection{Question 48}\label{question-48}}

Let \(K\) and \(L\) be finite fields. Show that \(K\) is contained in
\(L\) if and only if \(\# K = p^r\) and \(\# L = p^s\) for the same
prime \(p\), and \(r \leq s\).

\hypertarget{question-49}{%
\subsection{Question 49}\label{question-49}}

Let \(K\) and \(L\) be finite fields with \(K \subseteq L\). Prove that
\(L\) is Galois over \(K\) and that \(\mathrm{Gal}(L/K)\) is cyclic.

\hypertarget{question-50}{%
\subsection{Question 50}\label{question-50}}

Fix a field \(F\), a separable polynomial \(f\in F[x]\) of degree
\(n \geq 3\), and a splitting field \(L\) for \(f\). Prove that if
\([L:F] = n!\) then:

\begin{enumerate}
\def\labelenumi{\arabic{enumi}.}
\item
  \(f\) is irreducible.
\item
  For each root \(r\) of \(f\), \(r\) is the unique root of \(f\) in
  \(F(r)\).
\item
  For every root \(r\) of \(f\), there are no proper intermediate fields
  \(F \subset L \subset F(r)\).
\end{enumerate}

\hypertarget{question-51}{%
\subsection{Question 51}\label{question-51}}

\begin{enumerate}
\def\labelenumi{\arabic{enumi}.}
\item
  Show that \(\sqrt{2+\sqrt{2}}\) is a root of
  \(p(x) = x^2 - 4x^2 + 2 \in \mathbb{Q}[x]\).
\item
  Prove that \(\mathbb{Q}(\sqrt{2 + \sqrt{2}})\) is a Galois extension
  of \(\mathbb{Q}\) and find its Galois group. (Hint: note that
  \(\sqrt{2 - \sqrt{2}}\) is another root of \(p(x)\)).
\item
  Let \(f(x) = x^3 - 5\). Determine the splitting field \(K\) of
  \(f(x)\) over \(\mathbb{Q}\) and the Galois group of \(f(x)\). Give an
  example of a proper sub-extension \(\mathbb{Q} \subset L \subset K\),
  such that \(L/\mathbb{Q}\) is Galois.
\end{enumerate}

\hypertarget{question-52}{%
\subsection{Question 52}\label{question-52}}

An integral domain \(R\) is said to be an \emph{Euclidean domain} if
there is a function \(N: R \to \{n\in\mathbb{Z} \mid n\geq 0\}\) such
that \(N(0)=0\) and for each \(a,b\in R\) with \(b\neq 0\), there exist
elements \(q,r\in R\) with \begin{align*}
  a = qb + r, \quad \text{and} \quad r = 0 \, \text{ or } \, N(r) < N(b).
\end{align*}

Prove:

\begin{enumerate}
\def\labelenumi{\arabic{enumi}.}
\item
  The ring \(F[[x]]\) of power series over a field \(F\) is an Euclidean
  domain.
\item
  Every Euclidean domain is a PID.
\end{enumerate}

\hypertarget{question-53}{%
\subsection{Question 53}\label{question-53}}

Let \(F\) be a field, and let \(R\) be the subring of \(F[X]\) of
polynomials with \(X\) coefficient equal to \(0\). Prove that \(R\) is
not a UFD.

\hypertarget{question-54}{%
\subsection{Question 54}\label{question-54}}

\(R\) is a commutative ring with 1. Prove that if \(I\) is a maximal
ideal in \(R\), then \(R/I\) is a field. Prove that if \(R\) is a PID,
then every nonzero prime ideal in \(R\) is maximal. Conclude that if
\(R\) is a PID and \(p\in R\) is prime, then \(R/(p)\) is a field.

\hypertarget{question-55}{%
\subsection{Question 55}\label{question-55}}

Prove that any square matrix is conjugate to its transpose matrix. (You
may prove it over \(\mathbb{C}\)).

\hypertarget{question-56}{%
\subsection{Question 56}\label{question-56}}

Determine the number of conjugacy classes of \(16 \times 16\) matrices
with entries in \(\mathbb{Q}\) and minimal polynomial
\((x^2+1)^2(x^3+2)^2\).

\hypertarget{question-57}{%
\subsection{Question 57}\label{question-57}}

Let \(V\) be a vector space over a field \(F\). The evaluation map
\(e\colon V \to (V^\vee)^\vee\) is defined by
\(e(v)(f) \colonequals f(v)\) for \(v\in V\) and \(f\in V^\vee\).

\begin{enumerate}
\def\labelenumi{\arabic{enumi}.}
\item
  Prove that \(e\) is an injection.
\item
  Prove that \(e\) is an isomorphism if and only if \(V\) is finite
  dimensional.
\end{enumerate}

\hypertarget{question-58}{%
\subsection{Question 58}\label{question-58}}

Let \(R\) be a principal ideal domain that is not a field, and write
\(F\) for its field of fractions. Prove that \(F\) is not a finitely
generated \(R\)-module.

\hypertarget{question-59}{%
\subsection{Question 59}\label{question-59}}

Carefully state Zorn's lemma and use it to prove that every vector space
has a basis.

\hypertarget{question-60}{%
\subsection{Question 60}\label{question-60}}

Show that no finite group is the union of conjugates of a proper
subgroup.

\hypertarget{question-61}{%
\subsection{Question 61}\label{question-61}}

Classify all groups of order 18 up to isomorphism.

\hypertarget{question-62}{%
\subsection{Question 62}\label{question-62}}

Let \(\alpha,\beta\) denote the unique positive real \(5^{\text{th}}\)
root of 7 and \(4^{\text{th}}\) root of 5, respectively. Determine the
degree of \(\mathbb Q(\alpha,\beta)\) over \(\mathbb Q\).

\hypertarget{question-63}{%
\subsection{Question 63}\label{question-63}}

Show that the field extension
\(\mathbb Q\subseteq\mathbb Q\left( \sqrt{2+\sqrt2}\right)\) is Galois
and determine its Galois group.

\hypertarget{question-64}{%
\subsection{Question 64}\label{question-64}}

Let \(M\) be a square matrix over a field \(K\). Use a suitable
canonical form to show that \(M\) is similar to its transpose \(M^T\).

\hypertarget{question-65}{%
\subsection{Question 65}\label{question-65}}

Let \(G\) be a finite group and \(\pi_0\), \(\pi_1\) be two irreducible
representations of \(G\). Prove or disprove the following assertion:
\(\pi_0\) and \(\pi_1\) are equivalent if and only if
\(\det\pi_0(g)=\det\pi_1(g)\) for all \(g\in G\).

\hypertarget{question-66}{%
\subsection{Question 66}\label{question-66}}

Let \(R\) be a Noetherian ring. Prove that \(R[x]\) and \(R[[x]]\) are
both Noetherian. (The first part of the question is asking you to prove
the Hilbert Basis Theorem, not to use it!)

\hypertarget{question-67}{%
\subsection{Question 67}\label{question-67}}

Classify (with proof) all fields with finitely many elements.

\hypertarget{question-68}{%
\subsection{Question 68}\label{question-68}}

Suppose \(A\) is a commutative ring and \(M\) is a finitely presented
module. Given any surjection \(\phi:A^n\rightarrow M\) from a finite
free \(A\)-module, show that \(\ker\phi\) is finitely generated.

\hypertarget{question-69}{%
\subsection{Question 69}\label{question-69}}

Classify all groups of order 57.

\hypertarget{question-70}{%
\subsection{Question 70}\label{question-70}}

Show that a finite simple group cannot have a 2-dimensional irreducible
representation over \(\mathbb C\).

\begin{quote}
Hint: the determinant might prove useful.
\end{quote}

\hypertarget{question-71}{%
\subsection{Question 71}\label{question-71}}

Let \(G\) be a finite simple group. Assume that every proper subgroup of
\(G\) is abelian. Prove that then \(G\) is cyclic of prime order.

\hypertarget{question-72}{%
\subsection{Question 72}\label{question-72}}

Let \(a\in\mathbb N\), \(a>0\). Compute the Galois group of the
splitting field of the polynomial \(x^5-5a^4x+a\) over \(\mathbb Q\).

\hypertarget{question-73}{%
\subsection{Question 73}\label{question-73}}

Recall that an inner automorphism of a group is an automorphism given by
conjugation by an element of the group. An outer automorphism is an
automorphism that is not inner.

\begin{itemize}
\tightlist
\item
  Prove that \(S_5\) has a subgroup of order 20.
\item
  Use the subgroup from (a) to construct a degree 6 permutation
  representation of \(S_5\) (i.e., an embedding
  \(S_5 \hookrightarrow S_6\) as a transitive permutation group on 6
  letters).
\item
  Conclude that \(S_6\) has an outer automorphism.
\end{itemize}

\hypertarget{question-74}{%
\subsection{Question 74}\label{question-74}}

Let \(A\) be a commutative ring and \(M\) a finitely generated
\(A\)-module. Define \begin{align*}
  \Ann(M) = \{a \in A: am = 0 \text{ for all } m \in M\}
.\end{align*} Show that for a prime ideal \(\mathfrak p \subset A\), the
following are equivalent:

\begin{itemize}
\item
  \(\Ann(M) \not\subset \mathfrak p\)
\item
  The localization of \(M\) at the prime ideal \(\mathfrak p\) is \(0\).
\item
  \(M \otimes_A k(\mathfrak p) = 0\), where
  \(k(\mathfrak p) = A_{\mathfrak p}/\mathfrak p A_{\mathfrak p}\) is
  the residue field of \(A\) at \(\mathfrak p\).
\end{itemize}

\hypertarget{question-75}{%
\subsection{Question 75}\label{question-75}}

Let \(A = \CC[x,y]/(y^2-(x-1)^3 - (x-1)^2)\).

\begin{itemize}
\tightlist
\item
  Show that \(A\) is an integral domain and sketch the \(\RR\)-points of
  \(\text{Spec} A\).
\item
  Find the integral closure of \(A\). Recall that for an integral domain
  \(A\) with fraction field \(K\), the integral closure of \(A\) in
  \(K\) is the set of all elements of \(K\) integral over \(A\).
\end{itemize}

\hypertarget{question-76}{%
\subsection{Question 76}\label{question-76}}

Let \(R = k[x,y]\) where \(k\) is a field, and let \(I=(x,y)R\).

\begin{itemize}
\item
  Show that \begin{align*}
  0 \to R \mapsvia{\phi} R \oplus R \mapsvia{\psi} R \to k \to 0
  \end{align*} where \(\phi(a) = (-ya,xa)\), \(\psi((a,b)) = xa+yb\) for
  \(a,b \in R\), is a projective resolution of the \(R\)-module
  \(k \simeq R/I\).
\item
  Show that \(I\) is not a flat \(R\)-module by computing
  \(\Tor_i^R(I,k)\)
\end{itemize}

\hypertarget{question-77}{%
\subsection{Question 77}\label{question-77}}

\begin{itemize}
\tightlist
\item
  Find an irreducible polynomial of degree 5 over the field
  \(\mathbb Z/2\) of two elements and use it to construct a field of
  order 32 as a quotient of the polynomial ring \(\mathbb Z/2[x]\).
\item
  Using the polynomial found in part (a), find a \(5\times5\) matrix
  \(M\) over \(\mathbb Z/2\) of order 31, so that \(M^{31}=I\) but
  \(M\neq I\).
\end{itemize}

\hypertarget{question-78}{%
\subsection{Question 78}\label{question-78}}

Find the minimal polynomial of \(\sqrt2+\sqrt3\) over \(\mathbb Q\).
Justify your answer.

\hypertarget{question-79}{%
\subsection{Question 79}\label{question-79}}

\hfill

\begin{itemize}
\tightlist
\item
  Let \(R\) be a commutative ring with no nonzero nilpotent elements.
  Show that the only units in the polynomial ring \(R[x]\) are the units
  of \(R\), regarded as constant polynomials.
\item
  Find all units in the polynomial ring \(\mathbb Z_4[x]\).
\end{itemize}

\hypertarget{question-80}{%
\subsection{Question 80}\label{question-80}}

Let \(p\), \(q\) be two distinct primes. Prove that there is at most one
non-abelian group of order \(pq\) and describe the pairs \((p,q)\) such
that there is no non-abelian group of order \(pq\).

\hypertarget{question-81}{%
\subsection{Question 81}\label{question-81}}

\begin{itemize}
\tightlist
\item
  Let \(L\) be a Galois extension of a field \(K\) of degree 4. What is
  the minimum number of subfields there could be strictly between \(K\)
  and \(L\)? What is the maximum number of such subfields? Give examples
  where these bounds are attained.
\item
  How do these numbers change if we assume only that \(L\) is separable
  (but not necessarily Galois) over \(K\)?
\end{itemize}

\hypertarget{question-82}{%
\subsection{Question 82}\label{question-82}}

Let \(R\) be a commutative algebra over \(\mathbb C\). A derivation of
\(R\) is a \(\mathbb C\)-linear map \(D:R\rightarrow R\) such that (i)
\(D(1)=0\) and (ii) \(D(ab)=D(a)b+aD(b)\) for all \(a,b\in R\).

\begin{itemize}
\item
  Describe all derivations of the polynomial ring \(\mathbb C[x]\).
\item
  Let \(A\) be the subring (or \(\mathbb C\)-subalgebra) of
  \(\mathrm{End}_{\mathbb C}(\mathbb C[x])\) generated by all
  derivations of \(\mathbb C[x]\) and the left multiplications by \(x\).
  Prove that \(\mathbb C[x]\) is a simple left \(A\)-module.
  \textgreater{} Note that the inclusion
  \(A\rightarrow\mathrm{End}_{\mathbb C}(\mathbb C[x])\) defines a
  natural left \(A\)-module structure on \(\mathbb C[x]\).
\end{itemize}

\hypertarget{question-83}{%
\subsection{Question 83}\label{question-83}}

Let \(G\) be a non-abelian group of order \(p^3\) with \(p\) a prime.

\begin{itemize}
\item
  Determine the order of the center \(Z\) of \(G\).
\item
  Determine the number of inequivalent complex 1-dimensional
  representations of \(G\).
\item
  Compute the dimensions of all the inequivalent irreducible
  representations of \(G\) and verify that the number of such
  representations equals the number of conjugacy classes of \(G\).
\end{itemize}

\hypertarget{question-84}{%
\subsection{Question 84}\label{question-84}}

\begin{itemize}
\item
  Let \(G\) be a group (not necessarily finite) that contains a subgroup
  of index \(n\). Show that \(G\) contains a \textit{normal} subgroup
  \(N\) such that \(n\leq[G:N]\leq n!\)
\item
  Use part (a) to show that there is no simple group of order 36.
\end{itemize}

\hypertarget{question-85}{%
\subsection{Question 85}\label{question-85}}

Let \(p\) be a prime, let \(\mathbb F_p\) be the \(p\)-element field,
and let \(K=\mathbb F_p(t)\) be the field of rational functions in \(t\)
with coefficients in \(\mathbb F_p\). Consider the polynomial
\(f(x)= x^p-t\in K[x]\).

\begin{itemize}
\item
  Show that \(f\) does not have a root in \(K\).
\item
  Let \(E\) be the splitting field of \(f\) over \(K\). Find the
  factorization of \(f\) over \(E\).
\item
  Conclude that \(f\) is irreducible over \(K\).
\end{itemize}

\hypertarget{question-86}{%
\subsection{Question 86}\label{question-86}}

Recall that a ring \(A\) is called \textit{graded} if it admits a direct
sum decomposition \(A=\oplus_{n=0}^{\infty}A_n\) as abelian groups, with
the property that \(A_iA_j\subseteq A_{i+j}\) for all \(i,j\geq 0\).
Prove that a graded commutative ring \(A=\oplus_{n=0}^{\infty} A_n\) is
Noetherian if and only if \(A_0\) is Noetherian and \(A\) is finitely
generated as an algebra over \(A_0\).

\hypertarget{question-87}{%
\subsection{Question 87}\label{question-87}}

Let \(R\) be a ring with the property that \(a^2=a\) for all \(a\in R\).

\begin{itemize}
\item
  Compute the Jacobson radical of \(R\).
\item
  What is the characteristic of \(R\)?
\item
  Prove that \(R\) is commutative.
\item
  Prove that if \(R\) is finite, then \(R\) is isomorphic (as a ring) to
  \((\mathbb Z/2\mathbb Z)^d\) for some \(d\).
\end{itemize}

\hypertarget{question-88}{%
\subsection{Question 88}\label{question-88}}

Let \(\overline{\mathbb F_p}\) denote the algebraic closure of
\(\mathbb F_p\). Show that the Galois group
\(\Gal(\overline{\mathbb F_p}/\mathbb F_p)\) has no non-trivial finite
subgroups.

\hypertarget{question-89}{%
\subsection{Question 89}\label{question-89}}

Let \(C_p\) denote the cyclic group of order \(p\).

\begin{itemize}
\item
  Show that \(C_p\) has two irreducible representations over
  \(\mathbb Q\) (up to isomorphism), one of dimension 1 and one of
  dimension \(p-1\).
\item
  Let \(G\) be a finite group, and let
  \(\rho:G\rightarrow \GL_n(\mathbb Q)\) be a representation of \(G\)
  over \(\mathbb Q\). Let
  \(\rho_{\mathbb C}:G\rightarrow\GL_n(\mathbb C)\) denote \(\rho\)
  followed by the inclusion
  \(\GL_n(\mathbb Q)\rightarrow \GL_n(\mathbb C)\). Thus
  \(\rho_{\mathbb C}\) is a representation of \(G\) over \(\mathbb C\),
  called the \textit{complexification} of \(\rho\). We say that an
  irreducible representation \(\rho\) of \(G\) is
  \textit{absolutely irreducible} if its complexification remains
  irreducible over \(\mathbb C\).\textbackslash{} Now suppose \(G\) is
  abelian and that every representation of \(G\) over \(\mathbb Q\) is
  absolutely irreducible. Show that \(G\cong(C_2)^k\) for some \(k\)
  (i.e., is a product of cyclic groups of order 2).
\end{itemize}

\hypertarget{question-90}{%
\subsection{Question 90}\label{question-90}}

Let \(G\) be a finite group and \(\mathbb Z[G]\) the internal group
algebra. Let \(\mathcal Z\) be the center of \(\mathbb Z[G]\). For each
conjugacy class \(C\subseteq G\), let \(P_C=\sum_{g\in C}g\).

\begin{itemize}
\item
  Show that the elements \(P_C\) form a \(\mathbb Z\)-basis for
  \(\mathcal Z\). Hence \(\mathcal Z\cong\mathbb Z^d\) as an abelian
  group, where \(d\) is the number of conjugacy classes in \(G\).
\item
  Show that if a ring \(R\) is isomorphic to \(\mathbb Z^d\) as an
  abelian group, then every element in \(R\) satisfies a monic integral
  polynomial.

  \begin{quote}
  \textbf{Hint:} Let \(\{v_1,\dots,v_d\}\) be a basis of \(R\) and for a
  fixed non-zero \(r\in R\), write \(rv_i=\sum_j a_{ij}v_j\). Use the
  Hamilton-Cayley theorem.
  \end{quote}
\item
  Let \(\pi:G\rightarrow\GL(V)\) be an irreducible representation of
  \(G\) (over \(\mathbb C\)). Show that \(\pi(P_C)\) acts on \(V\) as
  multiplication by the scalar

  \begin{align*}
  \frac{|C|\chi_{\pi}(C)}{\dim V},
  \end{align*}

  where \(\chi_{\pi}(C)\) is the value of the character \(\chi_{\pi}\)
  on any element of \(C\).
\item
  Conclude that \(|C|\chi_{\pi}(C)/\dim V\) is an algebraic integer.
\end{itemize}

\hypertarget{question-91}{%
\subsection{Question 91}\label{question-91}}

\begin{itemize}
\item
  Suppose that \(G\) is a finitely generated group. Let \(n\) be a
  positive integer. Prove that \(G\) has only finitely many subgroups of
  index \(n\)
\item
  Let \(p\) be a prime number. If \(G\) is any finitely-generated
  abelian group, let \(t_p(G)\) denote the number of subgroups of \(G\)
  of index \(p\). Determine the possible values of \(t_p(G)\) as \(G\)
  varies over all finitely-generated abelian groups.
\end{itemize}

\hypertarget{question-92}{%
\subsection{Question 92}\label{question-92}}

Suppose that \(G\) is a finite group of order 2013. Prove that \(G\) has
a normal subgroup \(N\) of index 3 and that \(N\) is a cyclic group.
Furthermore, prove that the center of \(G\) has order divisible by 11.
(You will need the factorization \(2013=3\cdot11\cdot61\).)

\hypertarget{question-93}{%
\subsection{Question 93}\label{question-93}}

This question concerns an extension \(K\) of \(\mathbb Q\) such that
\([K:\mathbb Q]=8\). Assume that \(K/\mathbb Q\) is Galois and let
\(G=\Gal(K/\mathbb Q)\). Furthermore, assume that \(G\) is non-abelian.

\begin{itemize}
\item
  Prove that \(K\) has a unique subfield \(F\) such that \(F/\mathbb Q\)
  is Galois and \([F:\mathbb Q]=4\).
\item
  Prove that \(F\) has the form \(F=\mathbb Q(\sqrt{d_1},\sqrt{d_2})\)
  where \(d_1,d_2\) are non-zero integers.
\item
  Suppose that \(G\) is the quaternionic group. Prove that \(d_1\) and
  \(d_2\) are positive integers.
\end{itemize}

\hypertarget{question-94}{%
\subsection{Question 94}\label{question-94}}

This question concerns the polynomial ring \(R=\mathbb Z[x,y]\) and the
ideal \(I=(5,x^2+2)\) in \(R\).

\begin{itemize}
\item
  Prove that \(I\) is a prime ideal of \(R\) and that \(R/I\) is a PID.
\item
  Give an explicit example of a maximal ideal of \(R\) which contains
  \(I\). (Give a set of generators for such an ideal.)
\item
  Show that there are infinitely many distinct maximal ideals in \(R\)
  which contain \(I\).
\end{itemize}

\hypertarget{question-95}{%
\subsection{Question 95}\label{question-95}}

Classify all groups of order 2012 up to isomorphism.

\begin{quote}
Hint: 503 is prime.
\end{quote}

\hypertarget{question-96}{%
\subsection{Question 96}\label{question-96}}

For any positive integer \(n\), let \(G_n\) be the group generated by
\(a\) and \(b\) subject to the following three relations: \begin{align*}
a^2=1, \quad b^2=1, \quad \text{and} \quad (ab)^n=1.
.\end{align*}

\begin{itemize}
\tightlist
\item
  Find the order of the group \(G_n\)
\end{itemize}

\hypertarget{question-97}{%
\subsection{Question 97}\label{question-97}}

Determine the Galois groups of the following polynomials over
\(\mathbb Q\).

\begin{itemize}
\item
  \(f(x)=x^4+4x^2+1\)
\item
  \(f(x)=x^4+4x^2-5\).
\end{itemize}

\hypertarget{question-98}{%
\subsection{Question 98}\label{question-98}}

Let \(R\) be a (commutative) principal ideal domain, let \(M\) and \(N\)
be finitely generated free \(R\)-modules, and let
\(\varphi:M\rightarrow N\) be an \(R\)-module homomorphism.

\begin{itemize}
\item
  Let \(K\) be the kernel of \(\varphi\). Prove that \(K\) is a direct
  summand of \(M\).
\item
  Let \(C\) be the image of \(\varphi\). Show by example (specifying
  \(R\), \(M\), \(N\), and \(\varphi\)) that \(C\) need not be a direct
  summand of \(N\).
\end{itemize}

\hypertarget{question-99}{%
\subsection{Question 99}\label{question-99}}

In this problem, as you apply Sylow's Theorem, state precisely which
portions you are using.

\begin{itemize}
\item
  Prove that there is no simple group of order 30.
\item
  Suppose that \(G\) is a simple group of order 60. Determine the number
  of \(p\)-Sylow subgroups of \(G\) for each prime \(p\) dividing 60,
  then prove that \(G\) is isomorphic to the alternating group \(A_5\).
\end{itemize}

\begin{quote}
Note: in the second part, you needn't show that \(A_5\) is simple. You
need only show that if there is a simple group of order 60, then it must
be isomorphic to \(A_5\).
\end{quote}

\hypertarget{question-100}{%
\subsection{Question 100}\label{question-100}}

Describe the Galois group and the intermediate fields of the cyclotomic
extension \(\mathbb Q(\zeta_{12})/\mathbb Q\).

\hypertarget{question-101}{%
\subsection{Question 101}\label{question-101}}

Let \begin{align*}
R=\mathbb Z[x]/(x^2+x+1).
\end{align*}

\begin{itemize}
\tightlist
\item
  Answer the following questions with suitable justification.

  \begin{itemize}
  \tightlist
  \item
    Is \(R\) a Noetherian ring?
  \item
    Is \(R\) an Artinian ring?
  \end{itemize}
\item
  Prove that \(R\) is an integrally closed domain.
\end{itemize}

\hypertarget{question-102}{%
\subsection{Question 102}\label{question-102}}

Let \(R\) be a commutative ring. Recall that an element \(r\) of \(R\)
is \textit{nilpotent} if \(r^n=0\) for some positive integer \(n\) and
that the \textit{nilradical} of \(R\) is the set \(N(R)\) of nilpotent
elements.

\begin{itemize}
\item
  Prove that \begin{align*}
  N(R)=\cap_{P\text{ prime}}P.
  .\end{align*}

  \begin{quote}
  Hint: given a non-nilpotent element \(r\) of \(R\), you may wish to
  construct a prime ideal that does not contain \(r\) or its powers.
  \end{quote}
\item
  Given a positive integer \(m\), determine the nilradical of
  \(\mathbb Z/(m)\).
\item
  Determine the nilradical of \(\mathbb C[x,y]/(y^2-x^3)\).
\item
  Let \(p(x,y)\) be a polynomial in \(\mathbb C[x,y]\) such that for any
  complex number \(a\), \(p(a,a^{3/2})=0\). Prove that \(p(x,y)\) is
  divisible by \(y^2-x^3\).
\end{itemize}

\hypertarget{question-103}{%
\subsection{Question 103}\label{question-103}}

Given a finite group \(G\), recall that its \emph{regular
representation} is the representation on the complex group algebra
\(\mathbb C[G]\) induced by left multiplication of \(G\) on itself and
its \textit{adjoint representation} is the representation on the complex
group algebra \(\mathbb C[G]\) induced by conjugation of \(G\) on
itself.

\begin{itemize}
\item
  Let \(G=\GL_2(\mathbb F_2)\). Describe the number and dimensions of
  the irreducible representations of \(G\). Then describe the
  decomposition of its regular representation as a direct sum of
  irreducible representations.
\item
  Let \(G\) be a group of order 12. Show that its adjoint representation
  is reducible; that is, there is an \(H\)-invariant subspace of
  \(\mathbb C[H]\) besides 0 and \(\mathbb C[H]\).
\end{itemize}

\hypertarget{question-104}{%
\subsection{Question 104}\label{question-104}}

Let \(R\) be a commutative integral domain. Show that the following are
equivalent:

\begin{itemize}
\item
  \(R\) is a field;
\item
  \(R\) is a semi-simple ring;
\item
  Any \(R\)-module is projective.
\end{itemize}

\hypertarget{question-105}{%
\subsection{Question 105}\label{question-105}}

Let \(p\) be a positive prime number, \(\mathbb F_p\) the field with
\(p\) elements, and let \(G=\text{GL}_2(\mathbb F_p)\).

\begin{itemize}
\item
  Compute the order of \(G\), \(|G|\).
\item
  Write down an explicit isomorphism from \(\mathbb Z/p\mathbb Z\) to
  \begin{align*}
  U=\left\{
  \begin{pmatrix}
    1 & a\\
    0 & 1
  \end{pmatrix}
  \bigg|a\in\mathbb F_p\right\}.
  \end{align*}
\item
  How many subgroups of order \(p\) does \(G\) have?

  \begin{quote}
  Hint: compute \(gug\inv\) for \(g\in G\) and \(u\in U\); use this to
  find the size of the normalizer of \(U\) in \(G\).
  \end{quote}
\end{itemize}

\hypertarget{question-106}{%
\subsection{Question 106}\label{question-106}}

\begin{itemize}
\item
  Give definitions of the following terms:

  \begin{enumerate}
  \def\labelenumi{(\roman{enumi})}
  \tightlist
  \item
    a finite length (left) module, (ii) a composition series for a
    module, and (iii) the length of a module,
  \end{enumerate}
\item
  Let \(l(M)\) denote the length of a module \(M\). Prove that if
  \begin{align*}
  0\rightarrow M_1\rightarrow M_2\rightarrow\dots\rightarrow
  M_n\rightarrow 0
  .\end{align*}

  is an exact sequence of modules of finite length, then \begin{align*}
  \sum_{i=1}^n(-1)^kl(M_i)=0.
  .\end{align*}
\end{itemize}

\hypertarget{question-107}{%
\subsection{Question 107}\label{question-107}}

Let \(\mathbb F\) be a field of characteristic \(p\), and \(G\) a group
of order \(p^n\). Let \(R=\mathbb F[G]\) be the group ring (group
algebra) of \(G\) over \(\mathbb F\), and let \(u:=\sum_{x\in G}x\) (so
\(u\) is an element of \(R\)).

\begin{itemize}
\item
  Prove that \(u\) lies in the center of \(R\).
\item
  Verify that \(Ru\) is a 2-sided ideal of \(R\).
\item
  Show there exists a positive integer \(k\) such that \(u^k=0\).
  Conclude that for such a \(k\), \((Ru)^k=0\).
\item
  Show that \(R\) is \textbf{not} a semi-simple ring.

  \begin{quote}
  \textbf{Warning:} Please use the definition of a semi-simple ring: do
  \textbf{not} use the result that a finite length ring fails to be
  semisimple if and only if it has a non-zero nilpotent ideal.
  \end{quote}
\end{itemize}

\hypertarget{question-108}{%
\subsection{Question 108}\label{question-108}}

Let \(f(x)=a_nx^n+a_{n-1}x^{n-1}+\dots+a_0\in\mathbb Z[x]\) (where
\(a_n\neq 0\)) and let \(R=\mathbb Z[x]/(f)\). Prove that \(R\) is a
finitely generated module over \(\mathbb Z\) if and only if
\(a_n=\pm 1\).

\hypertarget{question-109}{%
\subsection{Question 109}\label{question-109}}

Consider the ring \begin{align*}
S=C[0,1]=\{f:[0,1]\rightarrow\mathbb R:f\text{ is continuous}\}
.\end{align*}

with the usual operations of addition and multiplication of functions.

\begin{itemize}
\item
  What are the invertible elements of \(S\)?
\item
  For \(a\in[0,1]\), define \(I_a=\{f\in S:f(a)=0\}\). Show that \(I_a\)
  is a maximal ideal of \(S\).
\item
  Show that the elements of any proper ideal of \(S\) have a common
  zero, i.e., if \(I\) is a proper ideal of \(S\), then there exists
  \(a\in[0,1]\) such that \(f(a)=0\) for all \(f\in I\). Conclude that
  every maximal ideal of \(S\) is of the form \(I_a\) for some
  \(a\in[0,1]\).

  \begin{quote}
  \textbf{Hint}: As \([0,1]\) is compact, every open cover of \([0,1]\)
  contains a finite subcover.
  \end{quote}
\end{itemize}

\hypertarget{question-110}{%
\subsection{Question 110}\label{question-110}}

Let \(F\) be a field of characteristic zero, and let \(K\) be an
\emph{algebraic} extension of \(F\) that possesses the following
property: every polynomial \(f\in F[x]\) has a root in \(K\). Show that
\(K\) is algebraically closed.\textbackslash{}

\begin{quote}
\textbf{Hint:} if \(K(\theta)/K\) is algebraic, consider \(F(\theta)/F\)
and its normal closure; primitive elements might be of help.
\end{quote}

\hypertarget{question-111}{%
\subsection{Question 111}\label{question-111}}

Let \(G\) be the unique non-abelian group of order 21.

\begin{itemize}
\item
  Describe all 1-dimensional complex representations of \(G\).
\item
  How many (non-isomorphic) irreducible complex representations does
  \(G\) have and what are their dimensions?
\item
  Determine the character table of \(G\).
\end{itemize}

\hypertarget{question-112}{%
\subsection{Question 112}\label{question-112}}

\begin{itemize}
\item
  Classify all groups of order \(2009=7^2\times 41\).
\item
  Suppose that \(G\) is a group of order 2009. How many intermediate
  groups are there---that is, how many groups H are there with
  \(1\subsetneq H\subsetneq G\), where both inclusions are proper?
  (There may be several cases to consider.)
\end{itemize}

\hypertarget{question-113}{%
\subsection{Question 113}\label{question-113}}

Let \(K\) be a field. A discrete valuation on \(K\) is a function
\(\nu: K\setminus\{0\}\rightarrow\mathbb Z\) such that

\begin{itemize}
\item
  \(\nu(ab)=\nu(a)+\nu(b)\)
\item
  \(\nu\) is surjective
\item
  \(\nu(a+b)\geq\text{min}\{(\nu(a),\nu(b)\}\) for
  \(a,b\in K\setminus\{0\}\) with \(a+b\neq 0\).
\end{itemize}

Let \(R:=\{x\in K\setminus\{0\}:\nu(x)\geq0\}\cup\{0\}\). Then \(R\) is
called the valuation ring of \(\nu\).

Prove the following:

\begin{itemize}
\item
  \(R\) is a subring of \(K\) containing the 1 in \(K\).
\item
  for all \(x\in K\setminus\{0\}\), either \(x\) or \(x\inv\) is in
  \(R\).
\item
  \(x\) is a unit of \(R\) if and only if \(\nu(x)=0\).
\item
  Let \(p\) be a prime number, \(K=\mathbb Q\), and
  \(\nu_p:\mathbb Q\setminus\{0\}\rightarrow\mathbb Z\) be the function
  defined by \(\nu_p(\frac ab)=n\) where \(\frac ab=p^n\frac cd\) and
  \(p\) does not divide \(c\) and \(d\). Prove that the corresponding
  valuation ring \(R\) is the ring of all rational numbers whose
  denominators are relatively prime to \(p\).
\end{itemize}

\hypertarget{question-114}{%
\subsection{Question 114}\label{question-114}}

Let \(F\) be a field of characteristic not equal to 2.

\begin{itemize}
\item
  Prove that any extension \(K\) of \(F\) of degree 2 is of the form
  \(F(\sqrt D)\) where \(D\in F\) is not a square in \(F\) and,
  conversely, that each such extension has degree 2 over \(F\).
\item
  Let \(D_1,D_2\in F\) neither of which is a square in \(F\). Prove that
  \([F(\sqrt{D_1},\sqrt{D_2}):F]=4\) if \(D_1D_2\) is not a square in
  \(F\) and is of degree 2 otherwise.
\end{itemize}

\hypertarget{question-115}{%
\subsection{Question 115}\label{question-115}}

Let \(F\) be a field and \(p(x)\in F[x]\) an irreducible polynomial.

\begin{itemize}
\item
  Prove that there exists a field extension \(K\) of \(F\) in which
  \(p(x)\) has a root.
\item
  Determine the dimension of \(K\) as a vector space over \(F\) and
  exhibit a vector space basis for \(K\).
\item
  If \(\theta\in K\) denotes a root of \(p(x)\), express \(\theta\inv\)
  in terms of the basis found in part (b).
\item
  Suppose \(p(x)=x^3+9x+6\). Show \(p(x)\) is irreducible over
  \(\mathbb Q\). If \(\theta\) is a root of \(p(x)\), compute the
  inverse of \((1+\theta)\) in \(\mathbb Q(\theta)\).
\end{itemize}

\hypertarget{question-116}{%
\subsection{Question 116}\label{question-116}}

Fix a ring \(R\), an \(R\)-module \(M\), and an \(R\)-module
homomorphism \(f:M\rightarrow M\).

\begin{itemize}
\item
  If \(M\) satisfies the descending chain condition on submodules, show
  that if \(f\) is injective, then \(f\) is surjective.

  \begin{quote}
  Hint: note that if \(f\) is injective, so are \(f\circ f\),
  \(f\circ f\circ f\), etc.
  \end{quote}
\item
  Give an example of a ring \(R\), an \(R\)-module \(M\), and an
  injective \(R\)-module homomorphism \(f:M\rightarrow M\) which is not
  surjective.
\item
  If \(M\) satisfies the ascending chain condition on submodules, show
  that if \(f\) is surjective, then \(f\) is injective.
\item
  Give an exampe of a ring \(R\), and \(R\)-module \(M\), and a
  surjective \(R\)-module homomorphism \(f:M\rightarrow M\) which is not
  injective.
\end{itemize}

\hypertarget{question-117}{%
\subsection{Question 117}\label{question-117}}

Let \(G\) be a finite group, \(k\) an algebraically closed field, and
\(V\) an irreducible \(k\)-linear representation of \(G\).

\begin{itemize}
\item
  Show that \(\hom_{kG}(V,V)\) is a division algebra with \(k\) in its
  center.
\item
  Show that \(V\) is finite-dimensional over \(k\), and conclude that
  \(\hom_{kG}(V,V)\) is also finite dimensional.
\item
  Show the inclusion \(k\hookrightarrow\hom_{kG}(V,V)\) found in (a) is
  an isomorphism. (For \(f\in\hom_{kG}(V,V)\), view \(f\) as a linear
  transformation and consider \(f-\alpha I\), where \(\alpha\) is an
  eigenvalue of \(f\)).
\end{itemize}

\hypertarget{question-118}{%
\subsection{Question 118}\label{question-118}}

Let \(f(x)\) be an irreducible polynomial of degree 5 over the field
\(\mathbb Q\) of rational numbers with exactly 3 real roots.

\begin{itemize}
\item
  Show that \(f(x)\) is not solvable by radicals.
\item
  Let \(E\) be the splitting field of \(f\) over \(\mathbb Q\).
  Construct a Galois extension \(K\) of degree 2 over \(\mathbb Q\)
  lying in \(E\) such that \textit{no} field \(F\) strictly between
  \(K\) and \(E\) is Galois over \(\mathbb Q\).
\end{itemize}

\hypertarget{question-119}{%
\subsection{Question 119}\label{question-119}}

Let \(F\) be a finite field. Show for any positive integer \(n\) that
there are irreducible polynomials of degree \(n\) in \(F[x]\).

\hypertarget{question-120}{%
\subsection{Question 120}\label{question-120}}

Show that the order of the group \(\text{GL}_n(\mathbb F_q)\) of
invertible \(n\times n\) matrices over the field \(\mathbb F_q\) of
\(q\) elements is given by \((q^n-1)(q^n-q)\dots(q^n-q^{n-1})\).

\hypertarget{question-121}{%
\subsection{Question 121}\label{question-121}}

\begin{itemize}
\item
  Let \(R\) be a commutative principal ideal domain. Show that any
  \(R\)-module \(M\) generated by two elements takes the form
  \(R/(a)\oplus R/(b)\) for some \(a,b\in R\). What more can you say
  about \(a\) and \(b\)?
\item
  Give a necessary and sufficient condition for two direct sums as in
  part (a) to be isomorphic as \(R\)-modules.
\end{itemize}

\hypertarget{question-122}{%
\subsection{Question 122}\label{question-122}}

Let \(G\) be the subgroup of \(\text{GL}_3(\mathbb C)\) generated by the
three matrices \begin{align*}
A=
\begin{pmatrix}
  0 & 0 & 1\\
  0 & 1 & 0\\
  1 & 0 & 0
\end{pmatrix},
\quad B=
\begin{pmatrix}
  0 & 0 & 1\\
  1 & 0 & 0\\
  0 & 1 & 0
\end{pmatrix},
\quad C=
\begin{pmatrix}
  i & 0 & 0\\
  0 & 1 & 0\\
  0 & 0 & 1
\end{pmatrix}
\end{align*}

where \(i^2=-1\). Here \(\mathbb C\) denotes the complex field.

\begin{itemize}
\item
  Compute the order of \(G\).
\item
  Find a matrix in \(G\) of largest possible order (as an element of
  \(G\)) and compute this order.
\item
  Compute the number of elements in \(G\) with this largest order.
\end{itemize}

\hypertarget{question-123}{%
\subsection{Question 123}\label{question-123}}

\begin{itemize}
\item
  Let \(G\) be a group of (finite) order \(n\). Show that any
  irreducible left module over the group algebra \(\mathbb CG\) has
  complex dimension at least \(\sqrt n\).
\item
  Give an example of a group \(G\) of order \(n\geq5\) and an
  irreducible left module over \(\mathbb CG\) of complex dimension
  \(\lfloor\sqrt n\rfloor\), the greatest integer to \(\sqrt n\).
\end{itemize}

\hypertarget{question-124}{%
\subsection{Question 124}\label{question-124}}

Use the rational canonical form to show that any square matrix \(M\)
over a field \(k\) is similar to its transpose \(M^t\), recalling that
\(p(M)=0\) for some \(p\in k[t]\) if and only if \(p(M^t)=0\).

\hypertarget{question-125}{%
\subsection{Question 125}\label{question-125}}

Let \(K\) be a field of characteristic zero and \(L\) a Galois extension
of \(K\). Let \(f\) be an irreducible polynomial in \(K[x]\) of degree 7
and suppose \(f\) has no zeroes in \(L\). Show that \(f\) is irreducible
in \(L[x]\).

\hypertarget{question-126}{%
\subsection{Question 126}\label{question-126}}

Let \(K\) be a field of characteristic zero and \(f\in K[x]\) an
irreducible polynomial of degree \(n\). Let \(L\) be a splitting field
for \(f\). Let \(G\) be the group of automorphisms of \(L\) which act
trivially on \(K\).

\begin{itemize}
\item
  Show that \(G\) embeds in the symmetric group \(S_n\).
\item
  For each \(n\), give an example of a field \(K\) and polynomial \(f\)
  such that \(G=S_n\).
\item
  What are the possible groups \(G\) when \(n=3\). Justify your answer.
\end{itemize}

\hypertarget{question-127}{%
\subsection{Question 127}\label{question-127}}

Show there are exactly two groups of order 21 up to isomorphism.

\hypertarget{question-128}{%
\subsection{Question 128}\label{question-128}}

Let \(K\) be the field \(\mathbb Q(z)\) of rational functions in a
variable \(z\) with coeffiecients in the rational field \(\mathbb Q\).
Let \(n\) be a positive integer. Consider the polynomial
\(x^n-z\in K[x]\).

\begin{itemize}
\item
  Show that the polynomial \(x^n-z\) is irreducible over \(K\).
\item
  Describe the splitting field of \(x^n-z\) over \(K\).
\item
  Determine the Galois group of the splitting field of \(x^5-z\) over
  the field \(K\).
\end{itemize}

\hypertarget{question-129}{%
\subsection{Question 129}\label{question-129}}

\begin{itemize}
\item
  Let \(p<q<r\) be prime integers. Show that a group of order \(pqr\)
  cannot be simple.
\item
  Consider groups of orders \(2^2\cdot 3\cdot p\) where \(p\) has the
  values 5, 7, and 11. For each of those values of \(p\), either display
  a simple group of order \(2^2\cdot 3\cdot p\), or show that there
  cannot be a simple group of that order.
\end{itemize}

\hypertarget{question-130}{%
\subsection{Question 130}\label{question-130}}

Let \(K/F\) be a finite Galois extension and let \(n=[K:F]\). There is a
theorem (often referred to as the ``normal basis theorem'') which states
that there exists an irreducible polynomial \(f(x)\in F[x]\) whose roots
form a basis for \(K\) as a vector space over \(F\). You may assume that
theorem in this problem.

\begin{itemize}
\item
  Let \(G=\Gal(K/F)\). The action of \(G\) on \(K\) makes \(K\) into a
  finite-dimensional representation space for \(G\) over \(F\). Prove
  that \(K\) is isomorphic to the regular representation for \(G\) over
  \(F\).

  \begin{quote}
  The regular representation is defined by letting \(G\) act on the
  group algebra \(F[G]\) by multiplication on the left.
  \end{quote}
\item
  Suppose that the Galois group \(G\) is cyclic and that \(F\) contains
  a primitive \(n^{\text{th}}\) root of unity. Show that there exists an
  injective homomorphism \(\chi:G\rightarrow F^{\times}\).
\item
  Show that \(K\) contains a non-zero element \(a\) with the following
  property: \begin{align*}
  g(a)=\chi(g)\cdot a
  .\end{align*}

  for all \(g\in G\).
\item
  If \(a\) has the property stated in (c), show that \(K=F(a)\) and that
  \(a^n\in F^{\times}\).
\end{itemize}

\hypertarget{question-131}{%
\subsection{Question 131}\label{question-131}}

Let \(G\) be the group of matrices of the form \begin{align*}
\begin{pmatrix}
  1 & a & b\\
  0 & 1 & c\\
  0 & 0 & 1
\end{pmatrix}
.\end{align*}

with entries in the finite field \(\mathbb F_p\) of \(p\) element, where
\(p\) is a prime.

\begin{itemize}
\item
  Prove that \(G\) is non-abelian.
\item
  Suppose \(p\) is odd. Prove that \(g^p=I_3\) for all \(g\in G\).
\item
  Suppose that \(p=2\). It is known that there are exactly two
  non-abelian groups of order 8, up to isomorphism: the dihedral group
  \(D_8\) and the quaternionic group. Assuming this fact without proof,
  determine which of these groups \(G\) is isomorphic to.
\end{itemize}

\hypertarget{question-132}{%
\subsection{Question 132}\label{question-132}}

There are five nonisomorphic groups of order 8. For each of those groups
\(G\), find the smallest positive integer n such that there is an
injective homomorphism \(\varphi: G\rightarrow S_n\).

\hypertarget{question-133}{%
\subsection{Question 133}\label{question-133}}

For any group \(G\) we define \(\Omega(G)\) to be the image of the group
homomorphism \(\rho:G\rightarrow\Aut(G)\) where \(\rho\) maps \(g\in G\)
to the conjugation automorphism \(x\mapsto gxg\inv\). Starting with a
group \(G_0\), we define \(G_1=\Omega(G_0)\) and \(G_{i+1}=\Omega(G_i)\)
for all \(i\geq 0\). If \(G_0\) is of order \(p^e\) for a prime \(p\)
and integer \(e\geq 2\), prove that \(G_{e-1}\) is the trivial group.

\hypertarget{question-134}{%
\subsection{Question 134}\label{question-134}}

Let \(\mathbb F_2\) be the field with two elements.

\begin{itemize}
\item
  What is the order of \(\text{GL}_3(\mathbb F_2)\)?
\item
  Use the fact that \(\text{GL}_3(\mathbb F_2)\) is a simple group
  (which you should not prove) to find the number of elements of order 7
  in \(\text{GL}_3(\mathbb F_2)\).
\end{itemize}

\hypertarget{question-135}{%
\subsection{Question 135}\label{question-135}}

Let \(G\) be a finite abelian group. Let \(f:\mathbb Z^m\rightarrow G\)
be a surjection of abelian groups. We may think of \(f\) as a
homomorphism of \(\mathbb Z\)-modules. Let \(K\) be the kernel of \(f\).

\begin{itemize}
\item
  Prove that \(K\) is isomorphic to \(\mathbb Z^m\).
\item
  We can therefore write the inclusion map \(K\rightarrow\mathbb Z^m\)
  as \(\mathbb Z^m\rightarrow\mathbb Z^m\) and represent it by an
  \(m\times m\) integer matrix \(A\). Prove that \(|\det A|=|G|\).
\end{itemize}

\hypertarget{question-136}{%
\subsection{Question 136}\label{question-136}}

Let \(R=C([0,1])\) be the ring of all continuous real-valued functions
on the closed interval \([0,1]\), and for each \(c\in[0,1]\), denote by
\(M_c\) the set of all functions \(f\in R\) such that \(f(c)=0\).

\begin{itemize}
\item
  Prove that \(g\in R\) is a unit if and only if \(g(c)\neq0\) for all
  \(c\in[0,1]\).
\item
  Prove that for each \(c\in[0,1]\), \(M_c\) is a maximal ideal of
  \(R\).
\item
  Prove that if \(M\) is a maximal ideal of \(T\), then \(M=M_c\) for
  some \(c\in[0,1]\).

  \begin{quote}
  Hint: compactness of \([0,1]\) may be relevant.
  \end{quote}
\end{itemize}

\hypertarget{question-137}{%
\subsection{Question 137}\label{question-137}}

Let \(R\) and \(S\) be commutative rings, and \(f:R\rightarrow S\) a
ring homomorphism.

\begin{itemize}
\item
  Show that if \(I\) is a prime ideal of \(S\), then \begin{align*}
  f\inv(I)=\{r\in R:f(r)\in I\}
  \end{align*}

  is a prime ideal of \(R\).
\item
  Let \(N\) be the set of nilpotent elements of \(R\): \begin{align*}
  N=\{r\in R:r^m=0\text{ for some }m\geq 1\}.
  .\end{align*}

  \(N\) is called the \textit{nilradical} of \(R\). Prove that it is an
  ideal which is contained in every prime ideal.
\item
  Part (a) lets us define a function \begin{align*}
  f^*:\{\text{prime ideals of }S\} &\rightarrow
  \{\text{prime ideals of }R\}.
  I &\mapsto f\inv(I).
  .\end{align*}

  Let \(N\) be the nilradical of \(R\). Show that if \(S=R/N\) and
  \(f:R\rightarrow R/N\) is the quotient map, then \(f^*\) is a
  bijection
\end{itemize}

\hypertarget{question-138}{%
\subsection{Question 138}\label{question-138}}

Consider the polynomial \(f(x)=x^{10}+x^5+1\in\mathbb Q[x]\) with
splitting field \(K\) over \(\mathbb Q\).

\begin{itemize}
\item
  Determine whether \(f(x)\) is irreducible over \(\mathbb Q\) and find
  \([K:\mathbb Q]\).
\item
  Determine the structure of the Galois group \(\Gal(K/\mathbb Q)\).
\end{itemize}

\hypertarget{question-139}{%
\subsection{Question 139}\label{question-139}}

For each prime number \(p\) and each positive integer \(n\), how many
elements \(\alpha\) are there in \(\mathbb F_{p^n}\) such that
\(F_p(\alpha)=F_{p^6}\)?

\hypertarget{question-140}{%
\subsection{Question 140}\label{question-140}}

Assume that \(K\) is a cyclic group, \(H\) is an arbitrary group, and
\(\varphi_1\) and \(\varphi_2\) are homomorphisms from \(K\) into
\(\Aut(H)\) such that \(\varphi_1(K)\) and \(\varphi_2(K)\) are
conjugate subgroups of \(\Aut(H)\).

Prove by constructing an explicit isomorphism that
\(H\rtimes_{\varphi_1}K\cong H\rtimes_{\varphi_2} K\).

\begin{quote}
Suppose \(\sigma_{\varphi_1}(K)\sigma\inv=\varphi_2(K)\) so that for
some \(a\in\mathbb Z\) we have
\(\sigma\varphi_1(k)\sigma\inv =\varphi_2(k)^a\) for all \(k\in K\).
Show that the map
\(\psi:H \rtimes_{\varphi_1}K\rightarrow H\rtimes_{\varphi_2}K\) defined
by \(\psi((h,k))=(\sigma(h),k^a)\) is a homomorphism. Show \(\psi\) is
bijective by construcing a 2-sided inverse.
\end{quote}

\hypertarget{real-analysis-85-questions}{%
\section{Real Analysis (85
Questions)}\label{real-analysis-85-questions}}

\hypertarget{question-1-1}{%
\subsection{Question 1}\label{question-1-1}}

Prove or disprove each of the following statements.

\begin{enumerate}
\def\labelenumi{(\alph{enumi})}
\item
  If \(f\) is of bounded variation on \([0,1]\), then it is continuous
  on \([0,1]\).
\item
  If \(f : [0, 1] \to [0, 1]\) is a continuous function, then there
  exists \(x_0 \in [0, 1]\) such that \(f(x_0) = x_0\).
\item
  Let \(\{f_n\}\) be a sequence of uniformly continuous functions on an
  interval \(I\). If \(\{f_n\}\) converges uniformly to a function \(f\)
  on \(I\), then \(f\) is also uniformly continuous on \(I\).
\item
  If \(f\) is differentiable on a connected set
  \(E \subset \mathbb{R}^n\), then for any \(x, y \in E\), there exists
  \(z \in E\) such that \(f(x) - f(y) = \nabla f(z)(x - y)\).
\end{enumerate}

\hypertarget{question-2-1}{%
\subsection{Question 2}\label{question-2-1}}

Prove or disprove each of the following statements.

\begin{enumerate}
\def\labelenumi{(\alph{enumi})}
\setcounter{enumi}{3}
\item
  If \(\lim_{n\to\infty} |a_n+1/a_n|\) exists, then
  \(\lim_{n\to \infty} |a_n|^{1/n}\) exists and the two limits are
  equal.
\item
  If \(\sum_{n=1}^\infty a_n x^n\) converges for all \(x \in [0, 1]\),
  then
  \(\lim_{x\to 1^-} \sum_{n=1}^\infty a_n x^n=\sum_{n=1}^\infty a_n\)
\end{enumerate}

\hypertarget{question-3-1}{%
\subsection{Question 3}\label{question-3-1}}

Prove or disprove each of the following statements.

\begin{enumerate}
\def\labelenumi{(\alph{enumi})}
\setcounter{enumi}{5}
\item
  If \(E \subset \mathbb{R}\) and

  \(\mu(E) = \inf\{\sum_{I_i \in S} |I_i| : S = \{I_i\}_{i=1}^n \text{ such that } E \subset \union_{i=1}^n I_i \text{ for some } n \in \mathbb{N}\}\)

  then \(\mu\) coincides with the outer measure of \(E\).
\item
  If \(E\) is a Borel set and \(f\) is a measurable function, then
  \(f^{-1}(E)\) is also measurable.
\end{enumerate}

\hypertarget{question-4-1}{%
\subsection{Question 4}\label{question-4-1}}

If \(f\) is a finite real valued measurable function on a measurable set
\(E \subset \mathbb{R}\), show that the set \(\{(x, f(x)) : x \in E\}\)
is measurable.

\hypertarget{question-5-1}{%
\subsection{Question 5}\label{question-5-1}}

Let g : \([0, 1] \times [0, 1] \to [0, 1]\) be a continuous function and
let \(\{f_n\}\) be a sequence of functions such that

\[f_n(x)=\begin{cases}{0,   0\leq x\leq 1/n},\\{\int_0^{x-\frac1n} g(t,f_n(t))dt, 1/n\leq x \leq 1.}\end{cases}\]

With the help of the Arzela-Ascoli theorem or otherwise, show that there
exists a continuous function \(f : [0, 1] \to \mathbb{R}\) such that

\(f(x) = \int_0^x g(t, f(t))dt\)

for all \(x \in [0, 1]\).

\begin{quote}
Hint: first show that \(|f_n(x_1) - f_n(x_2)| \leq |x_1 - x_2|\).
\end{quote}

\hypertarget{question-6-1}{%
\subsection{Question 6}\label{question-6-1}}

If \(\limsup_{n\rightarrow \infty} a_n\leq l\), show that
\(\limsup_{n\rightarrow \infty}\sum_{i=1}^n{a_i/n}\leq l\).

\hypertarget{question-7-1}{%
\subsection{Question 7}\label{question-7-1}}

If \(f\) is a nonnegative measurable function on \(\mathbb{R}\) and
\(p > 0\), show that
\[\int f^p ~dx = \int_0^{\infty} p t^{p-1} \abs{\{x : f(x) > t\}} ~dt\]
where \(\abs{\{x : f(x) > t\}}\) is the Lebesgue measure of the set
\(\{x : f(x) > t\}\).

\hypertarget{question-8-1}{%
\subsection{Question 8}\label{question-8-1}}

If \(f\) is a nonnegative measurable function on \([0, \pi]\) and
\(\int_0^\pi f(x)^3~dx < \infty\), show that \begin{align*}
\lim_{\alpha\to\infty} \int_{ \theset{x :f(x) > \alpha} } f(x)^2 ~dx=0
.\end{align*}

\hypertarget{question-9-1}{%
\subsection{Question 9}\label{question-9-1}}

Prove or disprove each of the following statements.

\begin{enumerate}
\def\labelenumi{(\alph{enumi})}
\item
  If \(f : [0, 1] \to \mathbb{R}\) is a measurable function, then given
  any \(\varepsilon > 0\), there exists a compact set
  \(K \subset [0, 1]\) such that \(f\) is continuous on \(K\) relative
  to \(K\).
\item
  If f is Borel measurable on \(\mathbb{R} \times \mathbb{R}\), then for
  any \(x \in \mathbb{R}\), the function \(g(y) = f(x, y)\) is also
  Borel measurable on \(\mathbb{R}\).
\item
  If \(E \subset \mathbb{R}\), then \(E\) is measurable if and only if
  given any \(\varepsilon > 0\), there exist a closed set \(F\) and an
  open set \(G\) such that \(F \subset E \subset G\) and the measure of
  \(G-F\) is less than \(\varepsilon\).
\end{enumerate}

\hypertarget{question-10-1}{%
\subsection{Question 10}\label{question-10-1}}

Prove or disprove each of the following statements.

\begin{enumerate}
\def\labelenumi{(\alph{enumi})}
\setcounter{enumi}{1}
\item
  If \({f_n}\) is a sequence of measurable functions that converges
  uniformly to \(f\) on \(\mathbb{R}\), then
  \(\int{f}=\lim_{k\to \infty} \int f_k\)
\item
  If \(\{f_k\}\) is a sequence of function in \(L_p[0,\infty)\) that
  converges to a function \(f \in L_p [0,\infty)\), then \(\{f_k\}\) has
  a subsequence that converges to \(f\) almost everywhere.
\end{enumerate}

\hypertarget{question-11-1}{%
\subsection{Question 11}\label{question-11-1}}

Prove or disprove each of the following statements.

\begin{enumerate}
\def\labelenumi{(\alph{enumi})}
\setcounter{enumi}{5}
\item
  If \(f\) is Riemann integrable on \([\eps, 1]\) for all
  \(0 < \eps < 1\), then \(f\) is Lebesgue integrable on \([0,1]\) if
  \(f\) is nonnegative and the following limit exists
  \(\lim_{\varepsilon\to 0^+} \int_\varepsilon^1 f dx\).
\item
  If \(f\) is integrable on \([0,1]\), then
  \(\lim_{n\to\infty} \int_0^1 f(x)\sin(n\pi x)dx = 0\).
\item
  If \(f\) is continuous on \([0, 1]\), then it is of bounded variation
  on {[}0, 1{]}\$.
\end{enumerate}

\hypertarget{question-12-1}{%
\subsection{Question 12}\label{question-12-1}}

\begin{enumerate}
\def\labelenumi{(\alph{enumi})}
\item
  Let \(f : \mathbb{R} \to \mathbb{R}\) be a differentiable function. If
  \(f'(-1) < 2\) and \(f'(1) > 2\), show that there exists
  \(x_0 \in (i1, 1)\) such that \(f'(x_0) = 2\).

  \begin{quote}
  Hint: consider the function \(f(x) - 2x\) and recall the proof of
  Rolle's theorem.)
  \end{quote}
\item
  Let \(f : (-1, 1) \to \mathbb{R}\) be a differentiable function on
  \((-1, 0) \union (0, 1)\) such that \(\lim_{x\to 0} f'(x) = L\). If
  \(f\) is continuous on \((-1, 1)\), show that \(f\) is indeed
  differentiable at \(0\) and \(f'(0) = L\).
\end{enumerate}

\hypertarget{question-13-1}{%
\subsection{Question 13}\label{question-13-1}}

Let \(C([0, 1])\) denote the space of all continuous real-valued
functions on \([0, 1]\).

\begin{enumerate}
\def\labelenumi{\alph{enumi}.}
\tightlist
\item
  Prove that \(C([0, 1])\) is complete under the uniform norm
  \(\norm{f}_u := \displaystyle\sup_{x\in [0,1]} |f (x)|\).
\item
  Prove that \(C([0, 1])\) is not complete under the \(L^1\dash\)norm
  \(\norm{f}_1 = \displaystyle\int_0^1 |f (x)| ~dx\).
\end{enumerate}

\hypertarget{question-14-1}{%
\subsection{Question 14}\label{question-14-1}}

Let \(\mathcal B\) denote the set of all Borel subsets of \(\RR\) and
\(\mu : \mathcal B \to [0, \infty)\) denote a finite Borel measure on
\(\RR\).

\begin{enumerate}
\def\labelenumi{\alph{enumi}.}
\item
  Prove that if \(\{F_k\}\) is a sequence of Borel sets for which
  \(F_k \supseteq F_{k+1}\) for all \(k\), then \[
  \lim _{k \rightarrow \infty} \mu\left(F_{k}\right)=\mu\left(\bigcap_{k=1}^{\infty} F_{k}\right)
  \]
\item
  Suppose \(mu\) has the property that \(mu(E) = 0\) for every
  \(E \in \mathcal B\) with Lebesgue measure \(m(E) = 0\). Prove that
  for every \(\eps > 0\) there exists \(\delta > 0\) so that if
  \(E \in \mathcal B\) with \(m(E) < \delta\), then \(mu(E) < \eps\).
\end{enumerate}

\hypertarget{question-15-1}{%
\subsection{Question 15}\label{question-15-1}}

Let \(\{f_k\}\) be any sequence of functions in \(L^2([0, 1])\)
satisfying \(\norm{f_k}_2 \leq M\) for all \(k \in \NN\).

Prove that if \(f_k \to f\) almost everywhere, then
\(f \in L^2([0, 1])\) with \(\norm{f}_2 \leq M\) and \[
\lim _{k \rightarrow \infty} \int_{0}^{1} f_{k}(x) dx = \int_{0}^{1} f(x) d x
\]

\begin{quote}
Hint: Try using Fatou's Lemma to show that \(\norm{f}_2 \leq M\) and
then try applying Egorov's Theorem.
\end{quote}

\hypertarget{question-16-1}{%
\subsection{Question 16}\label{question-16-1}}

Let \(f\) be a non-negative function on \(\RR^n\) and
\(\mathcal A = \{(x, t) \in \RR^n \times \RR : 0 \leq t \leq f (x)\}\).

Prove the validity of the following two statements:

\begin{enumerate}
\def\labelenumi{\alph{enumi}.}
\item
  \(f\) is a Lebesgue measurable function on \(\RR^n \iff \mathcal A\)
  is a Lebesgue measurable subset of \(\RR^{n+1}\)
\item
  If \(f\) is a Lebesgue measurable function on \(\RR^n\), then \[
  m(\mathcal{A})=\int_{\mathbb{R}^{n}} f(x) d x=\int_{0}^{\infty} m\left(\left\{x \in \mathbb{R}^{n}: f(x) \geq t\right\}\right) d t
  \]
\end{enumerate}

\hypertarget{question-17-1}{%
\subsection{Question 17}\label{question-17-1}}

\begin{enumerate}
\def\labelenumi{\alph{enumi}.}
\item
  Show that \(L^2([0, 1]) \subseteq L^1([0, 1])\) and argue that
  \(L^2([0, 1])\) in fact forms a dense subset of \(L^1([0, 1])\).
\item
  Let \(\Lambda\) be a continuous linear functional on \(L^1([0, 1])\).

  Prove the Riesz Representation Theorem for \(L^1([0, 1])\) by
  following the steps below:

  \begin{enumerate}
  \def\labelenumii{\roman{enumii}.}
  \tightlist
  \item
    Establish the existence of a function \(g \in L^2([0, 1])\) which
    represents \(\Lambda\) in the sense that \[
    \Lambda(f ) = f (x)g(x) dx \text{ for all } f \in L^2([0, 1]).
    \]
  \end{enumerate}

  \begin{quote}
  Hint: You may use, without proof, the Riesz Representation Theorem for
  \(L^2([0, 1])\).
  \end{quote}

  \begin{enumerate}
  \def\labelenumii{\roman{enumii}.}
  \setcounter{enumii}{1}
  \tightlist
  \item
    Argue that the \(g\) obtained above must in fact belong to
    \(L^\infty([0, 1])\) and represent \(\Lambda\) in the sense that \[
    \Lambda(f)=\int_{0}^{1} f(x) \overline{g(x)} d x \quad \text { for all } f \in L^{1}([0,1])
    \] with \[
    \|g\|_{L^{\infty}([0,1])}=\|\Lambda\|_{L^{1}([0,1])\dual}
    \]
  \end{enumerate}
\end{enumerate}

\hypertarget{question-18-1}{%
\subsection{Question 18}\label{question-18-1}}

Let \(\{a_n\}_{n=1}^\infty\) be a sequence of real numbers.

\begin{enumerate}
\def\labelenumi{\alph{enumi}.}
\item
  Prove that if \(\displaystyle\lim_{n\to\infty} a_n = 0\), then
  \(\displaystyle\lim_{n\to\infty} a_1 + \cdots + a_n = 0\). \[
  \lim _{n \rightarrow \infty} \frac{a_{1}+\cdots+a_{n}}{n}=0
  \]
\item
  Prove that if \(\displaystyle\sum_{n=1}^{\infty} \frac{a_{n}}{n}\)
  converges, then \[
  \lim _{n \rightarrow \infty} \frac{a_{1}+\cdots+a_{n}}{n}=0
  \]
\end{enumerate}

\hypertarget{question-19-1}{%
\subsection{Question 19}\label{question-19-1}}

Prove that \[
\left|\frac{d^{n}}{d x^{n}} \frac{\sin x}{x}\right| \leq \frac{1}{n}
\]

for all \(x \neq 0\) and positive integers \(n\).

\begin{quote}
Hint: Consider \(\displaystyle\int_0^1 \cos(tx) dt\)
\end{quote}

\hypertarget{question-20-1}{%
\subsection{Question 20}\label{question-20-1}}

Let \((X, \mathcal B, mu)\) be a measure space with \(mu(X) = 1\) and
\(\{B_n\}_{n=1}^\infty\) be a sequence of \(\mathcal B\)-measurable
subsets of \(X\), and \[
B \definedas \theset{x\in X \suchthat x\in B_n \text{ for infinitely many } n}.
\]

\begin{enumerate}
\def\labelenumi{\alph{enumi}.}
\item
  Argue that \(B\) is also a \(\mathcal{B} \dash\)measurable subset of
  \(X\).
\item
  Prove that if \(\sum_{n=1}^\infty \mu(B_n) < \infty\) then
  \(\mu(B)= 0\).
\item
  Prove that if \(\sum_{n=1}^\infty \mu(B_n) = \infty\) \textbf{and} the
  sequence of set complements \(\theset{B_n^c}_{n=1}^\infty\) satisfies
  \[
  \mu\left(\bigcap_{n=k}^{K} B_{n}^{c}\right)=\prod_{n=k}^{K}\left(1-\mu\left(B_{n}\right)\right)
  \] for all positive integers \(k\) and \(K\) with \(k < K\), then
  \(mu(B) = 1\).
\end{enumerate}

\begin{quote}
Hint: Use the fact that \(1 - x \leq e^{-x}\) for all \(x\).
\end{quote}

\hypertarget{question-21-1}{%
\subsection{Question 21}\label{question-21-1}}

Let \(\{u_n\}_{n=1}^\infty\) be an orthonormal sequence in a Hilbert
space \(\mathcal{H}\).

\begin{enumerate}
\def\labelenumi{\alph{enumi}.}
\item
  Prove that for every \(x \in \mathcal H\) one has \[
  \displaystyle\sum_{n=1}^{\infty}\left|\left\langle x, u_{n}\right\rangle\right|^{2} \leq\|x\|^{2}
  \]
\item
  Prove that for any sequence \(\{a_n\}_{n=1}^\infty \in \ell^2(\NN)\)
  there exists an element \(x\in\mathcal H\) such that \[
  a_n = \inner{x}{u_n} \text{ for all } n\in \NN
  \] and \[
  \norm{x}^2 = \sum_{n=1}^{\infty}\left|\left\langle x, u_{n}\right\rangle\right|^{2}
  \]
\end{enumerate}

\hypertarget{question-22-1}{%
\subsection{Question 22}\label{question-22-1}}

\begin{enumerate}
\def\labelenumi{\alph{enumi}.}
\item
  Show that if \(f\) is continuous with compact support on \(\RR\), then
  \[
  \lim _{y \rightarrow 0} \int_{\mathbb{R}}|f(x-y)-f(x)| d x=0
  \]
\item
  Let \(f\in L^1(\RR)\) and for each \(h > 0\) let \[
  \mathcal{A}_{h} f(x):=\frac{1}{2 h} \int_{|y| \leq h} f(x-y) d y
  \]
\item
  Prove that \(\left\|\mathcal{A}_{h} f\right\|_{1} \leq\|f\|_{1}\) for
  all \(h > 0\).
\end{enumerate}

\begin{enumerate}
\def\labelenumi{\roman{enumi}.}
\setcounter{enumi}{1}
\tightlist
\item
  Prove that \(\mathcal{A}_h f \to f\) in \(L^1(\RR)\) as \(h \to 0^+\).
\end{enumerate}

\hypertarget{question-23-1}{%
\subsection{Question 23}\label{question-23-1}}

Define \[
E:=\left\{x \in \mathbb{R}:\left|x-\frac{p}{q}\right|<q^{-3} \text { for infinitely many } p, q \in \mathbb{N}\right\}.
\]

Prove that \(m(E) = 0\).

\hypertarget{question-24-1}{%
\subsection{Question 24}\label{question-24-1}}

Let \[
f_{n}(x):=\frac{x}{1+x^{n}}, \quad x \geq 0.
\]

\begin{enumerate}
\def\labelenumi{\alph{enumi}.}
\item
  Show that this sequence converges pointwise and find its limit. Is the
  convergence uniform on \([0, \infty)\)?
\item
  Compute \[
  \lim _{n \rightarrow \infty} \int_{0}^{\infty} f_{n}(x) d x
  \]
\end{enumerate}

\hypertarget{question-25-1}{%
\subsection{Question 25}\label{question-25-1}}

Let \(f\) be a non-negative measurable function on \([0, 1]\).

Show that \[
\lim _{p \rightarrow \infty}\left(\int_{[0,1]} f(x)^{p} d x\right)^{\frac{1}{p}}=\|f\|_{\infty}.
\]

\hypertarget{question-26-1}{%
\subsection{Question 26}\label{question-26-1}}

Let \(f\in L^2([0, 1])\) and suppose \[
\int_{[0,1]} f(x) x^{n} d x=0 \text { for all integers } n \geq 0.
\] Show that \(f = 0\) almost everywhere.

\hypertarget{question-27-1}{%
\subsection{Question 27}\label{question-27-1}}

Suppose that

\begin{itemize}
\tightlist
\item
  \(f_n, f \in L^1\),
\item
  \(f_n \to f\) almost everywhere, and
\item
  \(\int\left|f_{n}\right| \rightarrow \int|f|\).
\end{itemize}

Show that \(\int f_{n} \rightarrow \int f\)

\hypertarget{question-28-1}{%
\subsection{Question 28}\label{question-28-1}}

Let \(f(x) = \frac 1 x\). Show that \(f\) is uniformly continuous on
\((1, \infty)\) but not on \((0,\infty)\).

\hypertarget{question-29-1}{%
\subsection{Question 29}\label{question-29-1}}

Let \(E\subset \RR\) be a Lebesgue measurable set. Show that there is a
Borel set \(B \subset E\) such that \(m(E\setminus B) = 0\).

\hypertarget{question-30-1}{%
\subsection{Question 30}\label{question-30-1}}

Suppose \(f(x)\) and \(xf(x)\) are integrable on \(\RR\). Define \(F\)
by \[
F(t):=\int_{-\infty}^{\infty} f(x) \cos (x t) d x
\] Show that \[
F'(t)=-\int_{-\infty}^{\infty} x f(x) \sin (x t) d x.
\]

\hypertarget{question-31-1}{%
\subsection{Question 31}\label{question-31-1}}

Let \(f\in L^1([0, 1])\). Prove that \[
\lim_{n \to \infty} \int_{0}^{1} f(x) \abs{\sin n x} ~d x= \frac{2}{\pi} \int_{0}^{1} f(x) ~d x
\]

\begin{quote}
Hint: Begin with the case that \(f\) is the characteristic function of
an interval.
\end{quote}

\hypertarget{question-32-1}{%
\subsection{Question 32}\label{question-32-1}}

Let \(f \geq 0\) be a measurable function on \(\RR\). Show that \[
\int_{\mathbb{R}} f=\int_{0}^{\infty} m(\{x: f(x)>t\}) d t
\]

\hypertarget{question-33-1}{%
\subsection{Question 33}\label{question-33-1}}

Compute the following limit and justify your calculations: \[
\lim_{n \rightarrow \infty} \int_{1}^{n} \frac{d x}{\left(1+\frac{x}{n}\right)^{n} \sqrt[n]{x}}
\]

\hypertarget{question-34-1}{%
\subsection{Question 34}\label{question-34-1}}

Let \(K\) be the set of numbers in \([0, 1]\) whose decimal expansions
do not use the digit \(4\).

\begin{quote}
We use the convention that when a decimal number ends with 4 but all
other digits are different from 4, we replace the digit \(4\) with
\(399\cdots\). For example, \(0.8754 = 0.8753999\cdots\).
\end{quote}

Show that \(K\) is a compact, nowhere dense set without isolated points,
and find the Lebesgue measure \(m(K)\).

\hypertarget{question-35-1}{%
\subsection{Question 35}\label{question-35-1}}

\begin{enumerate}
\def\labelenumi{\alph{enumi}.}
\tightlist
\item
  Let \(\mu\) be a measure on a measurable space \((X, \mathcal M)\) and
  \(f\) a positive measurable function.
\end{enumerate}

Define a measure \(\lambda\) by \[
\lambda(E):=\int_{E} f ~d \mu, \quad E \in \mathcal{M}
\]

Show that for \(g\) any positive measurable function, \[
\int_{X} g ~d \lambda=\int_{X} f g ~d \mu
\]

\begin{enumerate}
\def\labelenumi{\alph{enumi}.}
\setcounter{enumi}{1}
\tightlist
\item
  Let \(E \subset \RR\) be a measurable set such that \[
  \int_{E} x^{2} ~d m=0.
  \] Show that \(m(E) = 0\).
\end{enumerate}

\hypertarget{question-36-1}{%
\subsection{Question 36}\label{question-36-1}}

Let \[
f_{n}(x)=a e^{-n a x}-b e^{-n b x} \quad \text{ where } 0 < a < b.
\]

Show that

\begin{enumerate}
\def\labelenumi{\alph{enumi}.}
\tightlist
\item
  \(\sum_{n=1}^{\infty}\left|f_{n}\right| \text { is not in } L^{1}([0, \infty), m)\)
\end{enumerate}

\begin{quote}
Hint: \(f_n(x)\) has a root \(x_n\).
\end{quote}

\begin{enumerate}
\def\labelenumi{\alph{enumi}.}
\setcounter{enumi}{1}
\tightlist
\item
  \[
  \sum_{n=1}^{\infty} f_{n} \text { is in } L^{1}([0, \infty), m) 
  \quad \text { and } \quad 
  \int_{0}^{\infty} \sum_{n=1}^{\infty} f_{n}(x) ~d m=\ln \frac{b}{a}
  \]
\end{enumerate}

\hypertarget{question-37-1}{%
\subsection{Question 37}\label{question-37-1}}

Let \(f(x, y)\) on \([-1, 1]^2\) be defined by \[
f(x, y) = \begin{cases}
\frac{x y}{\left(x^{2}+y^{2}\right)^{2}} & (x, y) \neq (0, 0) \\
0 & (x, y) = (0, 0)
\end{cases}
\] Determine if \(f\) is integrable.

\hypertarget{question-38-1}{%
\subsection{Question 38}\label{question-38-1}}

Let \(f, g \in L^2(\RR)\). Prove that the formula \[
h(x):=\int_{-\infty}^{\infty} f(t) g(x-t) d t
\] defines a uniformly continuous function \(h\) on \(\RR\).

\hypertarget{question-39-1}{%
\subsection{Question 39}\label{question-39-1}}

Show that the space \(C^1([a, b])\) is a Banach space when equipped with
the norm \[
\|f\|:=\sup _{x \in[a, b]}|f(x)|+\sup _{x \in[a, b]}\left|f^{\prime}(x)\right|.
\]

\hypertarget{question-40-1}{%
\subsection{Question 40}\label{question-40-1}}

Let \[
f(x) = s \sum_{n=0}^{\infty} \frac{x^{n}}{n !}.
\]

Describe the intervals on which \(f\) does and does not converge
uniformly.

\hypertarget{question-41-1}{%
\subsection{Question 41}\label{question-41-1}}

Let \(f(x) = x^2\) and \(E \subset [0, \infty) \definedas \RR^+\).

\begin{enumerate}
\def\labelenumi{\arabic{enumi}.}
\item
  Show that \[
  m^*(E) = 0 \iff m^*(f(E)) = 0.
  \]
\item
  Deduce that the map
\end{enumerate}

\begin{align*}
\phi: \mathcal{L}(\RR^+) &\to \mathcal{L}(\RR^+) \\
E &\mapsto f(E)
\end{align*} is a bijection from the class of Lebesgue measurable sets
of \([0, \infty)\) to itself.

\hypertarget{question-42-1}{%
\subsection{Question 42}\label{question-42-1}}

Let \[
S = \spanof_\CC\theset{\chi_{(a, b)} \suchthat a, b \in \RR},
\] the complex linear span of characteristic functions of intervals of
the form \((a, b)\).

Show that for every \(f\in L^1(\RR)\), there exists a sequence of
functions \(\theset{f_n} \subset S\) such that \[
\lim _{n \rightarrow \infty}\left\|f_{n}-f\right\|_{1}=0
\]

\hypertarget{question-43-1}{%
\subsection{Question 43}\label{question-43-1}}

Let \[
f_{n}(x)=n x(1-x)^{n}, \quad n \in \mathbb{N}.
\]

\begin{enumerate}
\def\labelenumi{\arabic{enumi}.}
\item
  Show that \(f_n \to 0\) pointwise but not uniformly on \([0, 1]\).

  \begin{quote}
  Hint: Consider the maximum of \(f_n\).
  \end{quote}
\item
  \[
  \lim _{n \rightarrow \infty} \int_{0}^{1} n(1-x)^{n} \sin x d x=0
  \]
\end{enumerate}

\hypertarget{question-44-1}{%
\subsection{Question 44}\label{question-44-1}}

Let \(\phi\) be a compactly supported smooth function that vanishes
outside of an interval \([-N, N]\) such that
\(\int_{\mathrm{R}} \phi(x) d x=1\).

For \(f\in L^1(\RR)\), define \[
K_{j}(x):=j \phi(j x), \quad \quad f \ast K_{j}(x):=\int_{\mathbb{R}} f(x-y) K_{j}(y) ~d y
\] and prove the following:

\begin{enumerate}
\def\labelenumi{\arabic{enumi}.}
\item
  Each \(f\ast K_j\) is smooth and compactly supported.
\item
  \[
  \lim _{j \rightarrow \infty}\left\|f * K_{j}-f\right\|_{1}=0
  \]
\end{enumerate}

\begin{quote}
Hint: \[
\lim _{y \rightarrow 0} \int_{\mathbb{R}}|f(x-y)-f(x)| d y=0
\]
\end{quote}

\hypertarget{question-45-1}{%
\subsection{Question 45}\label{question-45-1}}

Let \(X\) be a complete metric space and define a norm \[
\|f\|:=\max \{|f(x)|: x \in X\}.
\]

Show that \((C^0(\RR), \norm{\wait} )\) (the space of continuous
functions \(f: X\to \RR\)) is complete.

\hypertarget{question-46-1}{%
\subsection{Question 46}\label{question-46-1}}

For \(n\in \NN\), define \[
e_{n}=\left(1+\frac{1}{n}\right)^{n} 
\quad \text { and } \quad 
E_{n}=\left(1+\frac{1}{n}\right)^{n+1}
\]

Show that \(e_n < E_n\), and prove Bernoulli's inequality: \[
(1+x)^{n} \geq 1+n x \text { for }-1<x<\infty \text { and } n \in \mathbb{N}
\]

Use this to show the following:

\begin{enumerate}
\def\labelenumi{\arabic{enumi}.}
\tightlist
\item
  The sequence \(e_n\) is increasing.
\item
  The sequence \(E_n\) is decreasing.
\item
  \(2 < e_n < E_n < 4\).
\item
  \(\lim _{n \rightarrow \infty} e_{n}=\lim _{n \rightarrow \infty} E_{n}\).
\end{enumerate}

\hypertarget{question-47-1}{%
\subsection{Question 47}\label{question-47-1}}

Let \(0 < \lambda < 1\) and construct a Cantor set \(C_\lambda\) by
successively removing middle intervals of length \(\lambda\).

Prove that \(m(C_\lambda) = 0\).

\hypertarget{question-48-1}{%
\subsection{Question 48}\label{question-48-1}}

Let \(f\) be Lebesgue measurable on \(\RR\) and \(E \subset \RR\) be
measurable such that \[
0<A=\int_{E} f(x) d x<\infty.
\]

Show that for every \(0 < t < 1\), there exists a measurable set
\(E_t \subset E\) such that \[
\int_{E_{t}} f(x) d x=t A.
\]

\hypertarget{question-49-1}{%
\subsection{Question 49}\label{question-49-1}}

Let \(E \subset \RR\) be measurable with \(m(E) < \infty\). Define \[
f(x)=m(E \cap(E+x)).
\]

Show that

\begin{enumerate}
\def\labelenumi{\arabic{enumi}.}
\tightlist
\item
  \(f\in L^1(\RR)\).
\item
  \(f\) is uniformly continuous.
\item
  \(\lim _{|x| \rightarrow \infty} f(x)=0\)
\end{enumerate}

\begin{quote}
Hint: \[
\chi_{E \cap(E+x)}(y)=\chi_{E}(y) \chi_{E}(y-x)
\]
\end{quote}

\hypertarget{question-50-1}{%
\subsection{Question 50}\label{question-50-1}}

Let \((X, \mathcal M, \mu)\) be a measure space. For \(f\in L^1(\mu)\)
and \(\lambda > 0\), define \[
\phi(\lambda)=\mu(\{x \in X | f(x)>\lambda\}) 
\quad \text { and } \quad 
\psi(\lambda)=\mu(\{x \in X | f(x)<-\lambda\})
\]

Show that \(\phi, \psi\) are Borel measurable and \[
\int_{X}|f| ~d \mu=\int_{0}^{\infty}[\phi(\lambda)+\psi(\lambda)] ~d \lambda
\]

\hypertarget{question-51-1}{%
\subsection{Question 51}\label{question-51-1}}

Without using the Riesz Representation Theorem, compute \[
\sup \left\{\left|\int_{0}^{1} f(x) e^{x} d x\right| \suchthat f \in L^{2}([0,1], m),~~ \|f\|_{2} \leq 1\right\}
\]

\hypertarget{question-52-1}{%
\subsection{Question 52}\label{question-52-1}}

Define \[
f(x) = \sum_{n=1}^{\infty} \frac{1}{n^{x}}.
\]

Show that \(f\) converges to a differentiable function on
\((1, \infty)\) and that \[
f'(x)  =\sum_{n=1}^{\infty}\left(\frac{1}{n^{x}}\right)^{\prime}.
\]

\begin{quote}
Hint: \[
\left(\frac{1}{n^{x}}\right)^{\prime}=-\frac{1}{n^{x}} \ln n
\]
\end{quote}

\hypertarget{question-53-1}{%
\subsection{Question 53}\label{question-53-1}}

Let \(f, g: [a, b] \to \RR\) be measurable with \[
\int_{a}^{b} f(x) ~d x=\int_{a}^{b} g(x) ~d x.
\]

Show that either

\begin{enumerate}
\def\labelenumi{\arabic{enumi}.}
\tightlist
\item
  \(f(x) = g(x)\) almost everywhere, or
\item
  There exists a measurable set \(E \subset [a, b]\) such that \[
  \int_{E} f(x) ~d x>\int_{E} g(x) ~d x
  \]
\end{enumerate}

\hypertarget{question-54-1}{%
\subsection{Question 54}\label{question-54-1}}

Let \(f\in L^1(\RR)\). Show that \[
\lim _{x \rightarrow 0} \int_{\mathbb{R}}|f(y-x)-f(y)| d y=0
\]

\hypertarget{question-55-1}{%
\subsection{Question 55}\label{question-55-1}}

Let \((X, \mathcal M, \mu)\) be a measure space and suppose
\(\theset{E_n} \subset \mathcal M\) satisfies \[
\lim _{n \rightarrow \infty} \mu\left(X \backslash E_{n}\right)=0.
\]

Define \[
G \definedas \theset{x\in X \suchthat x\in E_n \text{ for only finitely many  } n}.
\]

Show that \(G \in \mathcal M\) and \(\mu(G) = 0\).

\hypertarget{question-56-1}{%
\subsection{Question 56}\label{question-56-1}}

Let \(\phi\in L^\infty(\RR)\). Show that the following limit exists and
satisfies the equality \[
\lim _{n \rightarrow \infty}\left(\int_{\mathbb{R}} \frac{|\phi(x)|^{n}}{1+x^{2}} d x\right)^{\frac{1}{n}} = \norm{\phi}_\infty.
\]

\hypertarget{question-57-1}{%
\subsection{Question 57}\label{question-57-1}}

Let \(f, g \in L^2(\RR)\). Show that \[
\lim _{n \rightarrow \infty} \int_{\mathbb{R}} f(x) g(x+n) d x=0
\]

\hypertarget{question-58-1}{%
\subsection{Question 58}\label{question-58-1}}

Let \((X, d)\) and \((Y, \rho)\) be metric spaces, \(f: X\to Y\), and
\(x_0 \in X\).

Prove that the following statements are equivalent:

\begin{enumerate}
\def\labelenumi{\arabic{enumi}.}
\tightlist
\item
  For every \(\varepsilon > 0 \quad \exists \delta > 0\) such that
  \(\rho( f(x), f(x_0) ) < \varepsilon\) whenever
  \(d(x, x_0) < \delta\).
\item
  The sequence \(\theset{f(x_n)}_{n=1}^\infty \to f(x_0)\) for every
  sequence \(\theset{x_n} \to x_0\) in \(X\).
\end{enumerate}

\hypertarget{question-59-1}{%
\subsection{Question 59}\label{question-59-1}}

Let \(f: \RR \to \CC\) be continuous with period 1. Prove that \[
\lim _{N \rightarrow \infty} \frac{1}{N} \sum_{n=1}^{N} f(n \alpha)=\int_{0}^{1} f(t) d t \quad \forall \alpha \in \RR\setminus\QQ.
\]

\begin{quote}
Hint: show this first for the functions \(f(t) = e^{2\pi i k t}\) for
\(k\in \ZZ\).
\end{quote}

\hypertarget{question-60-1}{%
\subsection{Question 60}\label{question-60-1}}

Let \(\mu\) be a finite Borel measure on \(\RR\) and \(E \subset \RR\)
Borel. Prove that the following statements are equivalent:

\begin{enumerate}
\def\labelenumi{\arabic{enumi}.}
\tightlist
\item
  \(\forall \varepsilon > 0\) there exists \(G\) open and \(F\) closed
  such that \[
  F \subseteq E \subseteq G \quad \text{and} \quad \mu(G\setminus F) < \varepsilon.
  \]
\item
  There exists a \(V \in G_\delta\) and \(H \in F_\sigma\) such that \[
  H \subseteq E \subseteq V \quad \text{and}\quad \mu(V\setminus H) = 0
  \]
\end{enumerate}

\hypertarget{question-61-1}{%
\subsection{Question 61}\label{question-61-1}}

Define \[
f(x, y):=\left\{\begin{array}{ll}{\frac{x^{1 / 3}}{(1+x y)^{3 / 2}}} & {\text { if } 0 \leq x \leq y} \\ {0} & {\text { otherwise }}\end{array}\right.
\]

Carefully show that \(f \in L^1(\RR^2)\).

\hypertarget{question-62-1}{%
\subsection{Question 62}\label{question-62-1}}

Let \(\mathcal H\) be a Hilbert space.

\begin{enumerate}
\def\labelenumi{\arabic{enumi}.}
\tightlist
\item
  Let \(x\in \mathcal H\) and \(\theset{u_n}_{n=1}^N\) be an orthonormal
  set. Prove that the best approximation to \(x\) in \(\mathcal H\) by
  an element in \(\spanof_\CC\theset{u_n}\) is given by \[
    \hat x \definedas \sum_{n=1}^N \inner{x}{u_n}u_n.
    \]
\item
  Conclude that finite dimensional subspaces of \(\mathcal H\) are
  always closed.
\end{enumerate}

\hypertarget{question-63-1}{%
\subsection{Question 63}\label{question-63-1}}

Let \(f \in L^1(\RR)\) and \(g\) be a bounded measurable function on
\(\RR\).

\begin{enumerate}
\def\labelenumi{\arabic{enumi}.}
\tightlist
\item
  Show that the convolution \(f\ast g\) is well-defined, bounded, and
  uniformly continuous on \(\RR\).
\item
  Prove that one further assumes that \(g \in C^1(\RR)\) with bounded
  derivative, then \(f\ast g \in C^1(\RR)\) and \[
  \frac{d}{d x}(f * g)=f *\left(\frac{d}{d x} g\right)
  \]
\end{enumerate}

\hypertarget{question-64-1}{%
\subsection{Question 64}\label{question-64-1}}

Define \[
f(x)=c_{0}+c_{1} x^{1}+c_{2} x^{2}+\ldots+c_{n} x^{n} \text { with } n \text { even and } c_{n}>0.
\]

Show that there is a number \(x_m\) such that \(f(x_m) \leq f(x)\) for
all \(x\in \RR\).

\hypertarget{question-65-1}{%
\subsection{Question 65}\label{question-65-1}}

Let \(f: \RR \to \RR\) be Lebesgue measurable.

\begin{enumerate}
\def\labelenumi{\arabic{enumi}.}
\tightlist
\item
  Show that there is a sequence of simple functions \(s_n(x)\) such that
  \(s_n(x) \to f(x)\) for all \(x\in \RR\).
\item
  Show that there is a Borel measurable function \(g\) such that
  \(g = f\) almost everywhere.
\end{enumerate}

\hypertarget{question-66-1}{%
\subsection{Question 66}\label{question-66-1}}

Compute the following limit: \[
\lim _{n \rightarrow \infty} \int_{1}^{n} \frac{n e^{-x}}{1+n x^{2}} ~\sin \left(\frac x n\right) ~d x
\]

\hypertarget{question-67-1}{%
\subsection{Question 67}\label{question-67-1}}

Let \(f: [1, \infty) \to \RR\) such that \(f(1) = 1\) and \[
f^{\prime}(x)= \frac{1} {x^{2}+f(x)^{2}}
\]

Show that the following limit exists and satisfies the equality \[
\lim _{x \rightarrow \infty} f(x) \leq 1 + \frac \pi 4
\]

\hypertarget{question-68-1}{%
\subsection{Question 68}\label{question-68-1}}

Let \(f, g \in L^1(\RR)\) be Borel measurable.

\begin{enumerate}
\def\labelenumi{\arabic{enumi}.}
\tightlist
\item
  Show that
\end{enumerate}

\begin{itemize}
\tightlist
\item
  The function \[F(x, y) \definedas f(x-y) g(y)\] is Borel measurable on
  \(\RR^2\), and
\item
  For almost every \(y\in \RR\), \[F_y(x) \definedas f(x-y)g(y)\] is
  integrable with respect to \(y\).
\end{itemize}

\begin{enumerate}
\def\labelenumi{\arabic{enumi}.}
\setcounter{enumi}{1}
\tightlist
\item
  Show that \(f\ast g \in L^1(\RR)\) and \[
  \|f * g\|_{1} \leq\|f\|_{1}\|g\|_{1}
  \]
\end{enumerate}

\hypertarget{question-69-1}{%
\subsection{Question 69}\label{question-69-1}}

Let \(f: [0, 1] \to \RR\) be continuous. Show that \[
\sup \left\{\|f g\|_{1} \suchthat g \in L^{1}[0,1],~~ \|g\|_{1} \leq 1\right\}=\|f\|_{\infty}
\]

\hypertarget{question-70-1}{%
\subsection{Question 70}\label{question-70-1}}

\begin{enumerate}
\def\labelenumi{\arabic{enumi}.}
\item
  Give an example of a continuous \(f\in L^1(\RR)\) such that
  \(f(x) \not\to 0\) as\(\abs x \to \infty\).
\item
  Show that if \(f\) is \emph{uniformly} continuous, then \[
  \lim_{\abs{x} \to \infty} f(x) = 0.
  \]
\end{enumerate}

\hypertarget{question-71-1}{%
\subsection{Question 71}\label{question-71-1}}

Let \(\theset{a_n}\) be a sequence of real numbers such that \[
\theset{b_n} \in \ell^2(\NN) \implies \sum a_n b_n < \infty.
\] Show that \(\sum a_n^2 < \infty\).

\begin{quote}
Note: Assume \(a_n, b_n\) are all non-negative.
\end{quote}

\hypertarget{question-72-1}{%
\subsection{Question 72}\label{question-72-1}}

Let \(f: \RR \to \RR\) and suppose \[
\forall x\in \RR,\quad f(x) \geq \limsup _{y \rightarrow x} f(y)
\] Prove that \(f\) is Borel measurable.

\hypertarget{question-73-1}{%
\subsection{Question 73}\label{question-73-1}}

Let \((X, \mathcal M, \mu)\) be a measure space and suppose \(f\) is a
measurable function on \(X\). Show that \[
\lim _{n \rightarrow \infty} \int_{X} f^{n} ~d \mu =
\begin{cases}
\infty & or \\
\mu(f\inv(1)),
\end{cases}
\] and characterize the collection of functions of each type.

\hypertarget{question-74-1}{%
\subsection{Question 74}\label{question-74-1}}

Let \(f, g \in L^1([0, 1])\) and for all \(x\in [0, 1]\) define \[
F(x):=\int_{0}^{x} f(y) d y \quad \text { and } \quad G(x):=\int_{0}^{x} g(y) d y.
\]

Prove that \[
\int_{0}^{1} F(x) g(x) d x=F(1) G(1)-\int_{0}^{1} f(x) G(x) d x
\]

\hypertarget{question-75-1}{%
\subsection{Question 75}\label{question-75-1}}

Let \(\theset{f_n}\) be a sequence of continuous functions such that
\(\sum f_n\) converges uniformly.

Prove that \(\sum f_n\) is also continuous.

\hypertarget{question-76-1}{%
\subsection{Question 76}\label{question-76-1}}

Let \(I\) be an index set and \(\alpha: I \to (0, \infty)\).

\begin{enumerate}
\def\labelenumi{\arabic{enumi}.}
\item
  Show that \[
  \sum_{i \in I} a(i):=\sup _{\substack{ J \subset I \\ J \text { finite }}} \sum_{i \in J} a(i)<\infty \implies I \text{ is countable.}
  \]
\item
  Suppose \(I = \QQ\) and \(\sum_{q \in \mathbb{Q}} a(q)<\infty\).
  Define \[
  f(x):=\sum_{\substack{q \in \mathbb{Q}\\ q \leq x}} a(q).
  \] Show that \(f\) is continuous at \(x \iff x\not\in \QQ\).
\end{enumerate}

\hypertarget{question-77-1}{%
\subsection{Question 77}\label{question-77-1}}

Let \(f\in L^1(\RR)\). Show that \[
\forall\varepsilon > 0 ~~\exists \delta > 0 \text{ such that } m(E) < \delta \implies \int_{E}|f(x)| d x<\varepsilon
\]

\hypertarget{question-78-1}{%
\subsection{Question 78}\label{question-78-1}}

Let \(g\in L^\infty([0, 1])\) Prove that \[
\int_{[0,1]} f(x) g(x) d x=0 \quad\text{for all continuous } f:[0, 1] \to \RR \implies g(x) = 0 \text{ almost everywhere. }
\]

\hypertarget{question-79-1}{%
\subsection{Question 79}\label{question-79-1}}

\begin{enumerate}
\def\labelenumi{\arabic{enumi}.}
\item
  Let \(f \in C_c^0(\RR^n)\), and show \[
  \lim _{t \rightarrow 0} \int_{\mathbb{R}^{n}}|f(x+t)-f(x)| d x=0.
  \]
\item
  Extend the above result to \(f\in L^1(\RR^n)\) and show that \[
  f\in L^1(\RR^n),~ g\in L^\infty(\RR^n) \implies f \ast g \text{ is bounded and uniformly continuous. }
  \]
\end{enumerate}

\hypertarget{question-80-1}{%
\subsection{Question 80}\label{question-80-1}}

Let \(1 \leq p,q \leq \infty\) be conjugate exponents, and show that \[
f \in L^p(\RR^n) \implies \|f\|_{p}=\sup _{\|g\|_{q}=1}\left|\int f(x) g(x) d x\right|
\]

\hypertarget{question-81-1}{%
\subsection{Question 81}\label{question-81-1}}

Describe the process that extends a measure on an algebra
\(\mathcal{A}\) of subsets of \(X\), to a complete measure defined on a
\(\sigma\)-algebra \(\mathcal{B}\) containing \(\mathcal{A}\). State the
corresponding definitions and results (without proofs).

\hypertarget{question-82-1}{%
\subsection{Question 82}\label{question-82-1}}

State and prove Fatou's Lemma on a general measurable space.

\hypertarget{question-83-1}{%
\subsection{Question 83}\label{question-83-1}}

\begin{enumerate}
\def\labelenumi{\arabic{enumi}.}
\item
  State the Dominated Convergence Theorem for Lebesgue integrals.
\item
  Let \(\{f_n\}\) be a sequence of measurable functions on a Lebesgue
  measurable set \(E\) which converges \emph{in measure} to a function
  \(f\) on \(E\). Suppose that for every \(n\), \(|f_n| \leq g\) with
  \(g\) integrable on \(E\). Using the above theorem show that
  \begin{align*}
      \int_E |f_n-f| \longrightarrow 0 \, .
  \end{align*}
\end{enumerate}

\hypertarget{question-84-1}{%
\subsection{Question 84}\label{question-84-1}}

Let \(f\in L^1([0,1])\). Show that

\begin{enumerate}
\def\labelenumi{\arabic{enumi}.}
\item
  The limit \(\lim_{p\to 0^+} \| f \|_p\) exists.
\item
  If \(m \{x : f(x) = 0\} > 0\), then the above limit is zero.
\end{enumerate}

\hypertarget{question-85-1}{%
\subsection{Question 85}\label{question-85-1}}

Let \(f\) be a continuous function on \([0,1]\). Show that the following
statements are equivalent.

\begin{enumerate}
\def\labelenumi{\arabic{enumi}.}
\item
  \(f\) is absolutely continuous.
\item
  For any \(\epsilon > 0\) there exists \(\delta > 0\) such that
  \(m(f(E)) < \epsilon\) for any set \(E\subseteq [0,1]\) with
  \(m(E) < \delta\).
\item
  \(m(f(E)) = 0\) for any set \(E \subseteq [0,1]\) with \(m(E)=0\).
\end{enumerate}

\hypertarget{complex-analysis-125-questions}{%
\section{Complex Analysis (125
Questions)}\label{complex-analysis-125-questions}}

\hypertarget{question-1-2}{%
\subsection{Question 1}\label{question-1-2}}

Find the number of zeroes, counting multiplicities, of the polynomial

\(f(z) = 2z^5 - 6z^2 - z + 1 = 0\)

in the annulus \(1 \leq |z| \leq 2\).

\hypertarget{question-2-2}{%
\subsection{Question 2}\label{question-2-2}}

Find an analytic isomorphism from the open region between \(|z| = 1\)
and \(|z -\frac 1 2| =\frac 1 2\) to the upper half plane \(\Im z > 0\).
(You may leave your result as a composition of functions).

\hypertarget{question-3-2}{%
\subsection{Question 3}\label{question-3-2}}

Use Green theorem or otherwise to prove the Cauchy theorem.

\hypertarget{question-4-2}{%
\subsection{Question 4}\label{question-4-2}}

State and prove the divergence theorem on any rectangle in
\(\mathbb{R}^2\).

\hypertarget{question-5-2}{%
\subsection{Question 5}\label{question-5-2}}

Find an analytic isomorphism from the open region between \(x = 1\) and
\(x = 3\) to the upper half unit disk \(\{|z| < 1,\Im z > 0\}\). (You
may leave your result as a composition of functions)

\hypertarget{question-6-2}{%
\subsection{Question 6}\label{question-6-2}}

Use Cauchy's theorem to prove the argument principle.

\hypertarget{question-7-2}{%
\subsection{Question 7}\label{question-7-2}}

Evaluate the following by the method of residues:
\(\int_0^{\pi /2} \frac{1}{3+\sin^2x}dx\)

\hypertarget{question-8-2}{%
\subsection{Question 8}\label{question-8-2}}

Evaluate the improper integral

\(\int_0^\infty \frac{x^2~dx}{(x^2+1)(x^2+4)}\)

\hypertarget{question-9-2}{%
\subsection{Question 9}\label{question-9-2}}

\begin{enumerate}
\def\labelenumi{(\arabic{enumi})}
\item
  Assume \(\displaystyle f(z) = \sum_{n=0}^\infty c_n z^n\) converges in
  \(|z| < R\). Show that for \(r <R\),
  \[\frac{1}{2 \pi} \int_0^{2 \pi} |f(r e^{i \theta})|^2 d \theta =
  \sum_{n=0}^\infty |c_n|^2 r^{2n} \; .\]
\item
  Deduce Liouville's theorem from (1).
\end{enumerate}

\hypertarget{question-10-2}{%
\subsection{Question 10}\label{question-10-2}}

Let \(f\) be a continuous function in the region
\[D=\{z \suchthat  \abs{z}>R, 0\leq \arg z\leq \theta\}\quad\text{where}\quad
1\leq \theta \leq 2\pi.\] If there exists \(k\) such that
\(\displaystyle{\lim_{z\to\infty} zf(z)=k}\) for \(z\) in the region
\(D\). Show that \[\lim_{R'\to\infty} \int_{L} f(z) dz=i\theta k,\]
where \(L\) is the part of the circle \(|z|=R'\) which lies in the
region \(D\).

\hypertarget{question-11-2}{%
\subsection{Question 11}\label{question-11-2}}

Suppose that \(f\) is an analytic function in the region \(D\) which
contains the point \(a\). Let
\[F(z)= z-a-qf(z),\quad \text{where}~ q \ \text{is a complex
parameter}.\]

\begin{enumerate}
\def\labelenumi{(\arabic{enumi})}
\item
  Let \(K\subset D\) be a circle with the center at point \(a\) and also
  we assume that \(f(z)\not =0\) for \(z\in K\). Prove that the function
  \(F\) has one and only one zero \(z=w\) on the closed disc \(\bar{K}\)
  whose boundary is the circle \(K\) if
  \(\displaystyle{ |q|<\min_{z\in K} \frac{|z-a|}{|f(z)|}.}\)\\
\item
  Let \(G(z)\) be an analytic function on the disk \(\bar{K}\). Apply
  the residue theorem to prove that
  \(\displaystyle{ \frac{G(w)}{F'(w)}=\frac{1}{2\pi i}\int_K \frac{G(z)}{F(z)} dz,}\)
  where \(w\) is the zero from (1).\\
\item
  If \(z\in K\), prove that the function
  \(\displaystyle{\frac{1}{F(z)}}\) can be represented as a convergent
  series with respect to \(q\):
  \(\displaystyle{ \frac{1}{F(z)}=\sum_{n=0}^{\infty} \frac{(qf(z))^n}{(z-a)^{n+1}}.}\)
\end{enumerate}

\hypertarget{question-12-2}{%
\subsection{Question 12}\label{question-12-2}}

Evaluate \[\displaystyle{ \int_{0}^{\infty}\frac{x\sin x}{x^2+a^2} \,
dx }.\]

\hypertarget{question-13-2}{%
\subsection{Question 13}\label{question-13-2}}

Let \(f=u+iv\) be differentiable (i.e.~\(f'(z)\) exists) with continuous
partial derivatives at a point \(z=re^{i\theta}\), \(r\not= 0\). Show
that
\[\frac{\partial u}{\partial r}=\frac{1}{r}\frac{\partial v}{\partial \theta},\quad
\frac{\partial v}{\partial r}=-\frac{1}{r}\frac{\partial u}{\partial \theta}.\]

\hypertarget{question-14-2}{%
\subsection{Question 14}\label{question-14-2}}

Show that
\(\displaystyle \int_0^\infty \frac{x^{a-1}}{1+x^n} dx=\frac{\pi}{n\sin \frac{a\pi}{n}}\)
using complex analysis, \(0< a < n\). Here \(n\) is a positive integer.

\hypertarget{question-15-2}{%
\subsection{Question 15}\label{question-15-2}}

For \(s>0\), the \textbf{gamma function} is defined by
\(\displaystyle{\Gamma(s)=\int_0^{\infty} e^{-t}t^{s-1} dt}\).

\begin{enumerate}
\def\labelenumi{\arabic{enumi}.}
\item
  Show that the gamma function is analytic in the half-plane
  \(\Re (s)>0\), and is still given there by the integral formula above.
\item
  Apply the formula in the previous question to show that
  \[\Gamma(s)\Gamma(1-s)=\frac{\pi}{\sin \pi s}.\]
\end{enumerate}

\begin{quote}
Hint: You may need
\(\displaystyle{\Gamma(1-s)=t \int_0^{\infty}e^{-vt}(vt)^{-s} dv}\) for
\(t>0\).
\end{quote}

\hypertarget{question-16-2}{%
\subsection{Question 16}\label{question-16-2}}

Apply Rouché's Theorem to prove the Fundamental Theorem of Algebra: If
\[P_n(z) = a_0 + a_1z + \cdots + a_{n-1}z^{n-1} + a_nz^n\quad  (a_n \neq 0)\]
is a polynomial of degree n, then it has n zeros in \(\mathbb C\).

\hypertarget{question-17-2}{%
\subsection{Question 17}\label{question-17-2}}

Suppose \(f\) is entire and there exist \(A, R >0\) and natural number
\(N\) such that \[|f(z)| \geq A |z|^N\ \text{for}\ |z| \geq R.\] Show
that

\begin{enumerate}
\def\labelenumi{(\roman{enumi})}
\item
  \(f\) is a polynomial and
\item
  the degree of \(f\) is at least \(N\).
\end{enumerate}

\hypertarget{question-18-2}{%
\subsection{Question 18}\label{question-18-2}}

Let \(f: {\mathbb C} \rightarrow {\mathbb C}\) be an injective analytic
(also called \emph{univalent}) function. Show that there exist complex
numbers \(a \neq 0\) and \(b\) such that \(f(z) = az + b\).

\hypertarget{question-19-2}{%
\subsection{Question 19}\label{question-19-2}}

Let \(g\) be analytic for \(|z|\leq 1\) and \(|g(z)| < 1\) for
\(|z| = 1\).

\begin{enumerate}
\def\labelenumi{\arabic{enumi}.}
\item
  Show that \(g\) has a unique fixed point in \(|z| < 1\).
\item
  What happens if we replace \(|g(z)| < 1\) with \(|g(z)|\leq 1\) for
  \(|z|=1\)? Give an example if (a) is not true or give an proof if (a)
  is still true.
\item
  What happens if we simply assume that \(f\) is analytic for
  \(|z| < 1\) and \(|f(z)| < 1\) for \(|z| < 1\)? Suppose that
  \(f(z) \not\equiv z\). Can f have more than one fixed point in
  \(|z| < 1\)?
\end{enumerate}

\begin{quote}
Hint: The map
\(\displaystyle{\psi_{\alpha}(z)=\frac{\alpha-z}{1-\bar{\alpha}z}}\) may
be useful.
\end{quote}

\hypertarget{question-20-2}{%
\subsection{Question 20}\label{question-20-2}}

Find a conformal map from \(D = \{z :\  |z| < 1,\ |z - 1/2| > 1/2\}\) to
the unit disk \(\Delta=\{z: \ |z|<1\}\).

\hypertarget{question-21-2}{%
\subsection{Question 21}\label{question-21-2}}

Let \(f(z)\) be entire and assume values of \(f(z)\) lie outside a
\emph{bounded} open set \(\Omega\). Show without using Picard's theorems
that \(f(z)\) is a constant.

\hypertarget{question-22-2}{%
\subsection{Question 22}\label{question-22-2}}

\begin{enumerate}
\def\labelenumi{(\arabic{enumi})}
\item
  Assume \(\displaystyle f(z) = \sum_{n=0}^\infty c_n z^n\) converges in
  \(|z| < R\). Show that for \(r <R\),
  \[\frac{1}{2 \pi} \int_0^{2 \pi} |f(r e^{i \theta})|^2 d \theta
  = \sum_{n=0}^\infty |c_n|^2 r^{2n} \; .\]
\item
  Deduce Liouville's theorem from (1).
\end{enumerate}

\hypertarget{question-23-2}{%
\subsection{Question 23}\label{question-23-2}}

Let \(f(z)\) be entire and assume that \(f(z) \leq M |z|^2\) outside
some disk for some constant \(M\). Show that \(f(z)\) is a polynomial in
\(z\) of degree \(\leq 2\).

\hypertarget{question-24-2}{%
\subsection{Question 24}\label{question-24-2}}

Let \(a_n(z)\) be an analytic sequence in a domain \(D\) such that
\(\displaystyle \sum_{n=0}^\infty |a_n(z)|\) converges uniformly on
bounded and closed sub-regions of \(D\). Show that
\(\displaystyle \sum_{n=0}^\infty |a'_n(z)|\) converges uniformly on
bounded and closed sub-regions of \(D\).

\hypertarget{question-25-2}{%
\subsection{Question 25}\label{question-25-2}}

Let \(f(z)\) be analytic in an open set \(\Omega\) except possibly at a
point \(z_0\) inside \(\Omega\). Show that if \(f(z)\) is bounded in
near \(z_0\), then \(\displaystyle \int_\Delta f(z) dz = 0\) for all
triangles \(\Delta\) in \(\Omega\).

\hypertarget{question-26-2}{%
\subsection{Question 26}\label{question-26-2}}

Assume \(f\) is continuous in the region:
\(0< |z-a| \leq R, \; 0 \leq \arg(z-a) \leq \beta_0\)
(\(0 < \beta_0 \leq 2 \pi\)) and the limit
\(\displaystyle \lim_{z \rightarrow a} (z-a) f(z) = A\) exists. Show
that
\[\lim_{r \rightarrow 0} \int_{\gamma_r} f(z) dz  = i A \beta_0 \; , \; \;\]
where
\[\gamma_r : = \{ z \; | \; z = a + r e^{it}, \; 0 \leq  t \leq \beta_0 \}.\]

\hypertarget{question-27-2}{%
\subsection{Question 27}\label{question-27-2}}

Show that \(f(z) = z^2\) is uniformly continuous in any open disk
\(|z| < R\), where \(R>0\) is fixed, but it is not uniformly continuous
on \(\mathbb C\).

\hypertarget{question-28-2}{%
\subsection{Question 28}\label{question-28-2}}

\begin{enumerate}
\def\labelenumi{(\arabic{enumi})}
\item
  Show that the function \(u=u(x,y)\) given by
  \[u(x,y)=\frac{e^{ny}-e^{-ny}}{2n^2}\sin nx\quad \text{for}\ n\in {\mathbf N}\]
  is the solution on \(D=\{(x,y)\ | x^2+y^2<1\}\) of the Cauchy problem
  for the Laplace equation
  \[\frac{\partial ^2u}{\partial x^2}+\frac{\partial ^2u}{\partial y^2}=0,\quad
  u(x,0)=0,\quad \frac{\partial u}{\partial y}(x,0)=\frac{\sin nx}{n}.\]
\item
  Show that there exist points \((x,y)\in D\) such that
  \(\displaystyle{\limsup_{n\to\infty} |u(x,y)|=\infty}\).
\end{enumerate}

\hypertarget{question-29-2}{%
\subsection{Question 29}\label{question-29-2}}

\begin{enumerate}
\def\labelenumi{(\arabic{enumi})}
\item
  Assume \(\displaystyle f(z) = \sum_{n=0}^\infty c_n z^n\) converges in
  \(|z| < R\). Show that for \(r <R\),
  \[\frac{1}{2 \pi} \int_0^{2 \pi} |f(r e^{i \theta})|^2 d \theta =
  \sum_{n=0}^\infty |c_n|^2 r^{2n} \; .\]
\item
  Deduce Liouville's theorem from (1).
\end{enumerate}

\hypertarget{question-30-2}{%
\subsection{Question 30}\label{question-30-2}}

Let \(f\) be a continuous function in the region
\[D=\{z\ |  |z|>R, 0\leq \arg Z\leq \theta\}\quad\text{where}\quad
0\leq \theta \leq 2\pi.\] If there exists \(k\) such that
\(\displaystyle{\lim_{z\to\infty} zf(z)=k}\) for \(z\) in the region
\(D\). Show that \[\lim_{R'\to\infty} \int_{L} f(z) dz=i\theta k,\]
where \(L\) is the part of the circle \(|z|=R'\) which lies in the
region \(D\).

\hypertarget{question-31-2}{%
\subsection{Question 31}\label{question-31-2}}

Evaluate
\(\displaystyle{ \int_{0}^{\infty}\frac{x\sin x}{x^2+a^2} \,dx }\).

\hypertarget{question-32-2}{%
\subsection{Question 32}\label{question-32-2}}

Let \(f=u+iv\) be differentiable (i.e.~\(f'(z)\) exists) with continuous
partial derivatives at a point \(z=re^{i\theta}\), \(r\not= 0\). Show
that
\[\frac{\partial u}{\partial r}=\frac{1}{r}\frac{\partial v}{\partial \theta},\quad
\frac{\partial v}{\partial r}=-\frac{1}{r}\frac{\partial u}{\partial \theta}.\]

\hypertarget{question-33-2}{%
\subsection{Question 33}\label{question-33-2}}

Show that
\(\displaystyle \int_0^\infty \frac{x^{a-1}}{1+x^n} dx=\frac{\pi}{n\sin \frac{a\pi}{n}}\)
using complex analysis, \(0< a < n\). Here \(n\) is a positive integer.

\hypertarget{question-34-2}{%
\subsection{Question 34}\label{question-34-2}}

For \(s>0\), the \textbf{gamma function} is defined by
\(\displaystyle{\Gamma(s)=\int_0^{\infty} e^{-t}t^{s-1} dt}\).

\begin{enumerate}
\def\labelenumi{\arabic{enumi}.}
\item
  Show that the gamma function is analytic in the half-plane
  \(\Re (s)>0\), and is still given there by the integral formula above.
\item
  Apply the formula in the previous question to show that
  \[\Gamma(s)\Gamma(1-s)=\frac{\pi}{\sin \pi s}.\]
\end{enumerate}

\begin{quote}
Hint: You may need
\(\displaystyle{\Gamma(1-s)=t \int_0^{\infty}e^{-vt}(vt)^{-s} dv}\) for
\(t>0\).
\end{quote}

\hypertarget{question-35-2}{%
\subsection{Question 35}\label{question-35-2}}

Suppose \(f\) is entire and there exist \(A, R >0\) and natural number
\(N\) such that \[|f(z)| \geq A |z|^N\ \text{for}\ |z| \geq R.\] Show
that

\begin{enumerate}
\def\labelenumi{(\roman{enumi})}
\item
  \(f\) is a polynomial and
\item
  the degree of \(f\) is at least \(N\).
\end{enumerate}

\hypertarget{question-36-2}{%
\subsection{Question 36}\label{question-36-2}}

Let \(f: {\mathbb C} \rightarrow {\mathbb C}\) be an injective analytic
(also called univalent) function. Show that there exist complex numbers
\(a \neq 0\) and \(b\) such that \(f(z) = az + b\).

\hypertarget{question-37-2}{%
\subsection{Question 37}\label{question-37-2}}

Let \(g\) be analytic for \(|z|\leq 1\) and \(|g(z)| < 1\) for
\(|z| = 1\).

\begin{itemize}
\item
  Show that \(g\) has a unique fixed point in \(|z| < 1\).
\item
  What happens if we replace \(|g(z)| < 1\) with \(|g(z)|\leq 1\) for
  \(|z|=1\)? Give an example if (a) is not true or give an proof if (a)
  is still true.
\item
  What happens if we simply assume that \(f\) is analytic for
  \(|z| < 1\) and \(|f(z)| < 1\) for \(|z| < 1\)? Suppose that
  \(f(z) \not\equiv z\). Can f have more than one fixed point in
  \(|z| < 1\)?
\end{itemize}

\begin{quote}
Hint: The map
\(\displaystyle{\psi_{\alpha}(z)=\frac{\alpha-z}{1-\bar{\alpha}z}}\) may
be useful.
\end{quote}

\hypertarget{question-38-2}{%
\subsection{Question 38}\label{question-38-2}}

Find a conformal map from \(D = \{z :\  |z| < 1,\ |z - 1/2| > 1/2\}\) to
the unit disk \(\Delta=\{z: \ |z|<1\}\).

\hypertarget{question-39-2}{%
\subsection{Question 39}\label{question-39-2}}

Let \(f(z)\) be entire and assume values of \(f(z)\) lie outside a
\emph{bounded} open set \(\Omega\). Show without using Picard's theorems
that \(f(z)\) is a constant.

\hypertarget{question-40-2}{%
\subsection{Question 40}\label{question-40-2}}

\begin{enumerate}
\def\labelenumi{(\arabic{enumi})}
\item
  Assume \(\displaystyle f(z) = \sum_{n=0}^\infty c_n z^n\) converges in
  \(|z| < R\). Show that for \(r <R\),
  \[\frac{1}{2 \pi} \int_0^{2 \pi} |f(r e^{i \theta})|^2 d \theta
  = \sum_{n=0}^\infty |c_n|^2 r^{2n} \; .\]
\item
  Deduce Liouville's theorem from (1).
\end{enumerate}

\hypertarget{question-41-2}{%
\subsection{Question 41}\label{question-41-2}}

Let \(f(z)\) be entire and assume that \(f(z) \leq M |z|^2\) outside
some disk for some constant \(M\). Show that \(f(z)\) is a polynomial in
\(z\) of degree \(\leq 2\).

\hypertarget{question-42-2}{%
\subsection{Question 42}\label{question-42-2}}

Let \(a_n(z)\) be an analytic sequence in a domain \(D\) such that
\(\displaystyle \sum_{n=0}^\infty |a_n(z)|\) converges uniformly on
bounded and closed sub-regions of \(D\). Show that
\(\displaystyle \sum_{n=0}^\infty |a'_n(z)|\) converges uniformly on
bounded and closed sub-regions of \(D\).

\hypertarget{question-43-2}{%
\subsection{Question 43}\label{question-43-2}}

Let \(f(z)\) be analytic in an open set \(\Omega\) except possibly at a
point \(z_0\) inside \(\Omega\). Show that if \(f(z)\) is bounded in
near \(z_0\), then \(\displaystyle \int_\Delta f(z) dz = 0\) for all
triangles \(\Delta\) in \(\Omega\).

\hypertarget{question-44-2}{%
\subsection{Question 44}\label{question-44-2}}

Assume \(f\) is continuous in the region:
\(0< |z-a| \leq R, \; 0 \leq \arg(z-a) \leq \beta_0\)
(\(0 < \beta_0 \leq 2 \pi\)) and the limit
\(\displaystyle \lim_{z \rightarrow a} (z-a) f(z) = A\) exists. Show
that
\[\lim_{r \rightarrow 0} \int_{\gamma_r} f(z) dz  = i A \beta_0 \; , \; \;\]
where
\[\gamma_r : = \{ z \; | \; z = a + r e^{it}, \; 0 \leq  t \leq \beta_0 \}.\]

\hypertarget{question-45-2}{%
\subsection{Question 45}\label{question-45-2}}

Show that \(f(z) = z^2\) is uniformly continuous in any open disk
\(|z| < R\), where \(R>0\) is fixed, but it is not uniformly continuous
on \(\mathbb C\).

\begin{enumerate}
\def\labelenumi{(\arabic{enumi})}
\tightlist
\item
  Show that the function \(u=u(x,y)\) given by
  \[u(x,y)=\frac{e^{ny}-e^{-ny}}{2n^2}\sin nx\quad \text{for}\ n\in {\mathbf N}\]
  is the solution on \(D=\{(x,y)\ | x^2+y^2<1\}\) of the Cauchy problem
  for the Laplace equation
  \[\frac{\partial ^2u}{\partial x^2}+\frac{\partial ^2u}{\partial y^2}=0,\quad
  u(x,0)=0,\quad \frac{\partial u}{\partial y}(x,0)=\frac{\sin nx}{n}.\]
\end{enumerate}

\hypertarget{question-46-2}{%
\subsection{Question 46}\label{question-46-2}}

This question provides some insight into Cauchy's theorem. Solve the
problem without using Cauchy's theorem.

\begin{enumerate}
\def\labelenumi{\arabic{enumi}.}
\item
  Evaluate the integral \(\displaystyle{\int_{\gamma} z^n dz}\) for all
  integers \(n\). Here \(\gamma\) is any circle centered at the origin
  with the positive (counterclockwise) orientation.
\item
  Same question as (a), but with \(\gamma\) any circle not containing
  the origin.
\item
  Show that if \(|a|<r<|b|\), then
  \(\displaystyle{\int_{\gamma}\frac{dz}{(z-a)(z-b)} dz=\frac{2\pi i}{a-b}}\).
  Here \(\gamma\) denotes the circle centered at the origin, of radius
  \(r\), with the positive orientation.
\end{enumerate}

\hypertarget{question-47-2}{%
\subsection{Question 47}\label{question-47-2}}

\begin{enumerate}
\def\labelenumi{(\arabic{enumi})}
\item
  Assume the infinite series \(\displaystyle \sum_{n=0}^\infty c_n z^n\)
  converges in \(|z| < R\) and let \(f(z)\) be the limit. Show that for
  \(r <R\),
  \[\frac{1}{2 \pi} \int_0^{2 \pi} |f(r e^{i \theta})|^2 d \theta =
  \sum_{n=0}^\infty |c_n|^2 r^{2n} \; .\]
\item
  Deduce Liouville's theorem from (1).
\end{enumerate}

\begin{quote}
Liouville's theorem: If \(f(z)\) is entire and bounded, then \(f\) is
constant.
\end{quote}

\hypertarget{question-48-2}{%
\subsection{Question 48}\label{question-48-2}}

Let \(f\) be a continuous function in the region
\[D=\{z\ |  |z|>R, 0\leq \arg Z\leq \theta\}\quad\text{where}\quad
0\leq \theta \leq 2\pi.\] If there exists \(k\) such that
\(\displaystyle{\lim_{z\to\infty} zf(z)=k}\) for \(z\) in the region
\(D\). Show that \[\lim_{R'\to\infty} \int_{L} f(z) dz=i\theta k,\]
where \(L\) is the part of the circle \(|z|=R'\) which lies in the
region \(D\).

\hypertarget{question-49-2}{%
\subsection{Question 49}\label{question-49-2}}

Evaluate
\(\displaystyle{ \int_{0}^{\infty}\frac{x\sin x}{x^2+a^2} \, dx }\).

\hypertarget{question-50-2}{%
\subsection{Question 50}\label{question-50-2}}

Let \(f=u+iv\) be differentiable (i.e.~\(f'(z)\) exists) with continuous
partial derivatives at a point \(z=re^{i\theta}\), \(r\not= 0\). Show
that
\[\frac{\partial u}{\partial r}=\frac{1}{r}\frac{\partial v}{\partial \theta},\quad
\frac{\partial v}{\partial r}=-\frac{1}{r}\frac{\partial u}{\partial \theta}.\]

\hypertarget{question-51-2}{%
\subsection{Question 51}\label{question-51-2}}

Show that
\(\displaystyle \int_0^\infty \frac{x^{a-1}}{1+x^n} dx=\frac{\pi}{n\sin \frac{a\pi}{n}}\)
using complex analysis, \(0< a < n\). Here \(n\) is a positive integer.

\hypertarget{question-52-2}{%
\subsection{Question 52}\label{question-52-2}}

For \(s>0\), the \textbf{gamma function} is defined by
\(\displaystyle{\Gamma(s)=\int_0^{\infty} e^{-t}t^{s-1} dt}\).

\begin{itemize}
\item
  Show that the gamma function is analytic in the half-plane
  \(\Re (s)>0\), and is still given there by the integral formula above.
\item
  Apply the formula in the previous question to show that
  \[\Gamma(s)\Gamma(1-s)=\frac{\pi}{\sin \pi s}.\]
\end{itemize}

\begin{quote}
Hint: You may need
\(\displaystyle{\Gamma(1-s)=t \int_0^{\infty}e^{-vt}(vt)^{-s} dv}\) for
\(t>0\).
\end{quote}

\hypertarget{question-53-2}{%
\subsection{Question 53}\label{question-53-2}}

Suppose \(f\) is entire and there exist \(A, R >0\) and natural number
\(N\) such that \[|f(z)| \geq A |z|^N\ \text{for}\ |z| \geq R.\] Show
that

\begin{enumerate}
\def\labelenumi{(\roman{enumi})}
\item
  \(f\) is a polynomial and
\item
  the degree of \(f\) is at least \(N\).
\end{enumerate}

\hypertarget{question-54-2}{%
\subsection{Question 54}\label{question-54-2}}

Let \(f: {\mathbb C} \rightarrow {\mathbb C}\) be an injective analytic
(also called univalent) function. Show that there exist complex numbers
\(a \neq 0\) and \(b\) such that \(f(z) = az + b\).

\hypertarget{question-55-2}{%
\subsection{Question 55}\label{question-55-2}}

Let \(g\) be analytic for \(|z|\leq 1\) and \(|g(z)| < 1\) for
\(|z| = 1\).

\begin{itemize}
\item
  Show that \(g\) has a unique fixed point in \(|z| < 1\).
\item
  What happens if we replace \(|g(z)| < 1\) with \(|g(z)|\leq 1\) for
  \(|z|=1\)? Give an example if (a) is not true or give an proof if (a)
  is still true.
\item
  What happens if we simply assume that \(f\) is analytic for
  \(|z| < 1\) and \(|f(z)| < 1\) for \(|z| < 1\)? Suppose that
  \(f(z) \not\equiv z\). Can f have more than one fixed point in
  \(|z| < 1\)?
\end{itemize}

\begin{quote}
Hint: The map
\(\displaystyle{\psi_{\alpha}(z)=\frac{\alpha-z}{1-\bar{\alpha}z}}\) may
be useful.
\end{quote}

\hypertarget{question-56-2}{%
\subsection{Question 56}\label{question-56-2}}

Find a conformal map from \(D = \{z :\  |z| < 1,\ |z - 1/2| > 1/2\}\) to
the unit disk \(\Delta=\{z: \ |z|<1\}\).

\hypertarget{question-57-2}{%
\subsection{Question 57}\label{question-57-2}}

Let \(a_n \neq 0\) and assume that
\(\displaystyle \lim_{n \rightarrow \infty} \frac{|a_{n+1}|}{|a_n|} = L\).
Show that
\(\displaystyle \lim_{n \rightarrow \infty} \sqrt[n]{|a_n|} = L. %p_n^{\frac{1}{n}} = L.
\) In particular, this shows that when applicable, the ratio test can be
used to calculate the radius of convergence of a power series.

\hypertarget{question-58-2}{%
\subsection{Question 58}\label{question-58-2}}

\begin{enumerate}
\def\labelenumi{(\alph{enumi})}
\item
  Let \(z, w\) be complex numbers, such that \(\bar{z} w \neq 1\). Prove
  that
  \[\abs{\frac{w - z}{1 - \bar{w} z}} < 1 \; \; \; \mbox{if} \; |z| < 1 \; \mbox{and}\; |w| < 1,\]
  and also that
  \[\abs{\frac{w - z}{1 - \bar{w} z}} = 1 \; \; \; \mbox{if} \; |z| = 1 \; \mbox{or}\; |w| = 1.\]
\item
  Prove that for fixed \(w\) in the unit disk \(\mathbb D\), the mapping
  \[F: z \mapsto \frac{w - z}{1 - \bar{w} z}\] satisfies the following
  conditions:
\item
  \(F\) maps \(\mathbb D\) to itself and is holomorphic.~
\end{enumerate}

\begin{enumerate}
\def\labelenumi{(\roman{enumi})}
\setcounter{enumi}{1}
\item
  \(F\) interchanges \(0\) and \(w\), namely, \(F(0) = w\) and
  \(F(w) = 0\).
\item
  \(|F(z)| = 1\) if \(|z| = 1\).
\item
  \(F: {\mathbb D} \mapsto {\mathbb D}\) is bijective.
\end{enumerate}

\begin{quote}
Hint: Calculate \(F \circ F\).
\end{quote}

\hypertarget{question-59-2}{%
\subsection{Question 59}\label{question-59-2}}

Use \(n\)-th roots of unity (i.e.~solutions of \(z^n - 1 =0\)) to show
that
\[2^{n-1} \sin\frac{\pi}{n} \sin\frac{2\pi}{n} \cdots \sin\frac{(n-1)\pi}{n}
= n
\; .\]

\begin{quote}
Hint:
\(1 - \cos 2 \theta = 2 \sin^2 \theta,\; \sin 2 \theta = 2 \sin \theta \cos \theta\).
\end{quote}

\hypertarget{question-60-2}{%
\subsection{Question 60}\label{question-60-2}}

\begin{enumerate}
\def\labelenumi{(\alph{enumi})}
\tightlist
\item
  Show that in polar coordinates, the Cauchy-Riemann equations take the
  form
\end{enumerate}

\[\frac{\partial u}{\partial r} = \frac{1}{r} \frac{\partial v}{\partial \theta}
\; \; \; \text{and} \; \; \;
\frac{\partial v}{\partial r} = - \frac{1}{r} \frac{\partial u}{\partial \theta}\]

\begin{enumerate}
\def\labelenumi{(\alph{enumi})}
\setcounter{enumi}{1}
\tightlist
\item
  Use these equations to show that the logarithm function defined by
  \[\log z = \log r + i \theta \; \;
  \mbox{where} \; z = r e^{i \theta } \; \mbox{with} \; - \pi < \theta < \pi\]
  is a holomorphic function in the region
  \(r>0, \; - \pi < \theta < \pi\). Also show that \(\log z\) defined
  above is not continuous in \(r>0\).
\end{enumerate}

\hypertarget{question-61-2}{%
\subsection{Question 61}\label{question-61-2}}

Assume \(f\) is continuous in the region:
\(x \geq x_0, \; 0 \leq y \leq b\) and the limit
\[\displaystyle \lim_{x \rightarrow + \infty} f(x + iy) = A\] exists
uniformly with respect to \(y\) (independent of \(y\)).

Show that
\[\lim_{x \rightarrow + \infty} \int_{\gamma_x} f(z) dz  = iA b \; , \; \;\]
where \(\gamma_x : = \{ z \; | \; z = x + it, \; 0 \leq t \leq b\}.\)

\hypertarget{question-62-2}{%
\subsection{Question 62}\label{question-62-2}}

(Cauchy's formula for ``exterior'' region) Let \(\gamma\) be piecewise
smooth simple closed curve with interior \(\Omega_1\) and exterior
\(\Omega_2\). Assume \(f'(z)\) exists in an open set containing
\(\gamma\) and \(\Omega_2\) and
\(\lim_{z \rightarrow \infty } f(z) = A\). Show that
\[\frac{1}{2 \pi i} \int_\gamma \frac{f(\xi)}{\xi - z} \, d \xi =
\begin{cases}
A,          &     \text{if\ $z \in \Omega_1$}, \\
-f (z) + A, &  \text{if\ $z \in \Omega_2$}
\end{cases}\]

\hypertarget{question-63-2}{%
\subsection{Question 63}\label{question-63-2}}

Let \(f(z)\) be bounded and analytic in \(\mathbb C\). Let \(a \neq b\)
be any fixed complex numbers. Show that the following limit exists
\[\lim_{R \rightarrow \infty} \int_{|z|=R} \frac{f(z)}{(z-a)(z-b)} dz.\]
Use this to show that \(f(z)\) must be a constant (Liouville's theorem).

\hypertarget{question-64-2}{%
\subsection{Question 64}\label{question-64-2}}

Prove by \emph{justifying all steps} that for all
\(\xi \in {\mathbb C}\) we have
\(\displaystyle e^{- \pi \xi^2} = \int_{- \infty}^\infty e^{- \pi x^2} e^{2 \pi i x \xi} dx \; .\)

\begin{quote}
Hint: You may use that fact in Example 1 on p.~42 of the textbook
without proof, i.e., you may assume the above is true for real values of
\(\xi\).
\end{quote}

\hypertarget{question-65-2}{%
\subsection{Question 65}\label{question-65-2}}

Suppose that \(f\) is holomorphic in an open set containing the closed
unit disc, except for a pole at \(z_0\) on the unit circle. Let
\(f(z) = \sum_{n = 1}^\infty c_n z^n\) denote the power series in the
open disc. Show that

\begin{enumerate}
\def\labelenumi{(\arabic{enumi})}
\item
  \(c_n \neq 0\) for all large enough \(n\)'s, and
\item
  \(\displaystyle \lim_{n \rightarrow \infty} \frac{c_n}{c_{n+1}}= z_0\).
\end{enumerate}

\hypertarget{question-66-2}{%
\subsection{Question 66}\label{question-66-2}}

Let \(f(z)\) be a non-constant analytic function in \(|z|>0\) such that
\(f(z_n) = 0\) for infinite many points \(z_n\) with
\(\lim_{n \rightarrow \infty} z_n =0\). Show that \(z=0\) is an
essential singularity for \(f(z)\). (An example of such a function is
\(f(z) = \sin (1/z)\).)

\hypertarget{question-67-2}{%
\subsection{Question 67}\label{question-67-2}}

Let \(f\) be entire and suppose that
\(\lim_{z \rightarrow \infty} f(z) = \infty\). Show that \(f\) is a
polynomial.

\hypertarget{question-68-2}{%
\subsection{Question 68}\label{question-68-2}}

Expand the following functions into Laurent series in the indicated
regions:

\begin{enumerate}
\def\labelenumi{(\alph{enumi})}
\item
  \(\displaystyle f(z) = \frac{z^2 - 1}{ (z+2)(z+3)}, \; \; 2 < |z| < 3\),
  \(3 < |z| < + \infty\).
\item
  \(\displaystyle f(z) = \sin \frac{z}{1-z}, \; \; 0 < |z-1| < + \infty\)
\end{enumerate}

\hypertarget{question-69-2}{%
\subsection{Question 69}\label{question-69-2}}

Assume \(f(z)\) is analytic in region \(D\) and \(\Gamma\) is a
rectifiable curve in \(D\) with interior in \(D\). Prove that if
\(f(z)\) is real for all \(z \in \Gamma\), then \(f(z)\) is a constant.

\hypertarget{question-70-2}{%
\subsection{Question 70}\label{question-70-2}}

Find the number of roots of \(z^4 - 6z + 3 =0\) in \(|z|<1\) and
\(1 < |z| < 2\) respectively.

\hypertarget{question-71-2}{%
\subsection{Question 71}\label{question-71-2}}

Prove that \(z^4 + 2 z^3 - 2z + 10 =0\) has exactly one root in each
open quadrant.

\hypertarget{question-72-2}{%
\subsection{Question 72}\label{question-72-2}}

\begin{enumerate}
\def\labelenumi{(\arabic{enumi})}
\item
  Let \(f(z) \in H({\mathbb D})\), \(\text{Re}(f(z)) >0\),
  \(f(0)= a>0\). Show that
  \[\abs{ \frac{f(z)-a}{f(z)+a}} \leq |z|, \; \; \;
  |f'(0)| \leq 2a.\]
\item
  Show that the above is still true if \(\text{Re}(f(z)) >0\) is
  replaced with \(\text{Re}(f(z)) \geq 0\).
\end{enumerate}

\hypertarget{question-73-2}{%
\subsection{Question 73}\label{question-73-2}}

Assume \(f(z)\) is analytic in \({\mathbb D}\) and \(f(0)=0\) and is not
a rotation (i.e.~\(f(z) \neq e^{i \theta} z\)). Show that
\(\displaystyle \sum_{n=1}^\infty f^{n}(z)\) converges uniformly to an
analytic function on compact subsets of \({\mathbb D}\), where
\(f^{n+1}(z) = f(f^{n}(z))\).

\hypertarget{question-74-2}{%
\subsection{Question 74}\label{question-74-2}}

Let \(f(z) = \sum_{n=0}^\infty c_n z^n\) be analytic and one-to-one in
\(|z| < 1\). For \(0<r<1\), let \(D_r\) be the disk \(|z|<r\). Show that
the area of \(f(D_r)\) is finite and is given by
\[S = \pi \sum_{n=1}^\infty n |c_n|^2 r^{2n}.\] (Note that in general
the area of \(f(D_1)\) is infinite.)

\hypertarget{question-75-2}{%
\subsection{Question 75}\label{question-75-2}}

Let \(f(z) = \sum_{n= -\infty}^\infty c_n z^n\) be analytic and
one-to-one in \(r_0< |z| < R_0\). For \(r_0<r<R<R_0\), let \(D(r,R)\) be
the annulus \(r<|z|<R\). Show that the area of \(f(D(r,R))\) is finite
and is given by
\[S = \pi \sum_{n=- \infty}^\infty n |c_n|^2 (R^{2n} - r^{2n}).\]

\hypertarget{question-76-2}{%
\subsection{Question 76}\label{question-76-2}}

Let \(a_n(z)\) be an analytic sequence in a domain \(D\) such that
\(\displaystyle \sum_{n=0}^\infty |a_n(z)|\) converges uniformly on
bounded and closed sub-regions of \(D\). Show that
\(\displaystyle \sum_{n=0}^\infty |a'_n(z)|\) converges uniformly on
bounded and closed sub-regions of \(D\).

\hypertarget{question-77-2}{%
\subsection{Question 77}\label{question-77-2}}

Let \(f_n, f\) be analytic functions on the unit disk \({\mathbb D}\).
Show that the following are equivalent.

\begin{enumerate}
\def\labelenumi{(\roman{enumi})}
\item
  \(f_n(z)\) converges to \(f(z)\) uniformly on compact subsets in
  \(\mathbb D\).
\item
  \(\int_{|z|= r} |f_n(z) - f(z)| \, |dz|\) converges to \(0\) if
  \(0< r<1\).
\end{enumerate}

\hypertarget{question-78-2}{%
\subsection{Question 78}\label{question-78-2}}

Let \(f\) and \(g\) be non-zero analytic functions on a region
\(\Omega\). Assume \(|f(z)| = |g(z)|\) for all \(z\) in \(\Omega\). Show
that \(f(z) = e^{i \theta} g(z)\) in \(\Omega\) for some
\(0 \leq \theta < 2 \pi\).

\hypertarget{question-79-2}{%
\subsection{Question 79}\label{question-79-2}}

Suppose \(f\) is analytic in an open set containing the unit disc
\(\mathbb D\) and \(|f(z)| =1\) when \(|z|\)=1. Show that either
\(f(z) = e^{i \theta}\) for some \(\theta \in \mathbb R\) or there are
finite number of \(z_k \in \mathbb D\), \(k \leq n\) and
\(\theta \in \mathbb R\) such that
\(\displaystyle f(z) = e^{i\theta} \prod_{k=1}^n \frac{z-z_k}{1 - \bar{z}_k z } \, .\)

\begin{quote}
Also cf.~Stein et al, 1.4.7, 3.8.17
\end{quote}

\hypertarget{question-80-2}{%
\subsection{Question 80}\label{question-80-2}}

\begin{enumerate}
\def\labelenumi{(\arabic{enumi})}
\item
  Let \(p(z)\) be a polynomial, \(R>0\) any positive number, and
  \(m \geq 1\) an integer. Let
  \[M_R = \sup \{ |z^{m} p(z) - 1|: |z| = R  \}.\] Show that \(M_R>1\).
\item
  Let \(m \geq 1\) be an integer and
  \(K = \{z \in {\mathbb C}: r \leq |z| \leq R \}\) where \(r<R\). Show
  (i) using (1) as well as, (ii) without using (1) that there exists a
  positive number \(\varepsilon_0>0\) such that for each polynomial
  \(p(z)\),
  \[\sup \{|p(z) - z^{-m}|: z \in K  \} \geq \varepsilon_0 \, .\]
\end{enumerate}

\hypertarget{question-81-2}{%
\subsection{Question 81}\label{question-81-2}}

Let \(\displaystyle f(z) = \frac{1}{z} + \frac{1}{z^2 -1}\). Find all
the Laurent series of \(f\) and describe the largest annuli in which
these series are valid.

\hypertarget{question-82-2}{%
\subsection{Question 82}\label{question-82-2}}

Suppose \(f\) is entire and there exist \(A, R >0\) and natural number
\(N\) such that \(|f(z)| \leq A |z|^N\) for \(|z| \geq R\). Show that

\begin{enumerate}
\def\labelenumi{(\roman{enumi})}
\item
  \(f\) is a polynomial and
\item
  the degree of \(f\) is at most \(N\).
\end{enumerate}

\hypertarget{question-83-2}{%
\subsection{Question 83}\label{question-83-2}}

\begin{enumerate}
\def\labelenumi{(\arabic{enumi})}
\item
  Explicitly write down an example of a non-zero analytic function in
  \(|z|<1\) which has infinitely zeros in \(|z|<1\).
\item
  Why does not the phenomenon in (1) contradict the uniqueness theorem?
\end{enumerate}

\hypertarget{question-84-2}{%
\subsection{Question 84}\label{question-84-2}}

\begin{enumerate}
\def\labelenumi{(\arabic{enumi})}
\item
  Assume \(u\) is harmonic on open set \(O\) and \(z_n\) is a sequence
  in \(O\) such that \(u(z_n) = 0\) and \(\lim z_n \in O\). Prove or
  disprove that \(u\) is identically zero. What if \(O\) is a region?
\item
  Assume \(u\) is harmonic on open set \(O\) and \(u(z) = 0\) on a disc
  in \(O\). Prove or disprove that \(u\) is identically zero. What if
  \(O\) is a region?
\item
  Formulate and prove a Schwarz reflection principle for harmonic
  functions
\end{enumerate}

\begin{quote}
cf.~Theorem 5.6 on p.60 of Stein et al.
\end{quote}

\begin{quote}
Hint: Verify the mean value property for your new function obtained by
Schwarz reflection principle.
\end{quote}

\hypertarget{question-85-2}{%
\subsection{Question 85}\label{question-85-2}}

Let \(f\) be holomorphic in a neighborhood of \(D_r(z_0)\). Show that
for any \(s<r\), there exists a constant \(c>0\) such that
\[||f||_{(\infty, s)} \leq c ||f||_{(1, r)},\] where
\(\displaystyle |f||_{(\infty, s)} = \text{sup}_{z \in D_s(z_0)}|f(z)|\)
and \(\displaystyle ||f||_{(1, r)} = \int_{D_r(z_0)} |f(z)|dx dy\).

\begin{quote}
Note: Exercise 3.8.20 on p.107 in Stein et al is a straightforward
consequence of this stronger result using the integral form of the
Cauchy-Schwarz inequality in real analysis.
\end{quote}

\hypertarget{question-86-1}{%
\subsection{Question 86}\label{question-86-1}}

\begin{enumerate}
\def\labelenumi{(\arabic{enumi})}
\item
  Let \(f\) be analytic in \(\Omega: 0<|z-a|<r\) except at a sequence of
  poles \(a_n \in \Omega\) with \(\lim_{n \rightarrow \infty} a_n = a\).
  Show that for any \(w \in \mathbb C\), there exists a sequence
  \(z_n \in \Omega\) such that
  \(\lim_{n \rightarrow \infty} f(z_n) = w\).
\item
  Explain the similarity and difference between the above assertion and
  the Weierstrass-Casorati theorem.
\end{enumerate}

\hypertarget{question-87-1}{%
\subsection{Question 87}\label{question-87-1}}

Compute the following integrals.

\(i\) \(\displaystyle \int_0^\infty \frac{1}{(1 + x^n)^2} \, dx\),
\(n \geq 1\) (ii)
\(\displaystyle \int_0^\infty \frac{\cos x}{(x^2 + a^2)^2} \, dx\),
\(a \in \mathbb R\) (iii)
\(\displaystyle \int_0^\pi \frac{1}{a + \sin \theta} \, d \theta\),
\(a>1\)

\(iv\)
\(\displaystyle \int_0^{\frac{\pi}{2}} \frac{d \theta}{a+ \sin ^2 \theta},\)
\(a >0\). (v)
\(\displaystyle \int_{|z|=2} \frac{1}{(z^{5} -1) (z-3)} \, dz\) (v)
\(\displaystyle \int_{- \infty}^{\infty} \frac{\sin \pi a}{\cosh \pi x + \cos \pi a} e^{- i x \xi} \, d x\),
\(0< a <1\), \(\xi \in \mathbb R\) (vi)
\(\displaystyle \int_{|z| = 1} \cot^2 z \, dz\).

\hypertarget{question-88-1}{%
\subsection{Question 88}\label{question-88-1}}

Compute the following integrals.

\(i\) \(\displaystyle \int_0^\infty \frac{\sin x}{x} \, dx\) (ii)
\(\displaystyle \int_0^\infty (\frac{\sin x}{x})^2 \, dx\) (iii)
\(\displaystyle \int_0^\infty \frac{x^{a-1}}{(1 + x)^2} \, dx\),
\(0< a < 2\)

\(i\) \(\displaystyle \int_0^\infty \frac{\cos a x - \cos bx}{x^2} dx\),
\(a, b >0\) (ii)
\(\displaystyle \int_0^\infty \frac{x^{a-1}}{1 + x^n} \, dx\),
\(0< a < n\)

\(iii\) \(\displaystyle \int_0^\infty \frac{\log x}{1 + x^n} \, dx\),
\(n \geq 2\) (iv)
\(\displaystyle \int_0^\infty \frac{\log x}{(1 + x^2)^2} dx\) (v)
\(\displaystyle \int_0^{\pi} \log|1 - a \sin \theta| d \theta\),
\(a \in \mathbb C\)

\hypertarget{question-89-1}{%
\subsection{Question 89}\label{question-89-1}}

Let \(0<r<1\). Show that polynomials
\(P_n(z) = 1 + 2z + 3 z^2 + \cdots + n z^{n-1}\) have no zeros in
\(|z|<r\) for all sufficiently large \(n\)'s.

\hypertarget{question-90-1}{%
\subsection{Question 90}\label{question-90-1}}

Let \(f\) be an analytic function on a region \(\Omega\). Show that
\(f\) is a constant if there is a simple closed curve \(\gamma\) in
\(\Omega\) such that its image \(f(\gamma)\) is contained in the real
axis.

\hypertarget{question-91-1}{%
\subsection{Question 91}\label{question-91-1}}

\begin{enumerate}
\def\labelenumi{(\arabic{enumi})}
\item
  Show that \(\displaystyle \frac{\pi^2}{\sin^2 \pi z}\) and
  \(\displaystyle g(z) = \sum_{n = - \infty}^{ \infty} \frac{1}{(z-n)^2}\)
  have the same principal part at each integer point.
\item
  Show that \(\displaystyle h(z) = \frac{\pi^2}{\sin^2 \pi z} - g(z)\)
  is bounded on \(\mathbb C\) and conclude that
  \(\displaystyle \frac{\pi^2}{\sin^2 \pi z} = \sum_{n = - \infty}^{ \infty} \frac{1}{(z-n)^2} \, .\)
\end{enumerate}

\hypertarget{question-92-1}{%
\subsection{Question 92}\label{question-92-1}}

Let \(f(z)\) be an analytic function on
\({\mathbb C} \backslash \{ z_0 \}\), where \(z_0\) is a fixed point.
Assume that \(f(z)\) is bijective from
\({\mathbb C} \backslash \{ z_0 \}\) onto its image, and that \(f(z)\)
is bounded outside \(D_r(z_0)\), where \(r\) is some fixed positive
number. Show that there exist \(a, b, c, d \in \mathbb C\) with
\(ad-bc \neq 0\), \(c \neq 0\) such that
\(\displaystyle f(z) = \frac{az + b}{cz + d}\).

\hypertarget{question-93-1}{%
\subsection{Question 93}\label{question-93-1}}

Assume \(f(z)\) is analytic in \(\mathbb D: |z|<1\) and \(f(0)=0\) and
is not a rotation (i.e.~\(f(z) \neq e^{i \theta} z\)). Show that
\(\displaystyle \sum_{n=1}^\infty f^{n}(z)\) converges uniformly to an
analytic function on compact subsets of \({\mathbb D}\), where
\(f^{n+1}(z) = f(f^{n}(z))\).

\hypertarget{question-94-1}{%
\subsection{Question 94}\label{question-94-1}}

Let \(f\) be a non-constant analytic function on \(\mathbb D\) with
\(f(\mathbb D) \subseteq \mathbb D\). Use \(\psi_{a} (f(z))\) (where
\(a=f(0)\), \(\displaystyle \psi_a(z) = \frac{a - z}{1 - \bar{a}z}\)) to
prove that \[\displaystyle
\frac{|f(0)| - |z|}{1 + |f(0)||z|} \leq |f(z)| \leq
\frac{|f(0)| + |z|}{1 - |f(0)||z|}.\]

\hypertarget{question-95-1}{%
\subsection{Question 95}\label{question-95-1}}

Find a conformal map

\begin{enumerate}
\def\labelenumi{\arabic{enumi}.}
\item
  From \(\{ z: |z - 1/2| > 1/2, \text{Re}(z)>0 \}\) to \(\mathbb H\)
\item
  From \(\{ z: |z - 1/2| > 1/2, |z| <1 \}\) to \(\mathbb D\)
\item
  From the intersection of the disk \(|z + i| < \sqrt{2}\) with
  \({\mathbb H}\) to \({\mathbb D}\).
\item
  From \({\mathbb D} \backslash [a, 1)\) to
  \({\mathbb D} \backslash [0, 1)\) (\(0<a<1)\).

  \begin{quote}
  Hint: Short solution possible using Blaschke factor
  \end{quote}
\item
  From \(\{ z: |z| < 1, \text{Re}(z) > 0 \} \backslash (0, 1/2]\) to
  \(\mathbb H\).
\end{enumerate}

\hypertarget{question-96-1}{%
\subsection{Question 96}\label{question-96-1}}

Let \(C\) and \(C'\) be two circles and let \(z_1 \in C\),
\(z_2 \notin C\), \(z'_1 \in C'\), \(z'_2 \notin C'\). Show that there
is a unique fractional linear transformation \(f\) with \(f(C) = C'\)
and \(f(z_1) = z'_1\), \(f(z_2) = z'_2\).

\hypertarget{question-97-1}{%
\subsection{Question 97}\label{question-97-1}}

Assume \(f_n \in H(\Omega)\) is a sequence of holomorphic functions on
the region \(\Omega\) that are uniformly bounded on compact subsets and
\(f \in H(\Omega)\) is such that the set
\(\displaystyle \{z \in \Omega: \lim_{n \rightarrow \infty} f_n(z) = f(z) \}\)
has a limit point in \(\Omega\). Show that \(f_n\) converges to \(f\)
uniformly on compact subsets of \(\Omega\).

\hypertarget{question-98-1}{%
\subsection{Question 98}\label{question-98-1}}

Let \(\displaystyle{\psi_{\alpha}(z)=\frac{\alpha-z}{1-\bar{\alpha}z}}\)
with \(|\alpha|<1\) and \({\mathbb D}=\{z:\ |z|<1\}\). Prove that

\begin{itemize}
\item
  \(\displaystyle{\frac{1}{\pi}\iint_{{\mathbb D}} |\psi'_{\alpha}|^2 dx dy =1}\).
\item
  \(\displaystyle{\frac{1}{\pi}\iint_{{\mathbb D}} |\psi'_{\alpha}| dx dy =\frac{1-|\alpha|^2}{|\alpha|^2} \log \frac{1}{1-|\alpha|^2}}\).
\end{itemize}

\hypertarget{question-99-1}{%
\subsection{Question 99}\label{question-99-1}}

Prove that
\(\displaystyle{f(z)=-\frac{1}{2}\left(z+\frac{1}{z}\right)}\) is a
conformal map from half disc \(\{z=x+iy:\ |z|<1,\ y>0\}\) to upper half
plane \({\mathbb H}=\{z=x+iy:\ y>0\}\).

\hypertarget{question-100-1}{%
\subsection{Question 100}\label{question-100-1}}

Let \(\Omega\) be a simply connected open set and let \(\gamma\) be a
simple closed contour in \(\Omega\) and enclosing a bounded region \(U\)
anticlockwise. Let \(f: \ \Omega \to {\mathbb C}\) be a holomorphic
function and \(|f(z)|\leq M\) for all \(z\in \gamma\). Prove that
\(|f(z)|\leq M\) for all \(z\in U\).

\hypertarget{question-101-1}{%
\subsection{Question 101}\label{question-101-1}}

Compute the following integrals.

\begin{enumerate}
\def\labelenumi{(\roman{enumi})}
\item
  \(\displaystyle \int_0^\infty \frac{x^{a-1}}{1 + x^n} \, dx\),
  \(0< a < n\)
\item
  \(\displaystyle \int_0^\infty \frac{\log x}{(1 + x^2)^2}\, dx\)
\end{enumerate}

\hypertarget{question-102-1}{%
\subsection{Question 102}\label{question-102-1}}

Let \(0<r<1\). Show that the polynomials
\(P_n(z) = 1 + 2z + 3 z^2 + \cdots + n z^{n-1}\) have no zeros in
\(|z|<r\) for all sufficiently large \(n\)'s.

\hypertarget{question-103-1}{%
\subsection{Question 103}\label{question-103-1}}

Let \(f\) be holomorphic in a neighborhood of \(D_r(z_0)\). Show that
for any \(s<r\), there exists a constant \(c>0\) such that
\[\|f\|_{(\infty, s)} \leq c \|f\|_{(1, r)},\] where
\(\displaystyle \|f\|_{(\infty, s)} = \text{sup}_{z \in D_s(z_0)}|f(z)|\)
and \(\displaystyle \|f\|_{(1, r)} = \int_{D_r(z_0)} |f(z)|dx dy\).

\hypertarget{question-104-1}{%
\subsection{Question 104}\label{question-104-1}}

Let \(\displaystyle{\psi_{\alpha}(z)=\frac{\alpha-z}{1-\bar{\alpha}z}}\)
with \(|\alpha|<1\) and \({\mathbb D}=\{z:\ |z|<1\}\). Prove that

\begin{itemize}
\item
  \(\displaystyle{\frac{1}{\pi}\iint_{{\mathbb D}} |\psi'_{\alpha}|^2 dx dy =1}\).
\item
  \(\displaystyle{\frac{1}{\pi}\iint_{{\mathbb D}} |\psi'_{\alpha}| dx dy =\frac{1-|\alpha|^2}{|\alpha|^2} \log \frac{1}{1-|\alpha|^2}}\).
\end{itemize}

\hypertarget{question-105-1}{%
\subsection{Question 105}\label{question-105-1}}

Let \(\Omega\) be a simply connected open set and let \(\gamma\) be a
simple closed contour in \(\Omega\) and enclosing a bounded region \(U\)
anticlockwise. Let \(f: \ \Omega \to {\mathbb C}\) be a holomorphic
function and \(|f(z)|\leq M\) for all \(z\in \gamma\). Prove that
\(|f(z)|\leq M\) for all \(z\in U\).

\hypertarget{question-106-1}{%
\subsection{Question 106}\label{question-106-1}}

Compute the following integrals.

\begin{enumerate}
\def\labelenumi{(\roman{enumi})}
\item
  \(\displaystyle \int_0^\infty \frac{x^{a-1}}{1 + x^n} \, dx\),
  \(0< a < n\)
\item
  \(\displaystyle \int_0^\infty \frac{\log x}{(1 + x^2)^2}\, dx\)
\end{enumerate}

\hypertarget{question-107-1}{%
\subsection{Question 107}\label{question-107-1}}

Let \(f\) be holomorphic in a neighborhood of \(D_r(z_0)\). Show that
for any \(s<r\), there exists a constant \(c>0\) such that
\[\|f\|_{(\infty, s)} \leq c \|f\|_{(1, r)},\] where
\(\displaystyle \|f\|_{(\infty, s)} = \text{sup}_{z \in D_s(z_0)}|f(z)|\)
and \(\displaystyle \|f\|_{(1, r)} = \int_{D_r(z_0)} |f(z)|dx dy\).

\hypertarget{question-108-1}{%
\subsection{Question 108}\label{question-108-1}}

Let \(u(x,y)\) be harmonic and have continuous partial derivatives of
order three in an open disc of radius \(R>0\).

\begin{enumerate}
\def\labelenumi{(\alph{enumi})}
\item
  Let two points \((a,b), (x,y)\) in this disk be given. Show that the
  following integral is independent of the path in this disk joining
  these points:
  \[v(x,y) = \int_{a,b}^{x,y} ( -\frac{\partial u}{\partial y}dx +  \frac{\partial u}{\partial x}dy).\]\\
\item
  \hfill

  \begin{enumerate}
  \def\labelenumii{(\roman{enumii})}
  \item
    Prove that \(u(x,y)+i v(x,y)\) is an analytic function in this disc.
  \item
    Prove that \(v(x,y)\) is harmonic in this disc.
  \end{enumerate}
\end{enumerate}

\hypertarget{question-109-1}{%
\subsection{Question 109}\label{question-109-1}}

\begin{enumerate}
\def\labelenumi{(\alph{enumi})}
\item
  \(f(z)= u(x,y) +i v(x,y)\) be analytic in a domain
  \(D\subset {\mathbb C}\). Let \(z_0=(x_0,y_0)\) be a point in \(D\)
  which is in the intersection of the curves \(u(x,y)= c_1\) and
  \(v(x,y)=c_2\), where \(c_1\) and \(c_2\) are constants. Suppose that
  \(f'(z_0)\neq 0\). Prove that the lines tangent to these curves at
  \(z_0\) are perpendicular.
\item
  Let \(f(z)=z^2\) be defined in \({\mathbb C}\).
\item
  Describe the level curves of \(\mbox{\textrm Re}{(f)}\) and of
  \(\mbox{Im}{(f)}\).
\end{enumerate}

\begin{enumerate}
\def\labelenumi{(\roman{enumi})}
\setcounter{enumi}{1}
\tightlist
\item
  What are the angles of intersections between the level curves
  \(\mbox{\textrm Re}{(f)}=0\) and \(\mbox{\textrm Im}{(f)}\)? Is your
  answer in agreement with part a) of this question?
\end{enumerate}

\hypertarget{question-110-1}{%
\subsection{Question 110}\label{question-110-1}}

\begin{enumerate}
\def\labelenumi{(\alph{enumi})}
\item
  Let \(f: D\rightarrow \mathbb C\) be a continuous function, where
  \(D\subset \mathbb C\) is a domain. Let \(\alpha:[a,b]\rightarrow D\)
  be a smooth curve. Give a precise definition of the \emph{complex line
  integral} \[\int_{\alpha} f.\]
\item
  Assume that there exists a constant \(M\) such that
  \(|f(\tau)|\leq M\) for all \(\tau\in \mbox{\textrm Image}(\alpha\)).
  Prove that
  \[\big | \int_{\alpha} f \big |\leq M \times \mbox{\textrm length}(\alpha).\]
\item
  Let \(C_R\) be the circle \(|z|=R\), described in the counterclockwise
  direction, where \(R>1\). Provide an upper bound for
  \(\abs{ \int_{C_R} \dfrac{\log{(z)} }{z^2} }\) which depends
  \emph{only} on \(R\) and other constants.
\end{enumerate}

\hypertarget{question-111-1}{%
\subsection{Question 111}\label{question-111-1}}

\begin{enumerate}
\def\labelenumi{(\alph{enumi})}
\item
  Let \(f:{\mathbb C}\rightarrow {\mathbb C}\) be an entire function.
  Assume the existence of a non-negative integer \(m\), and of positive
  constants \(L\) and \(R\), such that for all \(z\) with \(|z|>R\) the
  inequality \[|f(z)| \leq L |z|^m\] holds. Prove that \(f\) is a
  polynomial of degree \(\leq m\).
\item
  Let \(f:{\mathbb C}\rightarrow {\mathbb C}\) be an entire function.
  Suppose that there exists a real number M such that for all
  \(z\in {\mathbb C}\) \[\mbox{\textrm Re} (f) \leq M.\] Prove that
  \(f\) must be a constant.
\end{enumerate}

\hypertarget{question-112-1}{%
\subsection{Question 112}\label{question-112-1}}

Prove that all the roots of the complex polynomial
\[z^7 - 5 z^3 +12 =0\] lie between the circles \(|z|=1\) and \(|z|=2\).

\hypertarget{question-113-1}{%
\subsection{Question 113}\label{question-113-1}}

Let \(F\) be an analytic function inside and on a simple closed curve
\(C\), except for a pole of order \(m\geq 1\) at \(z=a\) inside \(C\).
Prove that

\[
\frac{1}{2 \pi i}\oint_{C} F(\tau) d\tau = 
\lim_{\tau\rightarrow a} \frac{d^{m-1}}{d\tau^{m-1}}\big((\tau-a)^m F(\tau))\big)
.\]

\hypertarget{question-114-1}{%
\subsection{Question 114}\label{question-114-1}}

Find the conformal map that takes the upper half-plane comformally onto
the half-strip \(\{ w=x+iy:\ -\pi/2<x<\pi/2\ y>0\}\).

\hypertarget{question-115-1}{%
\subsection{Question 115}\label{question-115-1}}

Compute the integral
\(\displaystyle{\int_{-\infty}^{\infty} \frac{e^{-2\pi ix\xi}}{\cosh\pi x}dx}\)
where \(\displaystyle{\cosh z=\frac{e^{z}+e^{-z}}{2}}\).

\hypertarget{question-116-1}{%
\subsection{Question 116}\label{question-116-1}}

Use residues to compute the integral \begin{align*}
\int_{0}^{\infty} \dfrac{\cos x}{(x^2+1)^2} \mathrm{d}x
\end{align*}

\hypertarget{question-117-1}{%
\subsection{Question 117}\label{question-117-1}}

State and prove the Cauchy integral formula for holomorphic functions.

\hypertarget{question-118-1}{%
\subsection{Question 118}\label{question-118-1}}

Let \(f\) be an entire function and suppose that \(|f(z)| \leq A|z|^2\)
for all \(z\) and some constant \(A\). Show that \(f\) is a polynomial
of degree \(\leq 2\).

\hypertarget{question-119-1}{%
\subsection{Question 119}\label{question-119-1}}

\begin{enumerate}
\def\labelenumi{\arabic{enumi}.}
\item
  State the Schwarz lemma for analytic functions in the unit disc.
\item
  Let \(f: \mathbb{D} \to \mathbb{D}\) be an analytic map from the unit
  disc \(\mathbb{D}\) into itself. Use the Schwarz lemma to show that
  for each \(a\in \mathbb{D}\) we have \begin{align*}
  \dfrac{|f'(a)|}{1-|f(a)|^2} \leq \dfrac{1}{1-|a|^2}
  \end{align*}
\end{enumerate}

\hypertarget{question-120-1}{%
\subsection{Question 120}\label{question-120-1}}

State the Riemann mapping theorem and prove the uniqueness part.

\hypertarget{question-121-1}{%
\subsection{Question 121}\label{question-121-1}}

Compute the integrals \begin{align*}
\int_{|z-2|=1} \dfrac{e^z}{z(z-1)^2} \,
\mathrm{d}z, \quad \int_0^\infty \dfrac{\cos 2x}{x^2 + 2} \, \mathrm{d}x
\end{align*}

\hypertarget{question-122-1}{%
\subsection{Question 122}\label{question-122-1}}

Let \((f_n)\) be a sequence of holomorphic functions in a domain \(D\).
Suppose that \(f_n \to f\) uniformly on each compact subset of \(D\).
Show that

\begin{itemize}
\item
  \(f\) is holomorphic on \(D\).
\item
  \(f_n' \to f'\) uniformly on each compact subset of \(D\).
\end{itemize}

\hypertarget{question-123-1}{%
\subsection{Question 123}\label{question-123-1}}

If \(f\) is a non-constant entire function, then \(f(\mathbb{C})\) is
dense in the plane.

\hypertarget{question-124-1}{%
\subsection{Question 124}\label{question-124-1}}

\begin{enumerate}
\def\labelenumi{\arabic{enumi}.}
\item
  State Rouche's theorem.
\item
  Let \(f\) be analytic in a neighborhood of \(0\), and satisfying
  \(f'(0) \neq 0\). Use Rouche's theorem to show that there exists a
  neighborhood \(U\) of \(0\) such that \(f\) is a bijection in \(U\).
\end{enumerate}

\hypertarget{question-125-1}{%
\subsection{Question 125}\label{question-125-1}}

Let \(f\) be a meromorphic function in the plane such that
\begin{align*}
\lim_{|z|\to\infty} |f(z)| = \infty
\end{align*}

\begin{enumerate}
\def\labelenumi{\arabic{enumi}.}
\item
  Show that \(f\) has only finitely many poles.
\item
  Show that \(f\) is a rational function.
\end{enumerate}

\hypertarget{topology-158-questions}{%
\section{Topology (158 Questions)}\label{topology-158-questions}}

\hypertarget{question-1-3}{%
\subsection{Question 1}\label{question-1-3}}

Suppose \((X, d)\) is a metric space. State criteria for continuity of a
function \(f : X \to X\) in terms of:

\begin{enumerate}
\def\labelenumi{\roman{enumi}.}
\item
  open sets;
\item
  \(\eps\)'s and \(\delta\)'s; and
\item
  convergent sequences.
\end{enumerate}

Then prove that (iii) implies (i).

\hypertarget{question-2-3}{%
\subsection{Question 2}\label{question-2-3}}

Let \(X\) be a topological space.

\begin{enumerate}
\def\labelenumi{\roman{enumi}.}
\item
  State what it means for \(X\) to be compact.
\item
  Let \(X = \theset{0} \cup \theset{{1\over n} \mid n \in \ZZ^+ }\). Is
  \(X\) compact?
\item
  Let \(X = (0, 1]\). Is \(X\) compact?
\end{enumerate}

\hypertarget{question-3-3}{%
\subsection{Question 3}\label{question-3-3}}

Let \((X, d)\) be a compact metric space, and let \(f : X \to X\) be an
isometry: \[\forall~ x, y \in X, \qquad d(f (x), f (y)) = d(x, y).\]
Prove that \(f\) is a bijection.

\hypertarget{question-4-3}{%
\subsection{Question 4}\label{question-4-3}}

Suppose \((X, d)\) is a compact metric space and \(U\) is an open
covering of \(X\).

Prove that there is a number \(\delta > 0\) such that for every
\(x \in X\), the ball of radius \(\delta\) centered at \(x\) is
contained in some element of \(U\).

\hypertarget{question-5-3}{%
\subsection{Question 5}\label{question-5-3}}

Let \(X\) be a topological space, and \(B \subset A \subset X\). Equip
\(A\) with the subspace topology, and write \(\cl_X (B)\) or
\(\cl_A (B)\) for the closure of \(B\) as a subset of, respectively,
\(X\) or \(A\).

Determine, with proof, the general relationship between
\(\cl_X (B) \cap A\) and \(\cl_A (B)\)

\begin{quote}
I.e., are they always equal? Is one always contained in the other but
not conversely? Neither?
\end{quote}

\hypertarget{question-6-3}{%
\subsection{Question 6}\label{question-6-3}}

Prove that the unit interval \(I\) is compact. Be sure to explicitly
state any properties of \(\RR\) that you use.

\hypertarget{question-7-3}{%
\subsection{Question 7}\label{question-7-3}}

A topological space is \textbf{sequentially compact} if every infinite
sequence in \(X\) has a convergent subsequence.

Prove that every compact metric space is sequentially compact.

\hypertarget{question-8-3}{%
\subsection{Question 8}\label{question-8-3}}

Show that for any two topological spaces \(X\) and \(Y\) ,
\(X \cross Y\) is compact if and only if both \(X\) and \(Y\) are
compact.

\hypertarget{question-9-3}{%
\subsection{Question 9}\label{question-9-3}}

Recall that a topological space is said to be \textbf{connected} if
there does not exist a pair \(U, V\) of disjoint nonempty subsets whose
union is \(X\).

\begin{enumerate}
\def\labelenumi{\roman{enumi}.}
\item
  Prove that \(X\) is connected if and only if the only subsets of \(X\)
  that are both open and closed are \(X\) and the empty set.
\item
  Suppose that \(X\) is connected and let \(f : X \to \RR\) be a
  continuous map. If \(a\) and \(b\) are two points of \(X\) and \(r\)
  is a point of \(\RR\) lying between \(f (a)\) and \(f (b)\) show that
  there exists a point \(c\) of \(X\) such that \(f (c) = r\).
\end{enumerate}

\hypertarget{question-10-3}{%
\subsection{Question 10}\label{question-10-3}}

Let \[
X = \theset{(0, y) \mid - 1 \leq y \leq 1} \cup \theset{\qty{x, s = \sin\qty{1 \over x}} \mid 0 < x \leq 1}
.\]

Prove that \(X\) is connected but not path connected.

\hypertarget{question-11-3}{%
\subsection{Question 11}\label{question-11-3}}

Let \begin{align*}
X=\left\{(x, y) \in \mathbb{R}^{2} | x>0, y \geq 0, \text { and } \frac{y}{x} \text { is rational }\right\}
\end{align*} and equip \(X\) with the subspace topology induced by the
usual topology on \(\RR^2\).

Prove or disprove that \(X\) is connected.

\hypertarget{question-12-3}{%
\subsection{Question 12}\label{question-12-3}}

Write \(Y\) for the interval \([0, \infty)\), equipped with the usual
topology.

Find, with proof, all subspaces \(Z\) of \(Y\) which are retracts of
\(Y\).

\hypertarget{question-13-3}{%
\subsection{Question 13}\label{question-13-3}}

\begin{enumerate}
\def\labelenumi{\alph{enumi}.}
\item
  Prove that if the space \(X\) is connected and locally path connected
  then \(X\) is path connected.
\item
  Is the converse true? Prove or give a counterexample.
\end{enumerate}

\hypertarget{question-14-3}{%
\subsection{Question 14}\label{question-14-3}}

Let \(\theset{X_\alpha \mid \alpha \in A}\) be a family of connected
subspaces of a space \(X\) such that there is a point \(p \in X\) which
is in each of the \(X_\alpha\).

Show that the union of the \(X_\alpha\) is connected.

\hypertarget{question-15-3}{%
\subsection{Question 15}\label{question-15-3}}

Let \(X\) be a topological space.

\begin{enumerate}
\def\labelenumi{\alph{enumi}.}
\item
  Prove that \(X\) is connected if and only if there is no continuous
  nonconstant map to the discrete two-point space \(\theset{0, 1}\).
\item
  Suppose in addition that \(X\) is compact and \(Y\) is a connected
  Hausdorff space. Suppose further that there is a continuous map
  \(f : X \to Y\) such that every preimage \(f\inv (y)\) for
  \(y \in Y\), is a connected subset of \(X\).

  Show that \(X\) is connected.
\item
  Give an example showing that the conclusion of (b) may be false if
  \(X\) is not compact.
\end{enumerate}

\hypertarget{question-16-3}{%
\subsection{Question 16}\label{question-16-3}}

If \(X\) is a topological space and \(S \subset X\), define in terms of
open subsets of \(X\) what it means for \(S\) \textbf{not} to be
connected.

Show that if \(S\) is not connected there are nonempty subsets
\(A, B \subset X\) such that \[
A \cup B = S \qtext{and} A \cap \bar B = \bar A \cap B = \emptyset
\]

\begin{quote}
Here \(\bar A\) and \(\bar B\) denote closure with respect to the
topology on the ambient space \(X\).
\end{quote}

\hypertarget{question-17-3}{%
\subsection{Question 17}\label{question-17-3}}

A topological space is \textbf{totally disconnected} if its only
connected subsets are one-point sets.

Is it true that if \(X\) has the discrete topology, it is totally
disconnected?

Is the converse true? Justify your answers.

\hypertarget{question-18-3}{%
\subsection{Question 18}\label{question-18-3}}

Prove that if \((X, d)\) is a compact metric space, \(f : X \to X\) is a
continuous map, and \(C\) is a constant with \(0 < C < 1\) such that \[
d(f (x), f (y)) \leq C \cdot d(x, y) \quad \forall x, y
,\] then \(f\) has a fixed point.

\hypertarget{question-19-3}{%
\subsection{Question 19}\label{question-19-3}}

Prove that the product of two connected topological spaces is connected.

\hypertarget{question-20-3}{%
\subsection{Question 20}\label{question-20-3}}

\begin{enumerate}
\def\labelenumi{\alph{enumi}.}
\item
  Define what it means for a topological space to be:

  \begin{enumerate}
  \def\labelenumii{\roman{enumii}.}
  \item
    \textbf{Connected}
  \item
    \textbf{Locally connected}
  \end{enumerate}
\item
  Give, with proof, an example of a space that is connected but not
  locally connected.
\end{enumerate}

\hypertarget{question-21-3}{%
\subsection{Question 21}\label{question-21-3}}

Let \(X\) and \(Y\) be topological spaces and let \(f : X \to Y\) be a
function.

Suppose that \(X = A \cup B\) where \(A\) and \(B\) are closed subsets,
and that the restrictions \(f \mid_A\) and \(f \mid_B\) are continuous
(where \(A\) and \(B\) have the subspace topology).

Prove that \(f\) is continuous.

\hypertarget{question-22-3}{%
\subsection{Question 22}\label{question-22-3}}

Let \(X\) be a compact space and let \(f : X \times R \to R\) be a
continuous function such that \(f (x, 0) > 0\) for all \(x \in X\).

Prove that there is \(\eps > 0\) such that \(f (x, t) > 0\) whenever
\(\abs t < \eps\).

Moreover give an example showing that this conclusion may not hold if
\(X\) is not assumed compact.

\hypertarget{question-23-3}{%
\subsection{Question 23}\label{question-23-3}}

Define a family \(\mct\) of subsets of \(\RR\) by saying that
\(A \in T\) is \(\iff A = \emptyset\) or \(\RR \setminus A\) is a finite
set.

Prove that \(\mct\) is a topology on \(\RR\), and that \(\RR\) is
compact with respect to this topology.

\hypertarget{question-24-3}{%
\subsection{Question 24}\label{question-24-3}}

In each part of this problem \(X\) is a compact topological space.

Give a proof or a counterexample for each statement.

\begin{enumerate}
\def\labelenumi{\alph{enumi}.}
\item
  If \(\theset{F_n }_{n=1}^\infty\) is a sequence of nonempty
  \emph{closed} subsets of \(X\) such that \(F_{n+1} \subset F_{n}\) for
  all \(n\) then \[\intersect^\infty_{n=1} F_n\neq \emptyset.\]
\item
  If \(\theset{O_n}_{n=1}^\infty\) is a sequence of nonempty \emph{open}
  subsets of \(X\) such that \(O_{n+1} \subset O_n\) for all \(n\) then
  \[\intersect_{n=1}^\infty O_{n}\neq \emptyset.\]
\end{enumerate}

\hypertarget{question-25-3}{%
\subsection{Question 25}\label{question-25-3}}

Let \(\mcs, \mct\) be topologies on a set \(X\). Show that
\(\mcs \cap \mct\) is a topology on \(X\).

Give an example to show that \(\mcs \cup \mct\) need not be a topology.

\hypertarget{question-26-3}{%
\subsection{Question 26}\label{question-26-3}}

Let \(f : X \to Y\) be a continuous function between topological spaces.

Let \(A\) be a subset of \(X\) and let \(f (A)\) be its image in \(Y\) .

One of the following statements is true and one is false. Decide which
is which, prove the true statement, and provide a counterexample to the
false statement:

\begin{enumerate}
\def\labelenumi{\arabic{enumi}.}
\item
  If \(A\) is closed then \(f (A)\) is closed.
\item
  If \(A\) is compact then \(f (A)\) is compact.
\end{enumerate}

\hypertarget{question-27-3}{%
\subsection{Question 27}\label{question-27-3}}

A metric space is said to be \textbf{totally bounded} if for every
\(\eps > 0\) there exists a finite cover of \(X\) by open balls of
radius \(\eps\).

\begin{enumerate}
\def\labelenumi{\alph{enumi}.}
\item
  Show: a metric space \(X\) is totally bounded iff every sequence in
  \(X\) has a Cauchy subsequence.
\item
  Exhibit a complete metric space \(X\) and a closed subset \(A\) of
  \(X\) that is bounded but not totally bounded.
\end{enumerate}

\begin{quote}
You are not required to prove that your example has the stated
properties.
\end{quote}

\hypertarget{question-28-3}{%
\subsection{Question 28}\label{question-28-3}}

Suppose that \(X\) is a Hausdorff topological space and that
\(A \subset X\).

Prove that if \(A\) is compact in the subspace topology then \(A\) is
closed as a subset of X.

\hypertarget{question-29-3}{%
\subsection{Question 29}\label{question-29-3}}

\begin{enumerate}
\def\labelenumi{\alph{enumi}.}
\item
  Show that a continuous bijection from a compact space to a Hausdorff
  space is a homeomorphism.
\item
  Give an example that shows that the ``Hausdorff'' hypothesis in part
  (a) is necessary.
\end{enumerate}

\hypertarget{question-30-3}{%
\subsection{Question 30}\label{question-30-3}}

Let \(X\) be a topological space and let \[
\Delta = \theset{(x, y) \in X \times X \mid x = y}
.\]

Show that \(X\) is a Hausdorff space if and only if \(\Delta\) is closed
in \(X \times X\).

\hypertarget{question-31-3}{%
\subsection{Question 31}\label{question-31-3}}

If \(f\) is a function from \(X\) to \(Y\) , consider the graph \[
G = \theset{(x, y) \in X \times Y \mid f (x) = y}
.\]

\begin{enumerate}
\def\labelenumi{\alph{enumi}.}
\item
  Prove that if \(f\) is continuous and \(Y\) is Hausdorff, then \(G\)
  is a closed subset of \(X \times Y\).
\item
  Prove that if \(G\) is closed and \(Y\) is compact, then \(f\) is
  continuous.
\end{enumerate}

\hypertarget{question-32-3}{%
\subsection{Question 32}\label{question-32-3}}

Let X be a noncompact locally compact Hausdorff space, with topology
\(\mct\). Let \(\tilde X = X \cup \theset{\infty}\) (\(X\) with one
point adjoined), and consider the family \(\mcb\) of subsets of
\(\tilde X\) defined by \[
\mcb = T \cup \theset{S \cup \theset{\infty}\mid S \subset X,~~ X \backslash S \text{ is compact}}
.\]

\begin{enumerate}
\def\labelenumi{\alph{enumi}.}
\item
  Prove that \(\mcb\) is a topology on \(\tilde X\), that the resulting
  space is compact, and that \(X\) is dense in \(\tilde X\).
\item
  Prove that if \(Y \supset X\) is a compact space such that \(X\) is
  dense in \(Y\) and \(Y \backslash X\) is a singleton, then Y is
  homeomorphic to \(\tilde X\).
\end{enumerate}

\begin{quote}
The space \(\tilde X\) is called the \textbf{one-point compactification}
of \(X\).
\end{quote}

\begin{enumerate}
\def\labelenumi{\alph{enumi}.}
\setcounter{enumi}{2}
\item
  Find familiar spaces that are homeomorphic to the one point
  compactifications of

  \begin{enumerate}
  \def\labelenumii{\roman{enumii}.}
  \tightlist
  \item
    \(X = (0, 1)\) and
  \end{enumerate}
\end{enumerate}

\hypertarget{question-33-3}{%
\subsection{Question 33}\label{question-33-3}}

Prove that a metric space \(X\) is \textbf{normal}, i.e.~if
\(A, B \subset X\) are closed and disjoint then there exist open sets
\(A \subset U \subset X, ~B \subset V \subset X\) such that
\(U \cap V = \emptyset\).

\hypertarget{question-34-3}{%
\subsection{Question 34}\label{question-34-3}}

Prove that every compact, Hausdorff topological space is normal.

\hypertarget{question-35-3}{%
\subsection{Question 35}\label{question-35-3}}

Show that a connected, normal topological space with more than a single
point is uncountable.

\hypertarget{question-36-3}{%
\subsection{Question 36}\label{question-36-3}}

Give an example of a quotient map in which the domain is Hausdorff, but
the quotient is not.

\hypertarget{question-37-3}{%
\subsection{Question 37}\label{question-37-3}}

Let \(X\) be a compact Hausdorff space and suppose
\(R \subset X \times X\) is a closed equivalence relation.

Show that the quotient space \(X/R\) is Hausdorff.

\hypertarget{question-38-3}{%
\subsection{Question 38}\label{question-38-3}}

Let \(U \subset \RR^n\) be an open set which is bounded in the standard
Euclidean metric.

Prove that the quotient space \(\RR^n / U\) is not Hausdorff.

\hypertarget{question-39-3}{%
\subsection{Question 39}\label{question-39-3}}

Let \(A\) be a closed subset of a normal topological space \(X\).

Show that both \(A\) and the quotient \(X/A\) are normal.

\hypertarget{question-40-3}{%
\subsection{Question 40}\label{question-40-3}}

Define an equivalence relation \(\sim\) on \(\RR\) by \(x \sim y\) if
and only if \(x - y \in Q\). Let \(X\) be the set of equivalence
classes, endowed with the quotient topology induced by the canonical
projection \(\pi : \RR \to X\).

Describe, with proof, all open subsets of \(X\) with respect to this
topology.

\hypertarget{question-41-3}{%
\subsection{Question 41}\label{question-41-3}}

Let \(A\) denote a subset of points of \(S^2\) that looks exactly like
the capital letter A. Let \(Q\) be the quotient of \(S^2\) given by
identifying all points of \(A\) to a single point.

Show that \(Q\) is homeomorphic to a familiar topological space and
identify that space.

\hypertarget{question-42-3}{%
\subsection{Question 42}\label{question-42-3}}

\begin{enumerate}
\def\labelenumi{\alph{enumi}.}
\item
  Prove that a topological space that has a countable base for its
  topology also contains a countable dense subset.
\item
  Prove that the converse to (a) holds if the space is a metric space.
\end{enumerate}

\hypertarget{question-43-3}{%
\subsection{Question 43}\label{question-43-3}}

Recall that a topological space is \textbf{regular} if for every point
\(p \in X\) and for every closed subset \(F \subset X\) not containing
\(p\), there exist disjoint open sets \(U, V \subset X\) with
\(p \in U\) and \(F \subset V\).

Let \(X\) be a regular space that has a countable basis for its
topology, and let \(U\) be an open subset of \(X\).

\begin{enumerate}
\def\labelenumi{\alph{enumi}.}
\item
  Show that \(U\) is a countable union of closed subsets of \(X\).
\item
  Show that there is a continuous function \(f : X \to [0,1]\) such that
  \(f (x) > 0\) for \(x \in U\) and \(f (x) = 0\) for \(x \in U\).
\end{enumerate}

\hypertarget{question-44-3}{%
\subsection{Question 44}\label{question-44-3}}

Let \(S^1\) denote the unit circle in \(C\), \(X\) be any topological
space, \(x_0 \in X\), and \[\gamma_0, \gamma_1 : S^1 \to X\] be two
continuous maps such that \(\gamma_0 (1) = \gamma_1 (1) = x_0\).

Prove that \(\gamma_0\) is homotopic to \(\gamma_1\) if and only if the
elements represented by \(\gamma_0\) and \(\gamma_1\) in
\(\pi_1 (X, x_0 )\) are conjugate.

\hypertarget{question-45-3}{%
\subsection{Question 45}\label{question-45-3}}

\begin{enumerate}
\def\labelenumi{\alph{enumi}.}
\item
  State van Kampen's theorem.
\item
  Calculate the fundamental group of the space obtained by taking two
  copies of the torus \(T = S^1 \times S^1\) and gluing them along a
  circle \(S^1 \times {p}\) where \(p\) is a point in \(S^1\).
\item
  Calculate the fundamental group of the Klein bottle.
\item
  Calculate the fundamental group of the one-point union of
  \(S^1 \times S^1\) and \(S^1\).
\item
  Calculate the fundamental group of the one-point union of
  \(S^1 \times S^1\) and \(\RP^2\).
\end{enumerate}

\begin{quote}
\textbf{Note: multiple appearances!!}
\end{quote}

\hypertarget{question-46-3}{%
\subsection{Question 46}\label{question-46-3}}

Prove the following portion of van Kampen's theorem. If \(X = A\cup B\)
and \(A\), \(B\), and \(A \cap B\) are nonempty and path connected with
\(\pt \in A \cap B\), then there is a surjection \[
\pi_1 (A, \pt) \ast \pi_1 (B, \pt) \to \pi_1 (X, \pt)
.\]

\hypertarget{question-47-3}{%
\subsection{Question 47}\label{question-47-3}}

Let \(X\) denote the quotient space formed from the sphere \(S^2\) by
identifying two distinct points.

Compute the fundamental group and the homology groups of \(X\).

\hypertarget{question-48-3}{%
\subsection{Question 48}\label{question-48-3}}

Start with the unit disk \(\DD^2\) and identify points on the boundary
if their angles, thought of in polar coordinates, differ a multiple of
\(\pi/2\).

Let \(X\) be the resulting space. Use van Kampen's theorem to compute
\(\pi_1 (X, \ast)\).

\hypertarget{question-49-3}{%
\subsection{Question 49}\label{question-49-3}}

Let \(L\) be the union of the \(z\)-axis and the unit circle in the
\(xy\dash\)plane. Compute \(\pi_1 (\RR^3 \backslash L, \ast)\).

\hypertarget{question-50-3}{%
\subsection{Question 50}\label{question-50-3}}

Let \(A\) be the union of the unit sphere in \(\RR^3\) and the interval
\(\theset {(t, 0, 0) : -1 \leq t \leq 1} \subset \RR^3\).

Compute \(\pi_1 (A)\) and give an explicit description of the universal
cover of \(X\).

\hypertarget{question-51-3}{%
\subsection{Question 51}\label{question-51-3}}

\begin{enumerate}
\def\labelenumi{\alph{enumi}.}
\item
  Let \(S_1\) and \(S_2\) be disjoint surfaces. Give the definition of
  their connected sum \(S^1 \#S^2\).
\item
  Compute the fundamental group of the connected sum of the projective
  plane and the two-torus.
\end{enumerate}

\hypertarget{question-52-3}{%
\subsection{Question 52}\label{question-52-3}}

Compute the fundamental group, using any technique you like, of
\(\RP^2 \#\RP^2 \#\RP^2\).

\hypertarget{question-53-3}{%
\subsection{Question 53}\label{question-53-3}}

Let \[
V = \DD^2 \times S^1 = \theset{ (z, e^{it}) \suchthat \norm z \leq 1,~~ 0 \leq t < 2\pi}
\] be the ``solid torus'' with boundary given by the torus
\(T = S^1 \times S^1\) .

For \(n \in Z\) define \begin{align*}
\phi_n : T &\to T \\
(e^{is} , e^{it} ) &\mapsto (e^{is} , e^{i(ns+t)})
.\end{align*}

Find the fundamental group of the identification space \[
V_n = {V\disjoint V \over \sim n}
\] where the equivalence relation \(\sim_n\) identifies a point \(x\) on
the boundary \(T\) of the first copy of \(V\) with the point
\(\phi_n (x)\) on the boundary of the second copy of \(V\).

\hypertarget{question-54-3}{%
\subsection{Question 54}\label{question-54-3}}

Let \(S_k\) be the space obtained by removing \(k\) disjoint open disks
from the sphere \(S^2\). Form \(X_k\) by gluing \(k\) Möbius bands onto
\(S_k\) , one for each circle boundary component of \(S_k\) (by
identifying the boundary circle of a Möbius band homeomorphically with a
given boundary component circle).

Use van Kampen's theorem to calculate \(\pi_1 (X_k)\) for each \(k > 0\)
and identify \(X_k\) in terms of the classification of surfaces.

\hypertarget{question-55-3}{%
\subsection{Question 55}\label{question-55-3}}

\begin{enumerate}
\def\labelenumi{\roman{enumi}.}
\item
  Let \(A\) be a subspace of a topological space \(X\). Define what it
  means for \(A\) to be a \textbf{deformation retract} of \(X\).
\item
  Consider \(X_1\) the ``planar figure eight'' and
  \[X_2 = S^1 \cup ({0} \times [-1, 1])\] (the ``theta space''). Show
  that \(X_1\) and \(X_2\) have isomorphic fundamental groups.
\item
  Prove that the fundamental group of \(X_2\) is a free group on two
  generators.
\end{enumerate}

\hypertarget{question-56-3}{%
\subsection{Question 56}\label{question-56-3}}

\begin{enumerate}
\def\labelenumi{\alph{enumi}.}
\item
  Give the definition of a \textbf{covering space} \(\hat{X}\) (and
  \textbf{covering map} \(p : \hat{X} \to X\)) for a topological space
  \(X\).
\item
  State the homotopy lifting property of covering spaces. Use it to show
  that a covering map \(p : \hat{X} \to X\) induces an injection \[
  p^\ast : \pi_1 (\hat{X}, \hat{x}) \to \pi_1 (X, p(\hat{x}))
  \] on fundamental groups.
\item
  Let \(p : \hat{X} \to X\) be a covering map with \(Y\) and \(X\)
  path-connected. Suppose that the induced map \(p^\ast\) on \(\pi_1\)
  is an isomorphism. Prove that \(p\) is a homeomorphism.
\end{enumerate}

\hypertarget{question-57-3}{%
\subsection{Question 57}\label{question-57-3}}

\begin{enumerate}
\def\labelenumi{\alph{enumi}.}
\item
  Give the definitions of \textbf{covering space} and \textbf{deck
  transformation} (or covering transformation).
\item
  Describe the universal cover of the Klein bottle and its group of deck
  transformations.
\item
  Explicitly give a collection of deck transformations on
  \[\theset{(x, y) \mid -1 \leq x \leq 1, -\infty < y < \infty}\] such
  that the quotient is a Möbius band.
\item
  Find the universal cover of \(\RP^2 \times S^1\) and explicitly
  describe its group of deck transformations.
\end{enumerate}

\hypertarget{question-58-3}{%
\subsection{Question 58}\label{question-58-3}}

\begin{enumerate}
\def\labelenumi{\alph{enumi}.}
\item
  What is the definition of a \textbf{regular} (or Galois) covering
  space?
\item
  State, without proof, a criterion in terms of the fundamental group
  for a covering map \(p : \tilde X \to X\) to be regular.
\item
  Let \(\Theta\) be the topological space formed as the union of a
  circle and its diameter (so this space looks exactly like the letter
  \(\Theta\)). Give an example of a covering space of \(\Theta\) that is
  not regular.
\end{enumerate}

\hypertarget{question-59-3}{%
\subsection{Question 59}\label{question-59-3}}

Let \(S\) be the closed orientable surface of genus 2 and let \(C\) be
the commutator subgroup of \(\pi_1 (S, \ast)\). Let \(\tilde S\) be the
cover corresponding to \(C\). Is the covering map \(\tilde S \to S\)
regular?

\begin{quote}
The term ``normal'' is sometimes used as a synonym for regular in this
context.
\end{quote}

What is the group of deck transformations?

Give an example of a nontrivial element of \(\pi_1 (S, \ast)\) which
lifts to a trivial deck transformation.

\hypertarget{question-60-3}{%
\subsection{Question 60}\label{question-60-3}}

Describe the 3-fold connected covering spaces of \(S^1 \lor S^1\).

\hypertarget{question-61-3}{%
\subsection{Question 61}\label{question-61-3}}

Find all three-fold covers of the wedge of two copies of \(\RP^2\) .
Justify your answer.

\hypertarget{question-62-3}{%
\subsection{Question 62}\label{question-62-3}}

Describe, as explicitly as you can, two different (non-homeomorphic)
connected two-sheeted covering spaces of \(\RP^2 \lor \RP^3\), and prove
that they are not homeomorphic.

\hypertarget{question-63-3}{%
\subsection{Question 63}\label{question-63-3}}

Is there a covering map from \[
X_3 = \theset{x^2 + y^2 = 1} \cup \theset{(x - 2)^2 + y^2 = 1} \cup \theset{(x + 2)^2 + y^2 = 1} \subset \RR^2
\] to the wedge of two \(S^1\)'s? If there is, give an example; if not,
give a proof.

\hypertarget{question-64-3}{%
\subsection{Question 64}\label{question-64-3}}

\begin{enumerate}
\def\labelenumi{\alph{enumi}.}
\item
  Suppose \(Y\) is an \(n\)-fold connected covering space of the torus
  \(S^1 \times S^1\). Up to homeomorphism, what is \(Y\)? Justify your
  answer.
\item
  Let \(X\) be the topological space obtained by deleting a disk from a
  torus. Suppose \(Y\) is a 3-fold covering space of \(X\).

  What surfaces could \(Y\) be? Justify your answer, but you need not
  exhibit the covering maps explicitly.
\end{enumerate}

\hypertarget{question-65-3}{%
\subsection{Question 65}\label{question-65-3}}

Let \(S\) be a connected surface, and let \(U\) be a connected open
subset of \(S\). Let \(p : \tilde S \to S\) be the universal cover of
\(S\). Show that \(p\inv (U )\) is connected if and only if the
homeomorphism \(i_\ast : \pi_1 (U ) \to \pi_1 (S)\) induced by the
inclusion \(i : U \to S\) is onto.

\hypertarget{question-66-3}{%
\subsection{Question 66}\label{question-66-3}}

Suppose that X has universal cover \(p : \tilde X \to X\) and let
\(A \subset X\) be a subspace with \(p(\tilde a) = a \in A\). Show that
there is a group isomorphism \[
\ker(\pi_1 (A, a) \to \pi_1 (X, a)) \cong \pi_1 (p\inv A, \bar a)
.\]

\hypertarget{question-67-3}{%
\subsection{Question 67}\label{question-67-3}}

Prove that every continuous map \(f : \RP^2 \to S^1\) is homotopic to a
constant.

\begin{quote}
Hint: think about covering spaces.
\end{quote}

\hypertarget{question-68-3}{%
\subsection{Question 68}\label{question-68-3}}

Prove that the free group on two generators contains a subgroup
isomorphic to the free group on five generators by constructing an
appropriate covering space of \(S^1 \lor S^1\).

\hypertarget{question-69-3}{%
\subsection{Question 69}\label{question-69-3}}

Use covering space theory to show that \(\ZZ_2 \ast \ZZ\) (that is, the
free product of \(\ZZ_2\) and \(\ZZ\)) has two subgroups of index 2
which are not isomorphic to each other.

\hypertarget{question-70-3}{%
\subsection{Question 70}\label{question-70-3}}

\begin{enumerate}
\def\labelenumi{\alph{enumi}.}
\item
  Show that any finite index subgroup of a finitely generated free group
  is free. State clearly any facts you use about the fundamental groups
  of graphs.
\item
  Prove that if \(N\) is a nontrivial normal subgroup of infinite index
  in a finitely generated free group \(F\) , then \(N\) is not finitely
  generated.
\end{enumerate}

\hypertarget{question-71-3}{%
\subsection{Question 71}\label{question-71-3}}

Let \(p : X \to Y\) be a covering space, where \(X\) is compact,
path-connected, and locally path-connected.

Prove that for each \(x \in X\) the set \(p\inv (\theset{p(x)})\) is
finite, and has cardinality equal to the index of \(p_* (\pi_1 (X, x))\)
in \(\pi_1 (Y, p(x))\).

\hypertarget{question-72-3}{%
\subsection{Question 72}\label{question-72-3}}

Compute the homology of the one-point union of \(S^1 \times S^1\) and
\(S^1\).

\hypertarget{question-73-3}{%
\subsection{Question 73}\label{question-73-3}}

\begin{enumerate}
\def\labelenumi{\alph{enumi}.}
\item
  State the \textbf{Mayer-Vietoris theorem}.
\item
  Use it to compute the homology of the space \(X\) obtained by gluing
  two solid tori along their boundary as follows. Let \(\DD^2\) be the
  unit disk and let \(S^1\) be the unit circle in the complex plane
  \(\CC\). Let \(A = S^1 \times \DD^2\) and \(B = \DD^2 \times S^1\).

  Then \(X\) is the quotient space of the disjoint union
  \(A \disjoint B\) obtained by identifying \((z, w) \in A\) with
  \((zw^3 , w) \in B\) for all \((z, w) \in S^1 \times S^1\).
\end{enumerate}

\hypertarget{question-74-3}{%
\subsection{Question 74}\label{question-74-3}}

Let \(A\) and \(B\) be circles bounding disjoint disks in the plane
\(z = 0\) in \(\RR^3\). Let \(X\) be the subset of the upper half-space
of \(\RR^3\) that is the union of the plane \(z = 0\) and a
(topological) cylinder that intersects the plane in
\(\partial C = A \cup B\).

Compute \(H_* (X)\) using the Mayer--Vietoris sequence.

\hypertarget{question-75-3}{%
\subsection{Question 75}\label{question-75-3}}

Compute the integral homology groups of the space \(X = Y \cup Z\) which
is the union of the sphere \[
Y = \theset{x^2 + y^2 + z^2 = 1}
\] and the ellipsoid \[
Z =  \theset{x^2 + y^2 + {z^2 \over 4} = 1}
.\]

\hypertarget{question-76-3}{%
\subsection{Question 76}\label{question-76-3}}

Let \(X\) consist of two copies of the solid torus \(\DD^2 \times S^1\),
glued together by the identity map along the boundary torus
\(S^1 \times S^1\). Compute the homology groups of \(X\).

\hypertarget{question-77-3}{%
\subsection{Question 77}\label{question-77-3}}

Use the circle along which the connected sum is performed and the
Mayer-Vietoris long exact sequence to compute the homology of
\(\RP^2 \# \RP^2\).

\hypertarget{question-78-3}{%
\subsection{Question 78}\label{question-78-3}}

Express a Klein bottle as the union of two annuli.

Use the Mayer Vietoris sequence and this decomposition to compute its
homology.

\hypertarget{question-79-3}{%
\subsection{Question 79}\label{question-79-3}}

Let \(X\) be the topological space obtained by identifying three
distinct points on \(S^2\). Calculate \(H_* (X; Z)\).

\hypertarget{question-80-3}{%
\subsection{Question 80}\label{question-80-3}}

Compute \(H_0\) and \(H_1\) of the complete graph \(K_5\) formed by
taking five points and joining each pair with an edge.

\hypertarget{question-81-3}{%
\subsection{Question 81}\label{question-81-3}}

Compute the homology of the subset \(X \subset \RR^3\) formed as the
union of the unit sphere, the \(z\dash\)axis, and the \(xy\dash\)plane.

\hypertarget{question-82-3}{%
\subsection{Question 82}\label{question-82-3}}

Let \(X\) be the topological space formed by filling in two circles
\(S^1 \times \theset{p_1 }\) and \(S^1 \times \theset{p_2 }\) in the
torus \(S^1 \times S^1\) with disks.

Calculate the fundamental group and the homology groups of \(X\).

\hypertarget{question-83-3}{%
\subsection{Question 83}\label{question-83-3}}

\begin{enumerate}
\def\labelenumi{\alph{enumi}.}
\item
  Consider the quotient space \[
  T^2 = \RR^2 / \sim \qtext{where} (x, y) \sim (x + m, y + n) \text{ for } m, n \in \ZZ
  ,\] and let \(A\) be any \(2 \times 2\) matrix whose entries are
  integers such that \(\det A = 1\).

  Prove that the action of \(A\) on \(\RR^2\) descends via the quotient
  \(\RR^2 \to T^2\) to induce a homeomorphism \(T^2 \to T^2\).
\item
  Using this homeomorphism of \(T^2\), we define a new quotient space \[
  T_A^3 \definedas {T^2\cross \RR \over \sim} \qtext{where} ((x, y), t) \sim (A(x, y), t + 1)
  \] Compute \(H_1 (T_A^3 )\) if
  \(A=\left(\begin{array}{ll} 1 & 1 \\ 0 & 1 \end{array}\right).\)
\end{enumerate}

\hypertarget{question-84-3}{%
\subsection{Question 84}\label{question-84-3}}

Give a self-contained proof that the zeroth homology \(H_0 (X)\) is
isomorphic to \(\ZZ\) for every path-connected space \(X\).

\hypertarget{question-85-3}{%
\subsection{Question 85}\label{question-85-3}}

Give a self-contained proof that the zeroth homology \(H_0 (X)\) is
isomorphic to \(\ZZ\) for every path-connected space \(X\).

\hypertarget{question-86-2}{%
\subsection{Question 86}\label{question-86-2}}

It is a fact that if \(X\) is a single point then
\(H_1 (X) = \theset{0}\).

One of the following is the correct justification of this fact in terms
of the singular chain complex.

Which one is correct and why is it correct?

\begin{enumerate}
\def\labelenumi{\alph{enumi}.}
\item
  \(C_1 (X) = \theset{0}\).
\item
  \(C_1 (X) \neq \theset{0}\) but \(\ker \partial_1 = 0\) with
  \(\partial_1 : C_1 (X) \to C_0 (X)\).
\item
  \(\ker \partial_1 \neq 0\) but \(\ker \partial_1 = \im\partial_2\)
  with \(\partial_2 : C_2 (X) \to C_1 (X)\).
\end{enumerate}

\hypertarget{question-87-2}{%
\subsection{Question 87}\label{question-87-2}}

Compute the homology groups of \(S^2 \times S^2\).

\hypertarget{question-88-2}{%
\subsection{Question 88}\label{question-88-2}}

Let \(\Sigma\) be a closed orientable surface of genus \(g\). Compute
\(H_i(S^1 \times \Sigma; Z)\) for \(i = 0, 1, 2, 3\).

\hypertarget{question-89-2}{%
\subsection{Question 89}\label{question-89-2}}

Prove that if \(A\) is a retract of the topological space \(X\), then
for all nonnegative integers \(n\) there is a group \(G_n\) such that
\(H_{n} (X) \cong H_{n} (A) \oplus G_n\).

\begin{quote}
Here \(H_{n}\) denotes the \(n\)th singular homology group with integer
coefficients.
\end{quote}

\hypertarget{question-90-2}{%
\subsection{Question 90}\label{question-90-2}}

Does there exist a map of degree 2013 from \(S^2 \to S^2\).

\hypertarget{question-91-2}{%
\subsection{Question 91}\label{question-91-2}}

For each \(n \in \ZZ\) give an example of a map \(f_n : S^2 \to S^2\).

For which \(n\) must any such map have a fixed point?

\hypertarget{question-92-2}{%
\subsection{Question 92}\label{question-92-2}}

\begin{enumerate}
\def\labelenumi{\alph{enumi}.}
\item
  What is the degree of the antipodal map on the \(n\)-sphere? (No
  justification required)
\item
  Define a CW complex homeomorphic to the real projective
  \(n\dash\)space \(\RP^n\).
\item
  Let \(\pi : \RP^n \to X\) be a covering map. Show that if \(n\) is
  even, \(\pi\) is a homeomorphism.
\end{enumerate}

\hypertarget{question-93-2}{%
\subsection{Question 93}\label{question-93-2}}

Let \(A \subset X\). Prove that the relative homology group
\(H_0 (X, A)\) is trivial if and only if \(A\) intersects every path
component of \(X\).

\hypertarget{question-94-2}{%
\subsection{Question 94}\label{question-94-2}}

Let \(\DD\) be a closed disk embedded in the torus
\(T = S^1 \times S^1\) and let \(X\) be the result of removing the
interior of \(\DD\) from \(T\) . Let \(B\) be the boundary of \(X\),
i.e.~the circle boundary of the original closed disk \(\DD\).

\hypertarget{question-95-2}{%
\subsection{Question 95}\label{question-95-2}}

Let \(\DD\) be a closed disk embedded in the torus
\(T = S^1 \times S^1\) and let \(X\) be the result of removing the
interior of \(\DD\) from \(T\) . Let \(B\) be the boundary of \(X\),
i.e.~the circle boundary of the original closed disk \(\DD\).

Compute \(H_i (T, B)\) for all \(i\).

\hypertarget{question-96-2}{%
\subsection{Question 96}\label{question-96-2}}

For any \(n \geq 1\) let
\(S^n = \theset{(x_0 , \cdots , x_n )\mid \sum x_i^2 =1}\) denote the
\(n\) dimensional unit sphere and let
\[E = \theset{(x_0 , . . . , x_n )\mid x_n = 0}\] denote the
``equator''.

Find, for all \(k\), the relative homology \(H_k (S^n , E)\).

\hypertarget{question-97-2}{%
\subsection{Question 97}\label{question-97-2}}

Suppose that \(U\) and \(V\) are open subsets of a space \(X\), with
\(X = U \cup V\). Find, with proof, a general formula relating the Euler
characteristics of \(X, U, V\), and \(U \cap V\).

\begin{quote}
You may assume that the homologies of \(U, V, U \cap V, X\) are
finite-dimensional so that their Euler characteristics are well defined.
\end{quote}

\hypertarget{question-98-2}{%
\subsection{Question 98}\label{question-98-2}}

Describe a cell complex structure on the torus \(T = S^1 \times S^1\)
and use this to compute the homology groups of \(T\).

\begin{quote}
To justify your answer you will need to consider the attaching maps in
detail.
\end{quote}

\hypertarget{question-99-2}{%
\subsection{Question 99}\label{question-99-2}}

Let \(X\) be the space formed by identifying the boundary of a Möbius
band with a meridian of the torus \(T^2\).

Compute \(\pi_1 (X)\) and \(H_* (X)\).

\hypertarget{question-100-2}{%
\subsection{Question 100}\label{question-100-2}}

Compute the homology of the space \(X\) obtained by attaching a Möbius
band to \(\RP^2\) via a homeomorphism of its boundary circle to the
standard \(\RP^1\) in \(\RP^2\).

\hypertarget{question-101-2}{%
\subsection{Question 101}\label{question-101-2}}

Let \(X\) be a space obtained by attaching two 2-cells to the torus
\(S^1 \times S^1\), one along a simple closed curve
\(\theset{x} \times S^1\) and the other along \(\theset{y} \times S^1\)
for two points \(x \neq y\) in \(S^1\) .

\begin{enumerate}
\def\labelenumi{\alph{enumi}.}
\item
  Draw an embedding of \(X\) in \(\RR^3\) and calculate its fundamental
  group.
\item
  Calculate the homology groups of \(X\).
\end{enumerate}

\hypertarget{question-102-2}{%
\subsection{Question 102}\label{question-102-2}}

Let \(X\) be the space obtained as the quotient of a disjoint union of a
2-sphere \(S^2\) and a torus \(T = S^1 \times S^1\) by identifying the
equator in \(S^2\) with a circle \(S^1 \times \theset{p}\) in \(T\).

Compute the homology groups of \(X\).

\hypertarget{question-103-2}{%
\subsection{Question 103}\label{question-103-2}}

Let \(X = S^2 / \theset{p_1 = \cdots = p_k }\) be the topological space
obtained from the 2-sphere by identifying \(k\) distinct points on it
(\(k \geq 2\)).

Find:

\begin{enumerate}
\def\labelenumi{\alph{enumi}.}
\item
  The fundamental group of \(X\).
\item
  The Euler characteristic of \(X\).
\item
  The homology groups of \(X\).
\end{enumerate}

\hypertarget{question-104-2}{%
\subsection{Question 104}\label{question-104-2}}

Let \(X\) be the topological space obtained as the quotient of the
sphere
\(S^2 = \theset{\vector x \in \RR^3 \suchthat \norm{\vector x} = 1}\)
under the equivalence relation \(\vector x \sim -\vector x\) for
\(\vector x\) in the equatorial circle, i.e.~for
\(\vector x = (x_1, x_2, 0)\).

Calculate \(H_* (X; \ZZ)\) from a CW complex description of \(X\).

\hypertarget{question-105-2}{%
\subsection{Question 105}\label{question-105-2}}

Compute, by any means available, the fundamental group and all the
homology groups of the space obtained by gluing one copy \(A\) of
\(S^2\) to another copy \(B\) of \(S^2\) via a two-sheeted covering
space map from the equator of \(A\) onto the equator of \(B\).

\hypertarget{question-106-2}{%
\subsection{Question 106}\label{question-106-2}}

Use cellular homology to calculate the homology groups of
\(S^n \times S^m\).

\hypertarget{question-107-2}{%
\subsection{Question 107}\label{question-107-2}}

Denote the points of \(S^1 \times I\) by \((z, t)\) where \(z\) is a
unit complex number and \(0 \leq t \leq 1\). Let \(X\) denote the
quotient of \(S^1 \times I\) given by identifying \((z, 1)\) and
\((z_2 , 0)\) for all \(z \in S^1\).

Give a cell structure, with attaching maps, for \(X\), and use it to
compute \(\pi_1 (X, \ast)\) and \(H_1 (X)\).

\hypertarget{question-108-2}{%
\subsection{Question 108}\label{question-108-2}}

Let \(X = S_1 \cup S_2 \subset \RR^3\) be the union of two spheres of
radius 2, one about \((1, 0, 0)\) and the other about \((-1, 0, 0)\),
i.e.~ \begin{align*}
S_1 &= \theset{(x, y,z) \mid (x-1)^2 + y^2 +z^2 = 4} \\
S_2 &= \theset{(x, y, z) \mid (x + 1)^2 + y^2 + z^2 = 4}
.\end{align*}

\begin{enumerate}
\def\labelenumi{\alph{enumi}.}
\item
  Give a description of \(X\) as a CW complex.
\item
  Write out the cellular chain complex of \(X\).
\item
  Calculate \(H_* (X; Z)\).
\end{enumerate}

\hypertarget{question-109-2}{%
\subsection{Question 109}\label{question-109-2}}

Let \(M\) and \(N\) be finite CW complexes.

\begin{enumerate}
\def\labelenumi{\alph{enumi}.}
\item
  Describe a cellular structure of \(M \times N\) in terms of the
  cellular structures of \(M\) and \(N\).
\item
  Show that the Euler characteristic of \(M \times N\) is the product of
  the Euler characteristics of \(M\) and \(N\).
\end{enumerate}

\hypertarget{question-110-2}{%
\subsection{Question 110}\label{question-110-2}}

Suppose the space \(X\) is obtained by attaching a 2-cell to the torus
\(S^1 \times S^1\).

In other words, \(X\) is the quotient space of the disjoint union of the
closed disc \(\DD^2\) and the torus \(S^1 \times S^1\) by the
identification \(x \sim f(x)\) where \(S^1\) is the boundary of the unit
disc and \(f : S^1 \to S^1 \times S^1\) is a continuous map.

What are the possible homology groups of \(X\)? Justify your answer.

\hypertarget{question-111-2}{%
\subsection{Question 111}\label{question-111-2}}

Let \(X\) be the topological space constructed by attaching a closed
2-disk \(\DD^2\) to the circle \(S^1\) by a continuous map
\(\partial\DD^2 \to S^1\) of degree \(d > 0\) on the boundary circle.

\begin{enumerate}
\def\labelenumi{\alph{enumi}.}
\item
  Show that every continuous map \(X \to X\) has a fixed point.
\item
  Explain how to obtain all the connected covering spaces of \(X\).
\end{enumerate}

\hypertarget{question-112-2}{%
\subsection{Question 112}\label{question-112-2}}

Let \(X\) be a topological space obtained by attaching a 2-cell to
\(\RP^2\) via some map \(f: S^1 \to \RP^2\) .

What are the possibilities for the homology \(H_* (X; Z)\)?

\hypertarget{question-113-2}{%
\subsection{Question 113}\label{question-113-2}}

For any integer \(n \geq 2\) let \(X_n\) denote the space formed by
attaching a 2-cell to the circle \(S^1\) via the attaching map
\begin{align*}
a_n: S^1 &\to S^1 \\
e^{i\theta} &\mapsto e^{in\theta}
.\end{align*}

\begin{enumerate}
\def\labelenumi{\alph{enumi}.}
\item
  Compute the fundamental group and the homology of \(X_n\).
\item
  Exactly one of the \(X_n\) (for \(n \geq 2\)) is homeomorphic to a
  surface. Identify, with proof, both this value of \(n\) and the
  surface that \(X_n\) is homeomorphic to (including a description of
  the homeomorphism).
\end{enumerate}

\hypertarget{question-114-2}{%
\subsection{Question 114}\label{question-114-2}}

Let \(X\) be a CW complex and let \(\pi : Y \to X\) be a covering space.

\begin{enumerate}
\def\labelenumi{\alph{enumi}.}
\item
  Show that \(Y\) is compact iff \(X\) is compact and \(\pi\) has finite
  degree.
\item
  Assume that \(\pi\) has finite degree \(d\). Show show that
  \(\chi (Y ) = d \chi (X)\).
\item
  Let \(\pi :\RP^N \to X\) be a covering map. Show that if \(N\) is
  even, \(\pi\) is a homeomorphism.
\end{enumerate}

\hypertarget{question-115-2}{%
\subsection{Question 115}\label{question-115-2}}

For topological spaces \(X, Y\) the \textbf{mapping cone} \(C(f )\) of a
map \(f : X \to Y\) is defined to be the quotient space \begin{align*}
(X \times [0, 1])\disjoint Y / \sim &\qtext{where}  \\ 
(x, 0) &\sim (x', 0) \qtext{for all} x, x' \in X \text{ and } \\ 
(x, 1) &\sim f (x) \qtext{for all } x \in X
.\end{align*}

Let \(\phi_k : S^n \to S^n\) be a degree \(k\) map for some integer
\(k\).

Find \(H_i(C(\phi_k ))\) for all \(i\).

\hypertarget{question-116-2}{%
\subsection{Question 116}\label{question-116-2}}

Prove that a finite CW complex must be Hausdorff.

\hypertarget{question-117-2}{%
\subsection{Question 117}\label{question-117-2}}

State the classification theorem for surfaces (compact, without
boundary, but not necessarily orientable). For each surface in the
classification, indicate the structure of the first homology group and
the value of the Euler characteristic.

Also, explain briefly how the 2-holed torus and the connected sum
\(\RP^2 \# \RP^2\) fit into the classification.

\hypertarget{question-118-2}{%
\subsection{Question 118}\label{question-118-2}}

Give a list without repetitions of all compact surfaces (orientable or
non-orientable and with or without boundary) that have Euler
characteristic negative one.

Explain why there are no repetitions on your list.

\hypertarget{question-119-2}{%
\subsection{Question 119}\label{question-119-2}}

Describe the topological classification of all compact connected
surfaces \(M\) without boundary having Euler characteristic
\(\chi(M )\geq -2\).

No proof is required.

\hypertarget{question-120-2}{%
\subsection{Question 120}\label{question-120-2}}

How many surfaces are there, up to homeomorphism, which are:

\begin{itemize}
\tightlist
\item
  Connected,
\item
  Compact,
\item
  Possibly with boundary,
\item
  Possibly nonorientable, and
\item
  With Euler characteristic -3?
\end{itemize}

Describe one representative from each class.

\hypertarget{question-121-2}{%
\subsection{Question 121}\label{question-121-2}}

Prove that the Euler characteristic of a compact surface with boundary
which has \(k\) boundary components is less than or equal to \(2 - k\).

\hypertarget{question-122-2}{%
\subsection{Question 122}\label{question-122-2}}

Let \(X\) be the topological space obtained as the quotient space of a
regular \(2n\dash\)gon (\(n \geq 2\)) in \(\RR^2\) by identifying
opposite edges via translations in the plane.

First show that X is a compact, orientable surface without boundary, and
then identify its genus as a function of \(n\).

\hypertarget{question-123-2}{%
\subsection{Question 123}\label{question-123-2}}

\begin{enumerate}
\def\labelenumi{\alph{enumi}.}
\item
  Show that any compact connected surface with nonempty boundary is
  homotopy equivalent to a wedge of circles

  \begin{quote}
  Hint: you may assume that any compact connected surface without
  boundary is given by identifying edges of a polygon in pairs.
  \end{quote}
\item
  For each surface appearing in the classification of compact surfaces
  with nonempty boundary, say how many circles are needed in the wedge
  from part (a).

  \begin{quote}
  Hint: you should be able to do this even if you have not done part
  (a).
  \end{quote}
\end{enumerate}

\hypertarget{question-124-2}{%
\subsection{Question 124}\label{question-124-2}}

Let \(M_g^2\) be the compact oriented surface of genus \(g\).

Show that there exists a continuous map \(f : M_g^2 \to S^2\) which is
not homotopic to a constant map.

\hypertarget{question-125-2}{%
\subsection{Question 125}\label{question-125-2}}

Show that \(\RP^2 \lor S^1\) is \emph{not} homotopy equivalent to a
compact surface (possibly with boundary).

\hypertarget{question-126-1}{%
\subsection{Question 126}\label{question-126-1}}

Identify (with proof, but of course you can appeal to the classification
of surfaces) all of the compact surfaces without boundary that have a
cell decomposition having exactly one 0-cell and exactly two 1-cells
(with no restriction on the number of cells of dimension larger than 1).

\hypertarget{question-127-1}{%
\subsection{Question 127}\label{question-127-1}}

For any natural number \(g\) let \(\Sigma_g\) denote the (compact,
orientable) surface of genus \(g\).

Determine, with proof, all valued of \(g\) with the property that there
exists a covering space \(\pi : \Sigma_5 \to \Sigma_g\) .

\begin{quote}
Hint: How does the Euler characteristic behave for covering spaces?
\end{quote}

\hypertarget{question-128-1}{%
\subsection{Question 128}\label{question-128-1}}

Find \emph{all} surfaces, orientable and non-orientable, which can be
covered by a closed surface (i.e.~compact with empty boundary) of genus
2. Prove that your answer is correct.

\hypertarget{question-129-1}{%
\subsection{Question 129}\label{question-129-1}}

\begin{enumerate}
\def\labelenumi{\alph{enumi}.}
\item
  Write down (without proof) a presentation for \(\pi_1 (\Sigma_2 , p)\)
  where \(\Sigma_2\) is a closed, connected, orientable genus 2 surface
  and \(p\) is any point on \(\Sigma_2\) .
\item
  Show that \(\pi_1 (\Sigma_2 , p)\) is not abelian by showing that it
  surjects onto a free group of rank 2.
\item
  Show that there is no covering space map from \(\Sigma_2\) to
  \(S^1 \times S^1\) . You may use the fact that
  \(\pi_1 (S^1 \times S^1 ) \cong \ZZ^2\) together with the result in
  part (b) above.
\end{enumerate}

\hypertarget{question-130-1}{%
\subsection{Question 130}\label{question-130-1}}

Give an example, with explanation, of a closed curve in a surfaces which
is not nullhomotopic but is nullhomologous.

\hypertarget{question-131-1}{%
\subsection{Question 131}\label{question-131-1}}

Let \(M\) be a compact orientable surface of genus \(2\) without
boundary.

Give an example of a pair of loops \[\gamma_0 , \gamma_1 : S^1 \to M\]
with \(\gamma_0 (1) = \gamma_1 (1)\) such that there is a continuous map
\(\Gamma: [0, 1] \times S^1 \to M\) such that \[
\Gamma(0, t) = \gamma_0 (t), \quad \Gamma(1, t) = \gamma_1 (t) \qtext{for all} t \in S^1
,\] but such that there is no such map \(\Gamma\) with the additional
property that \(\Gamma_s (1) = \gamma_0 (1)\) for all \(s \in [0, 1]\).

(You are not required to prove that your example satisfies the stated
property.)

\hypertarget{question-132-1}{%
\subsection{Question 132}\label{question-132-1}}

Let \(C\) be cylinder. Let \(I\) and \(J\) be disjoint closed intervals
contained in \(\partial C\).

What is the Euler characteristic of the surface \(S\) obtained by
identifying \(I\) and \(J\)?

Can all surface with nonempty boundary and with this Euler
characteristic be obtained from this construction?

\hypertarget{question-133-1}{%
\subsection{Question 133}\label{question-133-1}}

Let \(\Sigma\) be a compact connected surface and let
\(p_1, \cdots , p_k \in \Sigma\).

Prove that \(H_2 \qty{\Sigma \setminus \union_{i=1}^k {p_i} } = 0\).

\hypertarget{question-134-1}{%
\subsection{Question 134}\label{question-134-1}}

Prove or disprove:

Every continuous map from \(S^2\) to \(S^2\) has a fixed point.

\hypertarget{question-135-1}{%
\subsection{Question 135}\label{question-135-1}}

\begin{enumerate}
\def\labelenumi{\alph{enumi}.}
\item
  State the \textbf{Lefschetz Fixed Point Theorem} for a finite
  simplicial complex \(X\).
\item
  Use degree theory to prove this theorem in case \(X = S^n\).
\end{enumerate}

\hypertarget{question-136-1}{%
\subsection{Question 136}\label{question-136-1}}

\begin{enumerate}
\def\labelenumi{\alph{enumi}.}
\tightlist
\item
  Prove that for every continuous map \(f : S^2 \to S^2\) there is some
  \(x\) such that either \(f (x) = x\) or \(f (x) = -x\).
\end{enumerate}

\begin{quote}
Hint: Where \(A : S^2 \to S^2\) is the antipodal map, you are being
asked to prove that either \(f\) or \(A \circ f\) has a fixed point.
\end{quote}

\begin{enumerate}
\def\labelenumi{\alph{enumi}.}
\setcounter{enumi}{1}
\tightlist
\item
  Exhibit a continuous map \(f : S^3 \to S^3\) such that for every
  \(x \in S^3\), \(f (x)\) is equal to neither \(x\) nor \(-x\).
\end{enumerate}

\begin{quote}
Hint: It might help to first think about how you could do this for a map
from \(S^1\) to \(S^1\).
\end{quote}

\hypertarget{question-137-1}{%
\subsection{Question 137}\label{question-137-1}}

Show that a map \(S^n \to S^n\) has a fixed point unless its degree is
equal to the degree of the antipodal map \(a : x \to -x\).

\hypertarget{question-138-1}{%
\subsection{Question 138}\label{question-138-1}}

Give an example of a homotopy class of maps of \(S^1 \lor S^1\) each
member of which must have a fixed point, and also an example of a map of
\(S^1 \lor S^1\) which doesn't have a fixed point.

\hypertarget{question-139-1}{%
\subsection{Question 139}\label{question-139-1}}

Prove or disprove:

Every map from \(\RP^2 \lor \RP^2\) to itself has a fixed point.

\hypertarget{question-140-1}{%
\subsection{Question 140}\label{question-140-1}}

Find all homotopy classes of maps from \(S^1 \times \DD^2\) to itself
such that every element of the homotopy class has a fixed point.

\hypertarget{question-141}{%
\subsection{Question 141}\label{question-141}}

Let \(X\) and \(Y\) be finite connected simplicial complexes and let
\(f : X \to Y\) and \(g : Y \to X\) be basepoint-preserving maps.

Show that no matter how you homotope
\(f \lor g : X \lor Y \to X \lor Y\), there will always be a fixed
point.

\hypertarget{question-142}{%
\subsection{Question 142}\label{question-142}}

Let \(f = \id_{\RP^2} \lor \ast\) and \(g = \ast \lor id_{S^1}\) be two
maps of \(\RP^2 \lor S^1\) to itself where \(\ast\) denotes the constant
map of a space to its basepoint.

Show that one map is homotopic to a map with no fixed points, while the
other is not.

\hypertarget{question-143}{%
\subsection{Question 143}\label{question-143}}

View the torus \(T\) as the quotient space \(\RR^2 /\ZZ^2\).

Let \(A\) be a \(2 \times 2\) matrix with \(\ZZ\) coefficients.

\begin{enumerate}
\def\labelenumi{\alph{enumi}.}
\item
  Show that the linear map \(A : \RR^2 \to \RR^2\) descends to a
  continuous map \(\mca : T \to T\).
\item
  Show that, with respect to a suitable basis for \(H_1 (T ; \ZZ)\), the
  matrix \(A\) represents the map induced on \(H_1\) by \(\mca\).
\item
  Find a necessary and sufficient condition on \(A\) for \(\mca\) to be
  homotopic to the identity.
\item
  Find a necessary and sufficient condition on \(A\) for \(\mca\) to be
  homotopic to a map with no fixed points.
\end{enumerate}

\hypertarget{question-144}{%
\subsection{Question 144}\label{question-144}}

\begin{enumerate}
\def\labelenumi{\alph{enumi}.}
\item
  Use the Lefschetz fixed point theorem to show that any degree-one map
  \(f : S^2 \to S^2\) has at least one fixed point.
\item
  Give an example of a map \(f : \RR^2 \to \RR^2\) having no fixed
  points.
\item
  Give an example of a degree-one map \(f : S^2 \to S^2\) having exactly
  one fixed point.
\end{enumerate}

\hypertarget{question-145}{%
\subsection{Question 145}\label{question-145}}

For which compact connected surfaces \(\Sigma\) (with or without
boundary) does there exist a continuous map \(f : \Sigma \to \Sigma\)
that is homotopic to the identity and has no fixed point?

Explain your answer fully.

\hypertarget{question-146}{%
\subsection{Question 146}\label{question-146}}

Use the Brouwer fixed point theorem to show that an \(n \times n\)
matrix with nonnegative entries has a real eigenvalue.

\hypertarget{question-147}{%
\subsection{Question 147}\label{question-147}}

Prove that \(\RR^2\) is not homeomorphic to \(\RR^n\) for \(n > 2\).

\hypertarget{question-148}{%
\subsection{Question 148}\label{question-148}}

Prove that any finite tree is contractible, where a \textbf{tree} is a
connected graph that contains no closed edge paths.

\hypertarget{question-149}{%
\subsection{Question 149}\label{question-149}}

Show that any continuous map \(f : \RP^2 \to S^1 \times S^1\) is
necessarily null-homotopic.

\hypertarget{question-150}{%
\subsection{Question 150}\label{question-150}}

Prove that, for \(n \geq 2\), every continuous map \(f: \RP^n \to S^1\)
is null-homotopic.

\hypertarget{question-151}{%
\subsection{Question 151}\label{question-151}}

Let \(S^2 \to \RP^2\) be the universal covering map.

Is this map null-homotopic? Give a proof of your answer.

\hypertarget{question-152}{%
\subsection{Question 152}\label{question-152}}

Suppose that a map \(f : S^3 \times S^3 \to \RP^3\) is not surjective.

Prove that \(f\) is homotopic to a constant function.

\hypertarget{question-153}{%
\subsection{Question 153}\label{question-153}}

Prove that there does not exist a continuous map \(f : S^2 \to S^2\)
from the unit sphere in \(\RR^3\) to itself such that
\(f (\vector x) \perp \vector x\) (as vectors in \(\RR^3\) for all
\(\vector x \in S^2\)).

\hypertarget{question-154}{%
\subsection{Question 154}\label{question-154}}

Let \(f\) be the map of \(S^1 \times [0, 1]\) to itself defined by \[
f (e^{i\theta} , s) = (e^{i(\theta+2\pi s)} , s)
,\] so that \(f\) restricts to the identity on the two boundary circles
of \(S^1 \times [0, 1]\).

Show that \(f\) is homotopic to the identity by a homotopy \(f_t\) that
is stationary on one of the boundary circles, but not by any homotopy
that is stationary on both boundary circles.

\begin{quote}
Hint: Consider what \(f\) does to the path
\(s \mapsto (e^{i\theta_0} , s)\) for fixed \(e^{i\theta_0} \in S^1\).
\end{quote}

\hypertarget{question-155}{%
\subsection{Question 155}\label{question-155}}

Show that \(S^1 \times S^1\) is not the union of two disks (where there
is no assumption that the disks intersect along their boundaries).

\hypertarget{question-156}{%
\subsection{Question 156}\label{question-156}}

Suppose that \(X \subset Y\) and \(X\) is a deformation retract of
\(Y\).

Show that if \(X\) is a path connected space, then \(Y\) is path
connected.

\hypertarget{question-157}{%
\subsection{Question 157}\label{question-157}}

Do one of the following:

\begin{enumerate}
\def\labelenumi{\alph{enumi}.}
\item
  Give (with justification) a contractible subset \(X \subset \RR^2\)
  which is not a retract of \(\RR^2\) .
\item
  Give (with justification) two topological spaces that have the same
  homology groups but that are not homotopy equivalent.
\end{enumerate}

\hypertarget{question-158}{%
\subsection{Question 158}\label{question-158}}

Recall that the \textbf{suspension} of a topological space, denoted
\(SX\), is the quotient space formed from \(X \times [-1, 1]\) by
identifying \((x, 1)\) with \((y, 1)\) for all \(x, y \in X\), and also
identifying \((x, -1)\) with \((y, -1)\) for all \(x, y \in X\).

\begin{enumerate}
\def\labelenumi{\alph{enumi}.}
\item
  Show that \(SX\) is the union of two contractible subspaces.
\item
  Prove that if \(X\) is path-connected then
  \(\pi_1 (SX) = \theset{0}\).
\item
  For all \(n \geq 1\), prove that \(H_{n} (X) \cong H_{n+1} (SX)\).
\end{enumerate}

\end{document}
